\documentclass[11pt, a4paper, norsk]{NTNUoving}
\usepackage[utf8]{inputenc}
\usepackage[T1]{fontenc}
\usepackage{tikz}

\ovingnr{3}    % Nummer på innlevering
\semester{Høsten 2019}
\fag{TMA 4115} %4110 er matte 2, 4115 er matte 3
\institutt{Institutt for matematiske fag}

\begin{document}
%#################################################
%Dette er for enkel copy-pasting
\ifx
%-------------------------
\begin{oppgave}
    \begin{punkt}
        \begin{align*}
        
        
        \end{align*}
    \end{punkt}
\end{oppgave}
%-------------------------
\begin{oppgave}
    \begin{punkt}
        
    \end{punkt}
\end{oppgave}
%-------------------------
\begin{align*} 
    \begin{bmatrix}
    1 & 2 & 5\\
    3 & 4 & 6
    \end{bmatrix}
 \end{align*}
 %-------------------------
\begin{align*}
    \left[
        \begin{array}{ccc|ccc}
        1 & 2 & 3 & 1 & 0 & 0 \\
        0 & -1 & -2 & -2 & 1 & 0 \\
        0 & -2 & -4 & -3 & 0 & 1 \\
        \end{array}
    \right]     
\end{align*}
%-------------------------
\begin{tikzpicture}
    \draw[step=1cm,gray,very thin] (-2.9,-2.9) grid (2.9,2.9);
    \draw (-3,0) -- (3,0);
    \draw (0,-3) -- (0,3);
    \draw[->] (-3,0) -- (3,0); 
    \draw[->] (0,-3) -- (0,3);
    \draw (0, 3.1) node {Im};
    \draw (3.1, 0) node {Re};
\end{tikzpicture}
\fi

%Her begynner dokumentet
%#####################################
\begin{oppgave}
    \begin{punkt}
        En delmengde $U$ er et underrom hvis $\textbf{0} \in U$.
        
        $x=y=0$ gir $x+y=0+0=0$. Delmengden er et underrom av $\mathbb{R}^2$.
    \end{punkt}
    
    \begin{punkt}
        Her ser vi at $\textbf{0}$ ikke ligger i mengden, siden $0+0 \neq 1$, så delmengden er ikke et underrom av $\mathbb{R}^2$.
    \end{punkt}
    \begin{punkt}
        Her ser vi at $\textbf{0}$ ligger i mengden, så delmengden er et underrom av $\mathbb{R}^2$.
    \end{punkt}
\end{oppgave}

\begin{oppgave}
    \begin{punkt}
        Kolonnerommet til $A$ er 
        \begin{align*}
            sp\left\{\begin{bmatrix} 0\\0\\0\\0\end{bmatrix},\begin{bmatrix} 1\\0\\0\\0\end{bmatrix},\begin{bmatrix} 1\\0\\0\\0\end{bmatrix},\begin{bmatrix} 0\\0\\0\\0\end{bmatrix},\begin{bmatrix} 0\\1\\0\\0\end{bmatrix},\begin{bmatrix} 0\\0\\1\\0\end{bmatrix},\begin{bmatrix} 1\\1\\1\\0\end{bmatrix}\right\}=sp\left\{\begin{bmatrix} 1\\0\\0\\0\end{bmatrix},\begin{bmatrix} 0\\1\\0\\0\end{bmatrix},\begin{bmatrix} 0\\0\\1\\0\end{bmatrix}\right\},
        \end{align*}
        som da gir at de tre vektorene utgjør en basis til kolonnerommet. Dimensjonen er 3, siden det er tre basisvektorer.
        
        
        Nullrommet til A er gitt ved $A\textbf{x}=\textbf{0}$, som vi løste i innlevering 2.
        
        Løsningen er gitt ved 
                \begin{align*}
            \begin{bmatrix}
                x_1\\
                x_2\\
                x_3\\
                x_4\\
                x_5\\
                x_6\\
                x_7
            \end{bmatrix}=
            \begin{bmatrix}
                a\\
                -c-d\\
                c\\
                b\\
                -d\\
                -d\\
                d
            \end{bmatrix}
        \end{align*}
        for $a, b, c, d \in \mathbb{R}$. Dette gir at en basis for nullrommet til A er gitt ved
        \begin{align}
        \left(
            \begin{bmatrix}
                1\\
                0\\
                0\\
                0\\
                0\\
                0\\
                0
            \end{bmatrix},
            \begin{bmatrix}
                0\\
                0\\
                0\\
                1\\
                0\\
                0\\
                0
            \end{bmatrix},
            \begin{bmatrix}
                0\\
                -1\\
                1\\
                0\\
                0\\
                0\\
                0
            \end{bmatrix},
            \begin{bmatrix}
                0\\
                -1\\
                0\\
                0\\
                -1\\
                -1\\
                1
            \end{bmatrix}
            \right),
        \end{align}
        som har dimensjon 4. 
        
        Radrommet til $A$ er gitt ved 
        \begin{align}
            sp\left\{[0,1,1,0,0,0,1], [0,0,0,0,1,0,1], [0,0,0,0,0,1,1] \right\}.
        \end{align}
        Disse vektorene er en basis for radrommet til $A$. Dimensjonen til radrommet er 3.
    \end{punkt}
    
    \begin{punkt}
        Kolonnerommet til $B$ er gitt ved 
        \begin{align*}
            sp\left\{\begin{bmatrix} 1\\2\\3\\4\end{bmatrix},\begin{bmatrix} 2\\3\\4\\5\end{bmatrix},\begin{bmatrix} 3\\4\\5\\6\end{bmatrix}\right\}
        \end{align*}
        Disse vektorene er lineært uavhengige, og utgjør da en basis for kolonnerommet. Kolonnerommet har dimensjon 3. 
        
        Nullrommet er gitt ved $B\textbf{x}=\textbf{0}$.
        \begin{align*}
            \left[
                \begin{array}{ccc|c}
                    1 & 2 & 3 & 0\\
                    2 & 3 & 4 & 0\\
                    3 & 4 & 5 & 0\\
                    4 & 5 & 6 & 0
                \end{array}
             \right]  
             &\sim 
             \left[
                \begin{array}{ccc|c}
                    1 & 2 & 3 & 0\\
                    0 & -1 & -2 & 0\\
                    0 & -2 & -4 & 0\\
                    0 & -3 & -6 & 0
                \end{array}
             \right]  
             \\&\sim 
             \left[
                \begin{array}{ccc|c}
                    1 & 0 & -1 & 0\\
                    0 & 1 & 2 & 0\\
                    0 & 0 & 0 & 0\\
                    0 & 0 & 0 & 0
                \end{array}
             \right]  
        \end{align*}
        La $x_3=t, t\in \mathbb{R}$. Da har vi
        \begin{align*}
            x_1-t=0\\
            x_2+2t=0
        \end{align*}
        En basis for nullrommet er da
        \begin{align*}
            \left(\begin{bmatrix}1\\\frac{1}{2} \\ 1 \end{bmatrix} \right),
        \end{align*}
        som har dimensjon 1.
        
        Radrommet til $B$ er gitt ved 
        \begin{align*}
            &sp\left\{\begin{bmatrix} 1 & 2 & 3\end{bmatrix},\begin{bmatrix} 2&3&4\end{bmatrix},\begin{bmatrix} 3&4&5\end{bmatrix}, \begin{bmatrix}4&5&6\end{bmatrix}\right\}\\=&sp\left\{\begin{bmatrix} 1 & 2 & 3\end{bmatrix},\begin{bmatrix} 2&3&4\end{bmatrix},\begin{bmatrix} 3&4&5\end{bmatrix}\right\}
        \end{align*}
        som utgjør en basis for $Row B$, som da har dimensjon 3.
    \end{punkt}
    
    \begin{punkt}
    \begin{align*}
        \begin{bmatrix} 0 \\ 1 \\ -2 \\ 3 \\ -1 \\ -1 \\ 1 \end{bmatrix} = NullA\begin{bmatrix} 0 \\ 3 \\ -2 \\1\end{bmatrix},
    \end{align*}
    altså ligger vektoren i nullrommet til $A$. Den er for stor til å kunne ligge i nullrommet til $B$. 
    \end{punkt}
    \begin{punkt}
        Dette er en vektor i $\mathbb{R}^4$, som passer med begge kolonnerommene. 
            \begin{align*}
        \begin{bmatrix} -1\\ -1 \\ -1 \\-1 \end{bmatrix} = ColB\begin{bmatrix} 1 \\ -1 \\ 0\end{bmatrix},
    \end{align*}
    altså ligger vektoren i $ColB$. $ColA$ har 0 som siste koordinat i alle basisvektorene, så ingen vektorer som har fjerde koordinat forskjellig fra null er i $ColA$.
    \begin{align*}
        \begin{bmatrix} -1\\ -1 \\ -1 \\-1 \end{bmatrix} \not\in ColA
    \end{align*}
    \end{punkt}
\end{oppgave}
\begin{oppgave}
    \begin{punkt}
        En basis for $\mathcal{P}_2$ er $(1, x, x^2)$. De er en basis fordi de er lineært uavhengige og de utspenner hele $\mathcal{P}_2$.
        \begin{align*}
            &\textbf{x}[1, x, x^2]=0\\
            &x=[0,0,0]
        \end{align*}
        Siden \textbf{x} har \textbf{0} som eneste løsning er vektorene lineært uavhengige. 
        Vektorene utspenner hele $\mathcal{P}_2$ fordi vi har tre lineært uavhengige vektorer, som da utspenner et rom av tre dimensjoner, som er like mange som $\mathcal{P}$.
    \end{punkt}
    \begin{punkt}
    $[1,2,3]$
    \end{punkt}
    \begin{punkt}
    $(x^0, x^1, ... , x^n)$
    \end{punkt}
\end{oppgave}
\begin{oppgave}
\begin{punkt}
    Dette er ikke et underrom av $\mathbb{R}^3$, siden det ikke inneholder \textbf{0}. Dette kan vi vite siden \textbf{u} er lineært uavhengig med $\textbf{a}_1$ og $\textbf{a}_2$, og det finnes dermed ingen måte å legge dem sammen på for å få \textbf{0}. Dette inkluderer da dette spesialtilfellet der skalaren vi ganger \textbf{u} mer er lik 1. 
\end{punkt}
\begin{punkt}
Ettersom $\textbf{a}_2$-$\textbf{a}_1$=\textbf{v} inneholder vektorrommet \textbf{0}, og det er derfor et underrom av $\mathbb{R}^3$.
\end{punkt}
\begin{punkt}
Ut ifra argumentasjonen i de to forrige deloppgavene ligger ikke \textbf{u}, men \textbf{v}, i kolonnerommet til matrisen. 
\end{punkt}
\end{oppgave}
\begin{oppgave}
    \begin{punkt}
        Denne påstanden er feil, siden nullmatrisen har kolonneromsdimensjon 0.
    \end{punkt}
    \begin{punkt}
        Siden du har flere kolonner enn rader, vil du få flere vektorer enn størrelsen på vektorene. Dermed må du ha minst én vektor som er lineært avhengig av de andre, og derfor får du minst én vektor i basisen til nullrommet. 
    \end{punkt}
\end{oppgave}
\begin{oppgave}[1]
    \begin{punkt}
        \begin{align*}
            T(c\cdot x)_y = -\sin (c\cdot x) \neq c\cdot T(x)_y = -c\sin x
        \end{align*}
        Dette er da ikke en lineærtransformasjon.
    \end{punkt}
    \begin{punkt}
        \begin{align*}
            T(c\cdot x) = e^{cx}+e^{cy}=e^c(e^x+e^y)\neq c(e^x+e^y)=cT(x)
        \end{align*}
        Dette er da ikke en lineærtransformasjon.
    \end{punkt}
    \begin{punkt}
        Dette ser ut som mye jobb men det er åpenbart en lineærtransformasjon. 
        Standardmatrisen får vi ved å putte inn en basis for rommet, som her er $\mathbb{R}^3$. Vi får da
        \begin{align}
            \begin{bmatrix}
                8 & -7 & 0 \\
                -8 & -7 & 3\\
                -4 & 5 & -8\\
                -6 & 6 & -4
            \end{bmatrix}
        \end{align}
        $KerT=NullA$. Vi gausseliminerer for å se hva $NullA$ er.
        \begin{align}
            \begin{bmatrix}
                8 & -7 & 0 \\
                -8 & -7 & 3\\
                -4 & 5 & -8\\
                -6 & 6 & -4
            \end{bmatrix}
            &\sim
            \begin{bmatrix}
                8 & -7 & 0 \\
                0 & -14 & 3\\
                0 & 1.5 & -8\\
                0 & 0.75 & -4
            \end{bmatrix}
            \\&\sim
            \begin{bmatrix}
                8 & -7 & 0 \\
                0 & -14 & 3\\
                0 & 0 & 1\\
                0 & 0 & 0
            \end{bmatrix}
            \\&\sim
            \begin{bmatrix}
                1 & 0 & 0 \\
                0 & 1 & 0\\
                0 & 0 & 1\\
                0 & 0 & 0
            \end{bmatrix}
        \end{align}
        Vi ser $NullA=\textbf{0}$. $KerT$ er da $\begin{bmatrix}0\\0\\0\end{bmatrix}$.
        $ImT=ColA$.
        \begin{align*}ColA=sp\left\{
            \begin{bmatrix}
                8 \\-8\\-4\\-6\end{bmatrix},
            \begin{bmatrix}
                -7\\-7\\5\\6\end{bmatrix},
            \begin{bmatrix}
                0\\3\\-8\\-4
            \end{bmatrix}
            \right\}
        \end{align*}
        Vi vet disse er lineært uavhengige fra gausseliminasjonen over. Spennet er da $ImT$. Siden dette er et spenn av tre lineært uavhengige vektorer utspenner de hele $\mathbb{R}^3$, og derfor er $T$ ikke surjektiv, siden den måtte ha utspent $\mathbb{R}^4$. Alle kolonnevektorene i $A$ er lineært uavhengige, og derfor er $T$ injektiv. 
    \end{punkt}
    \begin{punkt}
        \begin{align*}
            T(c\textbf{x})\neq cT(\textbf{x})
        \end{align*}
    \end{punkt}
    \begin{punkt}
        \begin{align*}
            T(c\textbf{x})=cx+2cy+3cz+4cw=c(x+2y+3z+4w)=cT(\textbf{x})
        \end{align*}
            Altså er dette en lineærtransformasjon. 
            Standardmatrisen er 
        \begin{align*}
            \begin{bmatrix}
                1 & 2 & 3 & 4
            \end{bmatrix}
        \end{align*}
        \begin{align*}
        KerT=NullA=sp\left\{
            \begin{bmatrix}
                -2\\1\\0\\0
            \end{bmatrix},
            \begin{bmatrix}
                -3\\0\\1\\0
            \end{bmatrix},
            \begin{bmatrix}
                -4\\0\\0\\1
            \end{bmatrix}
            \right\}
        \end{align*}
        \begin{align*}
            ImT=ColA=sp\{1,2,3,4\}=sp\{1\}=\mathbb{R}
        \end{align*}
        Ettersom $dim(ColA)=\mathbb{R}$, som er lik kodomenet. Derfor er $T$ surjektiv. 
        Siden $KerT\neq \textbf{0}$ er ikke $T$ injektiv. 
    \end{punkt}
\end{oppgave}
\begin{oppgave}[2]
    \begin{punkt}
        \begin{align*}
            S\left(\begin{bmatrix} 1\\0\end{bmatrix}\right)&=\begin{bmatrix} 1\\0\end{bmatrix}\\
            S\left(\begin{bmatrix} 0\\1\end{bmatrix}\right)&=\begin{bmatrix} 0\\-1\end{bmatrix}
        \end{align*}
        \begin{align*}
            A=\begin{bmatrix} -1&0\\0&1\end{bmatrix}
        \end{align*}
    \end{punkt}
    \begin{punkt}
        \begin{align*}
            R\left(\begin{bmatrix} 1\\0\end{bmatrix}\right)&=\begin{bmatrix} \frac{-\sqrt{2}}{2}\\\frac{\sqrt{2}}{2}\end{bmatrix}\\
            R\left(\begin{bmatrix} 0\\1\end{bmatrix}\right)&=\begin{bmatrix} \frac{-\sqrt{2}}{2}\\\frac{-\sqrt{2}}{2}\end{bmatrix}
        \end{align*}
        \begin{align*}
            A=\begin{bmatrix} \frac{-\sqrt{2}}{2}&\frac{-\sqrt{2}}{2}\\\frac{\sqrt{2}}{2}&\frac{-\sqrt{2}}{2}\end{bmatrix}
        \end{align*}
    \end{punkt}
\end{oppgave}
\begin{oppgave}[3]
$S \circ R:$
    \begin{align*}
    \begin{bmatrix} 1&0\\0&-1\end{bmatrix}\begin{bmatrix} \frac{-\sqrt{2}}{2}&\frac{-\sqrt{2}}{2}\\\frac{\sqrt{2}}{2}&\frac{-\sqrt{2}}{2}\end{bmatrix}=\begin{bmatrix} \frac{-\sqrt{2}}{2}&\frac{-\sqrt{2}}{2}\\\frac{-\sqrt{2}}{2}&\frac{\sqrt{2}}{2}\end{bmatrix}
    \end{align*}
    Dette tilsvarer å rotere $\frac{3\pi}{4}$ og å deretter speile om x-aksen.
    
$R \circ S:$
    \begin{align*}
    \begin{bmatrix} \frac{-\sqrt{2}}{2}&\frac{-\sqrt{2}}{2}\\\frac{\sqrt{2}}{2}&\frac{-\sqrt{2}}{2}\end{bmatrix}\begin{bmatrix} 1&0\\0&-1\end{bmatrix}=\begin{bmatrix} \frac{-\sqrt{2}}{2}&\frac{\sqrt{2}}{2}\\\frac{\sqrt{2}}{2}&\frac{\sqrt{2}}{2}\end{bmatrix}
    \end{align*}
    Dette tilsvarer å speile om x-aksen og å deretter rotere $\frac{3\pi}{4}$.
\end{oppgave}
\begin{oppgave}[4]
\begin{punkt}
    $D:$
    \begin{align*}
        &D(c\mathcal{P})=c\mathcal{P}'=cD(\mathcal{P})
        \\&D(\mathcal{P}_1+\mathcal{P}_2)=\mathcal{P}_1'+\mathcal{P}_2'=D(\mathcal{P}_1)+D(\mathcal{P}_2)
    \end{align*}
    $G:$
    \begin{align*}
        &G(c\mathcal{P})=x\cdot c\mathcal{P} = cG(\mathcal{P})
        \\&G(\mathcal{P}_1+\mathcal{P}_2)=x(\mathcal{P}_1+\mathcal{P}_2)=x\mathcal{P}_1+x\mathcal{P}_2=G(\mathcal{P}_1)+G(\mathcal{P}_2)
    \end{align*}
    \end{punkt}
    \begin{punkt}
        $ImD=\mathcal{P}_{n-1}$
        
        $ImG=sp\{x, x^2, ... , x^n\}$
        
        $KerD=\mathbb{R}$
        
        $KerG=\textbf{0}$
    \end{punkt}
    \begin{punkt}
        $D$ er surjektiv, siden $ImD$ er lik kodomenet. Den er ikke injektiv, siden $D(0)=D(1)$.
        
        $G$ er injektiv, siden $KerG=\textbf{0}$. Den er ikke surjektiv, siden $G(1)=G(0)$.
    \end{punkt}
    \begin{punkt}
    La $\mathcal{P}=(1, x, x^2, ... , x^n)$ være en basis.
        \begin{align*}
            D\circ G(\mathcal{[\textbf{x}]_\mathcal{P}})=[i\cdot x_i\: \forall\: i\in \{1, 2,..., n\}]
        \end{align*}
        \begin{align*}
            G\circ D(\mathcal{[\textbf{x}]_\mathcal{P}})=[(i-1)\cdot x_i\: \forall\: i\in \{2,..., n\}]
        \end{align*}
        Da har vi at $D\circ G-G(\mathcal{[\textbf{x}]_\mathcal{P}})\circ C=((\mathcal{[\textbf{x}]_\mathcal{P}}))$
    \end{punkt}
    \begin{punkt}
        $\mathcal{P}_2=(1, x, x^2)$, $\mathcal{P}_3=(1, x, x^2, x^3)$.
        
        $D_3:\begin{bmatrix}0 & 0 & 0\\0 & 1 & 0\\0 & 0 & 2 \end{bmatrix}$
        
        $G_3:\begin{bmatrix}0&0&0\\1 & 0 & 0 \\0 & 1 &  0\\0 & 0 & 1 \end{bmatrix}$
    \end{punkt}
\end{oppgave}
\begin{oppgave}[5]
    \begin{punkt}
    \begin{align*}
        &D(cf)=cf'=cD(f)
        \\&D(f_1+f_2)=f_1'+f_2'=D(f_1)+D(f_2)
    \end{align*}
    \end{punkt}
    \begin{punkt}
        $KerD=sp{1}$, $dim(KerD)=1$, en basis er $(1)$.
    \end{punkt}
    \begin{punkt}
        Analysens fundamentalteorem sier at integralet av en derivert funksjon er lik seg selv. Det tilsier at alle funksjoner i kodomenet vil ha minst én funksjon i domenet. Dermed er D surjektiv.
    \end{punkt}
\end{oppgave}
\end{document}