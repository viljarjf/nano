\documentclass[a4paper, 12pt]{article}
\usepackage[utf8]{inputenc}
\usepackage{siunitx}
\usepackage{amsmath}
\usepackage{amssymb}
\usepackage{esvect}
\usepackage{esint}
\usepackage[cmr]{emf}
\usepackage{color}   
\usepackage{hyperref}
\usepackage[table]{xcolor}
\usepackage{bm}
\usepackage{titlesec}
\usepackage{longtable}
\usepackage{multirow}
\usepackage{physics}
\newcommand{\sectionbreak}{\clearpage} % new page for each new section
\hypersetup{
    colorlinks=true, 
    linktoc=all,     
    linkcolor=blue,  
}

\makeatletter
%\renewcommand{\vec}{\vv}
\renewcommand{\vec}[1]{\bm{#1}}
\renewcommand{\phi}{\varphi}
\newcommand*{\defeq}{\stackrel{\text{def}}{=}}

% vector shortcuts
\newcommand{\E}{\ensuremath{\vec{E}}}
\renewcommand{\epsilon}{\varepsilon}
\newcommand{\e}{\ensuremath{\epsilon_0}}
\newcommand{\p}{\ensuremath{\vec{P}}}
\newcommand{\D}{\ensuremath{\vec{D}}}
\renewcommand{\j}{\ensuremath{\vec{J}}}
\newcommand{\B}{\ensuremath{\vec{B}}}
\renewcommand{\H}{\ensuremath{\vec{H}}}

% add equation numbering to align*
\newcommand\numberthis{\addtocounter{equation}{1}\tag{\theequation}}

% pasta from https://tex.stackexchange.com/questions/459167/typesetting-dalembertian-symbol?rq=1
\newcommand{\dalambert}{\mathop{\mathpalette\dalembertian@\relax}}
\newcommand{\dalembertian@}[2]{%
  \begingroup
  \sbox\z@{$\m@th#1\square$}%
  \dimen0=\fontdimen8
    \ifx#1\displaystyle\textfont\else
    \ifx#1\textstyle\textfont\else
    \ifx#1\scriptstyle\scriptfont\else
    \scriptscriptfont\fi\fi\fi3
  \makebox[\wd\z@]{%
    \hbox to \ht\z@{%
      \vrule width \dimen0
      \kern-\dimen0
      \vbox to \ht\z@{
        \hrule height \dimen0 width \ht\z@
        \vss
        \hrule height 2\dimen0
      }%
      \kern-2.5\dimen0
      \vrule width 2.5\dimen0
    }%
  }%
  \endgroup
}

\let\tmp\hat
\renewcommand{\hat}[1]{\vec{\tmp{#1}}}

%\renewcommand{\nabla}{\text{aruran}}

\makeatother

\begin{document}

\author{Viljar Johan Femoen}

\flushbottom
\maketitle
\thispagestyle{empty}
\vskip 20pt
\noindent {\Large{\textbf{Disclaimer}}}\vskip 2pt
\noindent This document contains some of the formulas and results from the course, with some motivation where I felt like it. 
This is not intended as learning material, bur rather as a collection of formulas with the required assumptions specified. 
Error reporting to \textit{viljar@timini.no} is much appreciated. 
A vector quantity without an arrow is the modulus of the corresponding vector, if not otherwise specified. 
I might also have changed the vector notation from arrows to bold letters, without changing this disclaimer. 
Vectors with hats are unit vectors.
The lectures and my notes are both in english, and as such this document will also be in english.

\textbf{Status: A little left regarding EM waves}

\tableofcontents

\subsection*{Task 1}

    For subtasks a to d, see \autoref{fig:cube size surface energy}.

    The main approximations in this task are ignoring edge-effects, as the 
    constant surface energy is only a valid approximation when sufficiently far from the edges.

    \begin{figure}[bh]
        \includegraphics[width=10cm]{figs/Energy_of_1g_NaCl_as_a_function_of_cube_size.png}

        \caption{The energy as a function of cube size is 
        illustrated with 1g of NaCl. 
        The total energy is comparable to the fusion enthalpy at 
        around 1 nm sidelength. 
        As such, the melting point of nanoparticles should be 
        \textit{lower} than in bulk}
        \label{fig:cube size surface energy}
    \end{figure}

    The ratio of surface- and bulk atoms can be seen in \autoref{fig:bulk surface ratio}.

    The smallest possible configuration with atoms that are technically in the bulk
    would be a particle consisting of 13 atoms, where 12 are on the surface, 
    all "touching" the single bulk atom. 
    However, the apparrent continuity of rational numbers is 
    broken long before this, and there should be gaps in the graph
    where there are no possible configurations with the calculated ratio.
    Additionally, the point at which the core of the particle stops
    exhibiting bulk behavior probably occurs at an earlier stage. 
    Finally, the simplification used for calculating the amount of
    surface atoms stops being valid at very small radii. 
    None of these points give a precise radius for which the model
    "stops making sense", so the simplest answer to the task is $a = 4r_{\text{Pd}}$.

    \begin{figure}[h]
        \includegraphics[width=10cm]{figs/Ratio_of_surface_and_bulk_atoms.png}

        \caption{The ratio of particles on the surface of a nanoparticle
        to the particles in the bulk, as a function of the nanoparticle radius.
        To be clear, the ratio is $\frac{n_\text{surface}}{n_\text{bulk}}$, 
        and not $\frac{n_\text{surface}}{n_\text{total}}$.
        }
        \label{fig:bulk surface ratio}
    \end{figure}

    The exact ratio can be calculated from the geometry in the given Table 1,
    and yields the following expression:
    $$
    r(n) = \frac{30n^2 + 6}{10n^3 -15n^2 + 11n -3},
    $$
    where $n$ is the number of layers. 
    This is plotted in \autoref{fig:bulk surface ratio}, 
    modelling the radius of the tetradecahedrons as 
    $r = r_\text{Pd}\left(2n + 1\right)$.

    At $r = 3\si{\nano\metre}$, around a quarter of 
    all the atoms in the particle are on the surface, which I
    would consider significant. If the interpretation instead should
    be "siginificant \textit{number} of atoms, then the 200-odd atoms at 
    $n = 5$ should suffice as significant, in my opinion.

\subsection*{Task 2}
    The lattice parameter tends to get reduced at smaller nanoparticle
    sizes. 

    With a fixed surface area, one simpe way to reduce the surface
    energy is to change the solvent to one with more favourable 
    interractions with the particles. Additionally, some surface 
    geometries are less energetically favourable than others, 
    even with constant surface area. 
    
\end{document}