\documentclass[11pt, a4paper, norsk]{NTNUoving}
\usepackage[utf8]{inputenc}
\usepackage[T1]{fontenc}
\usepackage{tikz}

\ovingnr{5}    % Nummer på innlevering
\semester{Våren 2020}
\fag{TMA 4115} %4105 er matte 2, 4115 er matte 3
\institutt{Institutt for matematiske fag}

\newenvironment{pkt}{\begin{punkt}}{\end{punkt}}

\newenvironment{matrise}[1][]{
        \left[
            \begin{array}{#1}
    }
    {
            \end{array}
        \right]      
}

\newenvironment{indreprod}{
    \langle
}{
\rangle}

\newcommand{\R}{\mathbb{R}}
\renewcommand{\P}{\mathcal{P}}

\begin{document}
%#################################################
%Dette er for enkel copy-pasting
\ifx
%-------------------------
\begin{oppgave}
    \begin{punkt}
        \begin{align*}
        
        
        \end{align*}
    \end{punkt}
\end{oppgave}
%-------------------------
\begin{tikzpicture}
        \draw[step=1cm,gray,very thin] (-5.9,-5.9) grid (5.9,5.9);
        \draw (-6,0) -- (6,0);
        \draw (0,-6) -- (0,6);
        \draw[->] (-6,0) -- (6,0); 
        \draw[->] (0,-6) -- (0,6);
        \draw (0, 6.2) node {y};
        \draw (6.2, 0) node {x};
\end{tikzpicture}
\fi

%Her begynner dokumentet
%#####################################

\begin{oppgave}
    \begin{pkt}
        \begin{align*}
            det
            \begin{matrise}[cc]
                1-\lambda & 2\\
                2 & 1-\lambda
            \end{matrise}
            &=(1-\lambda)^2-4
            \\&=(\lambda-3)(\lambda+1)        
        \end{align*}
        Det gir egenverdiene $\lambda = 3$ og $\lambda=-1$.
        
        $\lambda=3:$
        \begin{align*}
            \begin{matrise}[cc]
                -2 & 2 \\ 2 & -2
            \end{matrise}
            &\sim
            \begin{matrise}[cc]
                1 & -1\\0 & 0
            \end{matrise}
        \end{align*}
        Egenrommet er da $sp\left\{\begin{matrise} 1 \\ 1\end{matrise}\right\}$.

        $\lambda=-1:$
        \begin{align*}
            \begin{matrise}[cc]
                2 & 2 \\ 2 & 2
            \end{matrise}
            \sim 
            \begin{matrise}[cc]
                1 & 1 \\ 0 & 0
            \end{matrise}
        \end{align*}
        Egenrommet er da $sp\left\{\begin{matrise} 1 \\ -1\end{matrise}\right\}$.
    \end{pkt}
    \begin{pkt}
        \begin{align*}
        det
            \begin{matrise}[ccc]
                1-\lambda & 2 & 0\\
                2 & 1-\lambda & 0\\
                0 & 0 & -\lambda
            \end{matrise}
            =-\lambda(det\begin{matrise}[cc]
                1-\lambda & 2\\
                2 & 1-\lambda
            \end{matrise}
        \end{align*}
        Det gir egenverdiene $\lambda = 3$, $\lambda=0$, og $\lambda=-1$.
        
        $\lambda=0:$
        \begin{align*}
            \begin{matrise}[ccc]
                1 & 2 & 0\\
                2 & 1 & 0\\
                0 & 0 & 0
            \end{matrise}
            \sim
            \begin{matrise}[ccc]
                1 & 0 & 0\\
                0 & 1 & 0\\
                0 & 0 & 0
            \end{matrise}
        \end{align*}
        Egenrommet er da $sp\left\{\begin{matrise} 0 \\ 0 \\ 1\end{matrise}\right\}$.
        
        Fra oppgave 1 a) ser vi at for $\lambda=3$ er egenrommet $sp\left\{\begin{matrise} 1 \\ 1 \\ 0\end{matrise}\right\}$.
        
        Tilsvarende for $\lambda=-1$ er egenrommet $sp\left\{\begin{matrise} 1 \\ -1 \\ 0\end{matrise}\right\}$.
    \end{pkt}
    \begin{pkt}
        Her ser vi $\lambda=0$, med algebraisk multiplisitet lik 2. 
        
        Egenrommet for $\lambda=0$ er $sp\left\{\begin{matrise} 1 \\ 0\end{matrise}\right\}$.
    \end{pkt}
    \begin{pkt}
    \begin{math}
        \det
        \begin{matrise}[ccc]
            4-\lambda & 2 & 3\\
            -1 & 1-\lambda & -3\\
            2 & 4 & 9-\lambda
        \end{matrise}
        \\&=(4-\lambda)\det\begin{matrise}[cc]
            1-\lambda & -3 \\
            4 & 9-\lambda
        \end{matrise}
        -2\det\begin{matrise}[cc]
            -1 & -3\\
            2 & 9-\lambda
        \end{matrise}
        +3\det\begin{matrise}[cc]
            -1 & 1-\lambda\\
            2 & 4
        \end{matrise}
        \\&=(4-\lambda)((1-\lambda)(9-\lambda)+12)-2(\lambda-9+6)+3(-4-2(1-\lambda))
        \\&=  - \lambda^3 + 14 \lambda^2 - 57 \lambda+72        
    \end{math}

    Ved hjelp av stirremetoden ser vi at $\lambda = 3$ er en løsning. Polynomdivisjon gir dermed
    \begin{align*}
        - \lambda^3 + 14 \lambda^2 - 57 \lambda+72 &=(\lambda-3)(-\lambda^2+11\lambda-24)
        \\&=-(\lambda-3)^2(\lambda-8)
    \end{align*}
    $\lambda=3:$
    \begin{align*}
        \begin{matrise}[ccc]
            1 & 2 & 3\\
            -1 & -2 & -3\\
            2 & 4 & 6
        \end{matrise}
        \sim
        \begin{matrise}[ccc]
            1 & 2 & 3\\
            0 & 0 & 0\\
            0 & 0 & 0
        \end{matrise}
    \end{align*}
    Egenrommet er da $sp\left\{\begin{matrise} 6 \\ 3 \\ 2\end{matrise}\right\}$.
    
    $\lambda=8:$
    \begin{align*}
        \begin{matrise}[ccc]
            -4 & 2 & 3\\
            -1 & -7 & -3\\
            2 & 4 & 1
        \end{matrise}
        &\sim
        \begin{matrise}[ccc]
            -4 & 2 & 3\\
            0 & 30 & 15\\
            0 & 10 & 5
        \end{matrise}
        \\&\sim
        \begin{matrise}[ccc]
            -4 & 2 & 3\\
            0 & 2 & 1\\
            0 & 0 & 0
        \end{matrise}
        \\&\sim
        \begin{matrise}[ccc]
            2 & 0 & -1\\
            0 & 2 & 1\\
            0 & 0 & 0
        \end{matrise}
    \end{align*}
    Egenrommet er da $sp\left\{\begin{matrise} 1 \\ -1 \\ 2\end{matrise}\right\}$.
    \end{pkt}
\end{oppgave}

\begin{oppgave}
    \begin{pkt}
        \begin{align*}
            \det
            \begin{matrise}[cc]
                -\lambda & 1\\
                -1 & -\lambda
            \end{matrise}
            =\lambda^2+1,
        \end{align*}
        som ikke har noen reelle løsninger.
    \end{pkt}
    \begin{pkt}
        Geometrisk er denne matrisen en lineærtransformasjon som roterer alle vektorer $\frac{3\pi}{2}$. Den eneste reelle vektoren som sendes på seg selv er da \textbf{0}, som ikke er en egenvektor.
    \end{pkt}
\end{oppgave}

\begin{oppgave}
    \begin{pkt}
        $\begin{matrise}\cos\theta\\\sin\theta\end{matrise}$ for \textbf{e}$_1$, og $\begin{matrise}-\sin\theta\\\cos\theta\end{matrise}$ for \textbf{e}$_2$.
    \end{pkt}
    \begin{pkt}
        $\begin{matrise}[cc]\cos\theta & -\sin\theta\\\sin\theta & \cos\theta\end{matrise}$
    \end{pkt}
    \begin{pkt}
        Når man roterer en vektor vil den være parallell med start hver $\pi$ radianer. $T_\theta$ har dermed reelle egenverdier for $\theta=n\pi\,\forall\, n \in \mathbb{Z}$.
    \end{pkt}
\end{oppgave}

\begin{oppgave}
    \begin{pkt}
        Matrisen er på trappeform, som gir at egenverdiene er verdiene på diagonalen. Her er de 1, 2, -5, og 77.
    \end{pkt}
    \begin{pkt}
        $\lambda=1:$
        \begin{align*}
            \begin{matrise}[cccc]
                0 & 2 & 3 & 4\\
                0 & 1 & 3 & 4\\
                0 & 0 & -6 & 0\\
                0 & 0 & 0 & 76
            \end{matrise}
        \end{align*}
        Egenrommet er da $sp\left\{\begin{matrise} 1 \\ 0 \\ 0 \\ 0\end{matrise}\right\}$.
        
        $\lambda=2:$
        \begin{align*}
            \begin{matrise}[cccc]
                -1 & 2 & 3 & 4\\
                0 & 0 & 3 & 4\\
                0 & 0 & -7 & 0\\
                0 & 0 & 0 & 75
            \end{matrise}
        \end{align*}
        Egenrommet er da $sp\left\{\begin{matrise} 2 \\ 1 \\ 0 \\ 0\end{matrise}\right\}$.
        
        $\lambda=-5:$
        \begin{align*}
            \begin{matrise}[cccc]
                6 & 2 & 3 & 4\\
                0 & 7 & 3 & 4\\
                0 & 0 & 0 & 0\\
                0 & 0 & 0 & 82
            \end{matrise}
            &\sim
            \begin{matrise}[cccc]
                6 & 2 & 3 & 0\\
                0 & 7 & 3 & 0\\
                0 & 0 & 0 & 0\\
                0 & 0 & 0 & 1
            \end{matrise}
            \\&\sim
            \begin{matrise}[cccc]
                6 & 0 & \frac{15}{7} & 0\\
                0 & 7 & 3 & 0\\
                0 & 0 & 0 & 0\\
                0 & 0 & 0 & 1
            \end{matrise}
            \\&\sim
            \begin{matrise}[cccc]
                14 & 0 & 5 & 0\\
                0 & 14 & 6 & 0\\
                0 & 0 & 0 & 0\\
                0 & 0 & 0 & 1
            \end{matrise}
        \end{align*}
        Egenrommet er da $sp\left\{\begin{matrise} 5 \\ 6 \\ -14 \\ 0\end{matrise}\right\}$.
        
        $\lambda=77:$
        \begin{align*}
            \begin{matrise}[cccc]
                -76 & 2 & 3 & 4\\
                0 & -75 & 3 & 4\\
                0 & 0 & -82 & 0\\
                0 & 0 & 0 & 0
            \end{matrise}
            &\sim
            \begin{matrise}[cccc]
                -38 & 1 & 0 & 2\\
                0 & -75 & 0 & 4\\
                0 & 0 & 1 & 0\\
                0 & 0 & 0 & 0
            \end{matrise}
            \\&\sim
            \begin{matrise}[cccc]
                -17 & 0 & 0 & \frac{77}{75}\\
                0 & -75 & 0 & 4\\
                0 & 0 & 1 & 0\\
                0 & 0 & 0 & 0
            \end{matrise}
            \\&\sim
            \begin{matrise}[cccc]
                -1425 & 0 & 0 & 77\\
                0 & -75 & 0 & 4\\
                0 & 0 & 1 & 0\\
                0 & 0 & 0 & 0
            \end{matrise}
        \end{align*}
        
        Egenrommet er da $sp\left\{\begin{matrise} 75*77 \\ 4*1425 \\ 0 \\ 1425*75\end{matrise}\right\}$.
    \end{pkt}
    \begin{pkt}
        Nei, dette var et fryktelig rart spørsmål. For en $n\times n$-matrise blir egenverdiene løsninger på et $n$-tegradspolynom, altså er det ikke på noen måte gitt at dette er enkelt.
    \end{pkt}
\end{oppgave}

\begin{oppgave}
\begin{pkt}
    Det virker slitsomt å regne ut alle egenverdiene og egenrommene. Derfor velger jeg bare å sette inn og se. Nullvektoren og enhetsvektoren er trivielt ikke egenvektorer.
    \begin{align*}
        \begin{matrise}[cccc]
            28 & 30 & -20 & -2\\
            6 & 40 & -10 & -4\\
            4 & 10 & 20 & -6\\
            2 & 20 & -30 & 32
        \end{matrise}
        \begin{matrise}
        1\\2\\3\\4
        \end{matrise}
        =
        \begin{matrise}
            28+60-60-8\\
            6+80-30-16\\
            4+20+60-24\\
            2+40-90+128
        \end{matrise}
        =
        \begin{matrise}
            20\\40\\60\\80
        \end{matrise}
        =20
        \begin{matrise}
        1\\2\\3\\4
        \end{matrise}
    \end{align*}
    Denne er da en egenvektor med egenverdi 20.
    \begin{align*}
        \begin{matrise}[cccc]
            28 & 30 & -20 & -2\\
            6 & 40 & -10 & -4\\
            4 & 10 & 20 & -6\\
            2 & 20 & -30 & 32
        \end{matrise}
        \begin{matrise}
        3\\2\\2\\1
        \end{matrise}
        =
        \begin{matrise}
            84+60-40-2\\
            18+80-20-4\\
            12+20+40-6\\
            6+40-60+32
        \end{matrise}
        =
        \begin{matrise}
            102\\
            74\\
            66\\
            18
        \end{matrise}
    \end{align*}
    Denne er da ikke en egenvektor.
    
    Litt magi gir at 20 er en egenverdi med multiplisitet minst lik 2, ettersom den fjerde vektoren også ligger i det egenrommet. I tillegg er den siste vektoren en egenvektor, med 40 som egenverdi. 
    \end{pkt}
    \begin{pkt}
        Enda mer magi gir at 40 også har multiplisitet lik 2.
        \begin{align*}
        \begin{matrise}[cccc]
            8 & 30 & -20 & -2\\
            6 & 20 & -10 & -4\\
            4 & 10 & 0 & -6\\
            2 & 20 & -30 & 12
        \end{matrise}
        &\sim
        \begin{matrise}[cccc]
            2 & 20 & -30 & 12\\
            8 & 30 & -20 & -2\\
            6 & 20 & -10 & -4\\
            4 & 10 & 0 & -6
        \end{matrise}
        \\&\sim
        \begin{matrise}[cccc]
            2 & 20 & -30 & 12\\
            0 & -50 & 100 & -50\\
            0 & -40 & 80 & -40\\
            0 & -30 & 60 & -30
        \end{matrise}
        \\&\sim
        \begin{matrise}[cccc]
            1 & 10 & -15 & 6\\
            0 & 1 & -2 & 1\\
            0 & 0 & 0 & 0\\
            0 & 0 & 0 & 0
        \end{matrise}
        \\&\sim
        \begin{matrise}[cccc]
            1 & 4 & -3 & 0\\
            0 & 1 & -2 & 1\\
            0 & 0 & 0 & 0\\
            0 & 0 & 0 & 0
        \end{matrise}
        \end{align*}
        
        Egenrommet er da $sp\left\{\begin{matrise} 3 \\ 0 \\ 1 \\ 2\end{matrise},\begin{matrise} 4 \\ -1 \\ 0 \\ 1\end{matrise}\right\}$.
        
        Egenrommet til $\lambda = 20$ er $sp\left\{\begin{matrise} 1 \\ 2 \\ 3 \\ 4\end{matrise},\begin{matrise} 2 \\ 1 \\ 2 \\ 3\end{matrise}\right\}$, som vi fant i forrige deloppgave.
    \end{pkt}
\end{oppgave}

\begin{oppgave}
    For at $A^2=A$ må $A$ være kvadratisk. Dersom matrisen har en invers må $A=A^{-1}$, som gir $A=A^2=AA^{-1}=I_n$. Egenverdien for denne er $\lambda=1$, med multiplisitet $n$. Om $A$ ikke har en invers vil determinanten være 0, som gir at 0 også er en egenverdi. 
\end{oppgave}
\begin{oppgave}
    Dersom $\lambda=0$ er en egenverdi, er $\det (A-0I_n)=\det A=0$, og derfor er ikke matrisen inverterbar. 
\end{oppgave}


\begin{oppgave}[1]
    \begin{pkt}
        Egenverdiene er 2 og 4. Egenvektorene er $\begin{matrise}6 \\0\\-3\end{matrise}$, $\begin{matrise}0\\6 \\3\end{matrise}$, og $\begin{matrise}-1\\-3\\2\end{matrise}$. Matrisen er diagonaliserbar ettersom den har tre (lineært uahvengige) egenvektorer.
    \end{pkt}
    \begin{pkt}
        Determinanten er $(3-\lambda)(-\lambda)(\lambda^2+1)$. Det gir egenverdiene 3, 0, $i$, og $-i$. Egenvektorene er matematisk enkle å finne, men det er mye jobb. Matrisen er diagonaliserbar, ettersom den har 4 unike egenverdier. 
    \end{pkt}
\end{oppgave}
\begin{oppgave}[2]
    Egenverdiene til $A$ er $\pm \sqrt{3}$. Egenvektorene er $\begin{matrise}1+\sqrt{3}\\1+i\end{matrise}$ og $\begin{matrise}1-\sqrt{3}\\1+i\end{matrise}$. $D$ og $P$ er henholdsvis $I_2$, der hver rad er ganget med hver sin egenverdi, og matrisen av egenvektorene. 
\end{oppgave}
\begin{oppgave}[3]
    La $z=a+bi$. Da er egenverdiene gitt ved $(r_1-\lambda)(r_2-\lambda)-a^2-b^2=0$. abc-formelen gir dermed at egenverdiene er 
    \begin{align*}
        \frac{\pm\sqrt{4a^2+4b^2+r_1^2+r_2^2-2r_1r_2}+r_1+r_2}{2}
    \end{align*}
\end{oppgave}
\begin{oppgave}[4]
    $A$ er en reell symmetrisk matrise, og er dermed diagonaliserbar.
    
    Egenverdiene er gitt ved $(a-b-\lambda)((a-\lambda)^2-b^2)=(a-b-\lambda)(a-\lambda-b)(a-\lambda+b)$. Egenvektorene er $\begin{matrise}0 \\0\\1\end{matrise}$, $\begin{matrise}1 \\1\\0\end{matrise}$, og $\begin{matrise}1 \\-1\\0\end{matrise}$.
\end{oppgave}
\begin{oppgave}[5]
\begin{pkt}
    \begin{align*}
        T\left(\begin{matrise}
            1\\0\\0
        \end{matrise}\right)
        = \begin{matrise}
            1\\0\\0
        \end{matrise}
    \end{align*}
    
    \begin{align*}
        T\left(\begin{matrise}
            0\\1\\0
        \end{matrise}\right)
        = \begin{matrise}
            1\\2\\0
        \end{matrise}
    \end{align*}
    
    \begin{align*}
        T\left(\begin{matrise}
            0\\0\\1
        \end{matrise}\right)
        = \begin{matrise}
            0\\2\\3
        \end{matrise}
    \end{align*}
    Det gir at standardmatrisen til $T$ er $\begin{matrise}[ccc]1&1&0\\0&2&2\\0&0&3\end{matrise}$.
    \end{pkt}
    \begin{pkt}
        Egenverdiene er 1, 2 og 3. Det gir at matrisen er diagonaliserbar. Rask hoderegning tilsier at egenvektorene burde være noe sånt som $\begin{matrise}1 \\0\\0\end{matrise}$, $\begin{matrise}1 \\1\\0\end{matrise}$, og $\begin{matrise}1 \\2\\1\end{matrise}$.
    \end{pkt}
\end{oppgave}
\begin{oppgave}[6]
    \begin{pkt}
        $\frac{\pi}{6}$
    \end{pkt}
    \begin{pkt}
        For den generelle rotasjonsmatrisen er egenverdiene $(\cos\theta-\lambda)^2+\sin^2\theta = \lambda^2-2\lambda\cos\theta +1$. Det gir at egenverdiene er $ \pm\sqrt{\operatorname{cos} ^{2}\left( \theta \right) - 1} + \operatorname{cos} \left( \theta \right) =\pm i\sin\theta +\cos\theta$. Her blir det $\pm \frac{1}{2}i+\frac{\sqrt{3}}{2}$. 
        
        De tilhørende egenvektorene er $\begin{matrise}
        \pm i \\ 1
        \end{matrise}$.
    \end{pkt}
    \begin{pkt}
        \begin{align*}
            T(\textbf{v}_1) &= \frac{i}{2}\begin{matrise}\sqrt{3}  \\ 1 \end{matrise}+\frac{1}{2}\begin{matrise}-1\\\sqrt{3}  \end{matrise}
            \\&=\frac{1}{2}\begin{matrise}i\sqrt{3}-1  \\ 1+\sqrt{3} \end{matrise}\\
            T(\textbf{v}_2) &= -\frac{i}{2}\begin{matrise}\sqrt{3}  \\ 1 \end{matrise}+\frac{1}{2}\begin{matrise}-1\\\sqrt{3}  \end{matrise}
            \\&=\frac{1}{2}\begin{matrise}-i\sqrt{3}-1  \\ 1+\sqrt{3} \end{matrise}
        \end{align*}
        Det gir at standardmatrisen for $T$ med egenvektorene som basis er $\begin{matrise}[cc]
        i\lambda_1 & -i\lambda_2 \\ 1\lambda_1 & 1\lambda_2 
        \end{matrise} = [\lambda_1\textbf{v}_1, \, \lambda_2\textbf{v}_2] $.
    \end{pkt}
\end{oppgave}
\end{document}