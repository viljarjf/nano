\subsection{Wat}
    Lag shit som ikke finnes fra scratch. 
    Forskjellen fra genetic engineering er skalaen, syntetisk biotek er mye mer omfattende endringer (eller helt fra bunnen av)

\subsection{Fordeler}
    Altså det meste kan allerede gjøres av andre disipliner (kjemi, elektronikk o.l.)
    men celler er fabrikker som kan kopiere seg selv, og du slipper å sette opp infrastruktur og sånt.
    De kan også til en viss grad reparere seg selv.

\subsection{Verktøy}
    \begin{itemize}
        \item Modellsystemer som vi forstår ganske godt (E.Coli, gjær, alger o.l.)
        \item Genbibliotek
        \item Sekvenseringsverktøy (for å øke biblioteket bl.a.)
        \item Verktøy for å lage den DNA-tråden man vil
    \end{itemize}

\subsection{Standardisering og abstrahering}
    Finn ut hvordan individuelle "komponenter" virker, standardiser det (ala resistorer, kondensatorer o.l, eller som muttere og bolter)
    sånn at vi kan abstrahere det 
    (altså vi trenger ikke vite hvordan det virker, så lenge vi vet hvordan det oppfører seg, og dermed vet hvordan vi bruker det)
