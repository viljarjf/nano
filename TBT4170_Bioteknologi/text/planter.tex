\subsection{Historie}
    krysspollinering,
    mutasjonsavling (få masse frø/totipotente celler til å mutere),
    et trekk må være arvelig,
    transgene planter (stapp andre gener inn i planter)

\subsection{Plantecellekultur}
    callus culture (fast),
    suspension culture (væske).

    Planter som stammer fra sånne har mutasjoner: 
    midlertidige forandringer,
    epigenetisk (varer hele livet og evt noen generasjoner, ofte pga metylering),
    ekte mutasjoner (endring i genomet)

\subsection{Genetic engineering of plants}
    Ti-plasmid (tumor-inducing plasmid), agrobacterium overfører T-DNA inn i planteceller og gir dem kreft, 
    men vi kan bytte ut T-DNAet med det vi vil stappe inn.

    Partikkelbombadering. Bare fest dna på gull og skyt det på planter.

    En bakteriofag har et Cre (causes recombination) system som man kan sette loxP-gen på hver side av et gen man vil kutte ut senere 
    (f.eks. hvis du har et sketchy gen du bruker for å teste om genmodifiseringen funka), som Cre-enzymet vil kjenne igjen 
    og få loxP-genene til å sette seg sammen (og dermed kutte ut det i midten).
    

    DAE CRISPR??

\subsection{GMO}
    is guud

    herbicide resistance, tørke og flom resistens, 
