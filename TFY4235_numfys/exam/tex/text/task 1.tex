\subsection*{Task 1}

I did this on paper, but did not prioritize writing it in \LaTeX.

\subsection*{Task 2}

A higher-order numerical solver can account for non-linear behaviour between 
simulation steps. As this is not the case for our potential, there is no need
to use a non-linear solver, i.e. a solver with higher order than Euler. In fact, 
the system does not really need multiple timesteps; the state on each change in 
the potential modulation function is determined explicitly by the convolution 
of the state at the previous transition and a gaussian of an appropriate variance.
Of course, the non-zero potential must also be accounted for, but with the noise
term already accounted for this is a deterministic problem with an exact solution.
One could thereby solve the system in the Fourier space instead, 
for ease of convolution, and need only calculate the state twice per $\tau$.
I did not implement this programatically, but if my algebra is correct,
this should result in the same particle distribution as the relatively naiive 
(as in, difficult to scale) particle simulation.

\subsection*{Task 3}

See Appendix A, or 
\href{
    https://www.github.com/viljarjf/nano/tree/main/TFY4235_numfys/2
    }{\color{blue}{the GitHub source}}

I decided to write the simulations in C, as I like the language. The GCC compiler
was used on WSL Ubuntu to compile the code, writing its output as binary data
directly to numpy files. These were subssequently read into python for 
visualization with matplotlib.

Source files are found in src/, header files in include/, 
python scripts in scripts/, build artefacts will be generated in build/, 
the binary will be placed in bin/, and numpy data will be placed in data/.

\subsection*{Task 4}

The C standard library was used to genreate pseudorandom numbers between 0 and 1.
These were subsequently transformed into the standard normal distribution
using the Box-Müller algorithm.

\subsection*{Task 5}

See Task 3

\subsection*{Task 6}

See Task 3

\subsection*{Task 7}

The given formula for the Boltzmann distribution is not normalized,
as a factor $\Delta U$ is missing. 

\begin{figure}
    \centering
    \includegraphics[width=\columnwidth]{figures/71.png}
    \caption{
        Simulations with the flashing off correlate well with the distribution
        proposed by statistical mechanics.
    }
    \label{fig:71}
\end{figure}

\begin{figure}
    \centering
    \includegraphics[width=\columnwidth]{figures/72.png}
    \caption{
        Simulations with the flashing off correlate with the distribution
        proposed by statistical mechanics, although seperating it from a 
        uniform distribution is difficult.
    }
    \label{fig:72}
\end{figure}

Figure \ref{fig:71} and \ref{fig:72} show the particle distribution with the potential on,
for $\Delta U = 10k_BT$ and $\Delta U = 0.1k_BT$, respectively. From this, one might observe
that the higher potential forces the particles into the bottom of the potential, 
rather than staying mostly uniform. 

\subsection*{Task 8}

A high $\Delta U$ makes the particles favour the low-energy state at the
bottom of the saw potentials. A high value would increase the speed at which
particles flow towards this state when the potential is turned on, 
reducing the time necessary to keep the potential on. 
However, since a short time with the potential on would yield 
a short time with the potential off, thereby reducing the time the particle 
has to diffuse to the other side of the potential barrier. 

\subsection*{Task 9}

\begin{figure}
    \centering
    \includegraphics[width=\columnwidth]{figures/9.png}
    \caption{
        Average drift velocity realised for 1000 particles for each $\tau$.
    }
    \label{fig:9}
\end{figure}

Figure \ref{fig:9} depicts the drift velocity as a funcion of $\tau$. 
From this, $\tau_\text{op}$ is estimated as \SI{3}{\second}.

\subsection*{Task 10}

The given figure uses $t_r = \frac{r_\text{eff}}{2D}$, 
which for the experiment in Task 9 is 
$\frac{(\alpha\SI{20}{\micro\metre})^2}{2\cdot\SI{1.92}{\micro\metre\squared\per\second}} 
= \SI{0.25}{\second}$.

The optimal flux is achieved at $\frac{t_r}{t_{off}} \approx 2$, 
i.e. $\tau = t_{off} / 0.75 = t_r / 1.5 \approx \SI{2.7}{\second}$.
This misses the estimated $\tau_{op}$ by about 10\%, 
which is deemed reasonable.

\subsection*{Task 11}

The only factor containing the radius is $\omega \propto \frac{1}{r}$, 
meaning a doubling of radius is equivalent to halving the time.
Therefore, the $\tau_{op}$ for $r_2$ should be three times higher.

\begin{figure}
    \centering
    \includegraphics[width=\columnwidth]{figures/10.png}
    \caption{
        Average drift velocity realised for 1000 particles for each $\tau$,
        for $r = 3r_1$.
    }
    \label{fig:10}
\end{figure}

Looking at the data in figure \ref{fig:10}, estimating $\tau_{op}$ 
is not feasible. \SI{9}{\second} seems reasonable enough.


\subsection*{Task 12}

\begin{figure}
    \centering
    \includegraphics[width=\columnwidth]{figures/12.png}
    \caption{
        Simulating 100 000 particles over time, they exibit regular 
        diffusion as expected.
    }
    \label{fig:12}
\end{figure}

As can be seen in figure \ref{fig:12}, the particles exibit diffusion as expected.

\subsection*{Task 13}

\begin{figure}
    \centering
    \includegraphics[width=\columnwidth]{figures/13.png}
    \caption{
        To increase the timescale, fewer particles were simulated 
        with the potential turned on than with it off. 
        The resulting final distribution is highly dependent on where in
        the potential period the simulation is stopped; if the potential 
        is off then the distribution is more spread out,
        whereas the distribution becomes more sharp peaks if the potential is on.
    }
    \label{fig:13}
\end{figure}

Figure \ref{fig:13} depicts the distribution after a few periods of the 
potential. The behaviour is like a gaussian convoluted with a delta function
for each potential minima, modulated by a gaussian travelling along the $x$-axis.
What type of differential equation that would produce such a function as its
solution is not familiar to me. 
It looks reasonably like a wave packet in fourier space, 
suggesting it might be similar to waves. 
Obviously, the original differential equation looks like the output, 
which makes every stochastic differential equation I have encountered behave like this
(sample size of 1).
