\subsection*{Planar particle system}

\begin{figure}
    \centering
    \includegraphics[width=\columnwidth]{figures/f.png}
    \caption{
        With $\vec{B} = B_0\vec{e}_z$ pointing out of the plane, a $n_x = n_y = 30$ simulation was run.
        Lighter colors means a positive $z$-component, and darker means a negative $z$-component. 
        With periodic boundary conditions, the system has stabilized into two contiguous domains;
        one in the positive and one with negative $z$-direction.
    }
    \label{fig:g}
\end{figure}

The simulation software was written with support for two-dimensional systems in mind, 
and the ground state observed in figure \ref{fig:g} indicates that the generalization is correct. 
Domains and boundaries are present in a similar way to the one-dimensional case, as should be expected.

Next, a non-zero temperature was investigated. 
Physically, we expect to see the system be generally more disordered, 
and the unstable equilibrium with $S_{j,\,z} < 0$ should be practically impossible. 
However, at low thermal energies, the dominant effect should still be the magnetic field, 
and the system is expected to be in equilibrium with $M(T) \approx 1$, where 
\begin{equation*}
    M(T) = \left<M(T, t)\right>_t = \left<\frac{1}{N}\sum_{j = 1}^NS_{j,\,z}\right>_t
\end{equation*}
is the magnetization of the material.

As the temperature increases, the magnetization is expected to fall as the thermal energy becomes more dominant. 
For a sufficiently large critical temperature $T_c$, it is expected that $M(T > T_c) \approx 0$.
In this simulation, thermal noise is modelled as a gaussian proportional to the square root of the temperature.
For the temporal average of the average $z$-component of the system to be 0, 
this term needs to be considerably larger than the magnetic term. 

\begin{figure}
    \centering
    \includegraphics[width=\columnwidth]{figures/g.png}
    \caption{
        A range of temperatures were used during simulations with two different particle amounts.
        For both simulations, $\alpha = 0.5$, a large damping factor helping the system reach equilibrium
        within a short time. The temporal average in $M(T)$ was calculated on the last \SI{10}{\pico\second}.
    }
    \label{fig:h}
\end{figure}

Simulating with $n_x = n_y = 10$ and $80$, and $\alpha = 0.5$, yields quantitatively the expected results.
As can be seen in figure \ref{fig:h}, the magnetization drops with increasing temperature, 
eventually becoming approximately zero. 
The amount of particles in the simulation plays a large role in the noise of the signal, 
and consequently $T_c$. For a $10x10$ grid, $T_c$ can be said to be around \SI{29}{\kelvin},
whereas for the $80x80$ grid, $M(T)=0$ is not reached within \SI{50}{\kelvin}.

As discussed, the magnetization decreases up to $T_c$. 
The value of $T_c$ is predicted to increase with increasing $B_0$, 
as the effect of the magnetic field becomes harder to overcome for the thermal noise.
This would lead to an interesting effect in magnetic materials: 
the maximal temperature a magnetic material can stay stable at increases 
with the thickness of the material, 
as each layer magnetically interracts with the neighbouring layers.
The implications for nanomagnets are that the material would either need to be 
cooled to near-absolute-zero temperatures, or sit in a ridiculosly strong magnetic field.

\begin{figure}
    \centering
    \includegraphics[width=\columnwidth]{figures/i.png}
    \caption{
        Increasing the value of $|\vec{B}|$ significantly increases $T_c$.
        The simulation was run on a $80x80$ grid of particles, and the temporal
        average was performed on the final \SI{5}{\pico\second} of the simulation
        for each temperature step, corresponding to 500 timesteps. 
        Temperature-steps were performed in 0.5-degree increments.
    }
    \label{fig:i}
\end{figure}

Simulations support this, as seen in figure \ref{fig:i}. 
Increasing the magnetic field dramatically increases $T_c$. 
Note, however, that a multiple-tesla magnetic field is difficult to attain for longer periods,
but having it be at the correct strength for a few picoseconds should be possible.

