\subsection*{Planar particle system}

\begin{figure}
    \centering
    \includegraphics[width=\columnwidth]{figures/f.png}
    \caption{
        With $\vec{B} = B_0\vec{e}_z$ pointing out of the plane, a $n_x = n_y = 30$ simulation was run.
        Lighter colors means a positive $z$-component, and darker means a negative $z$-component. 
        With periodic boundary conditions, the system has stabilized into two contiguous domains;
        one in the positive and one with negative $z$-direction.
    }
    \label{fig:g}
\end{figure}

The simulation software was written with support for two-dimensional systems in mind, 
and the ground state observed in figure \ref{fig:g} indicates that the generalization is correct. 
Domains and boundaries are present in a similar way to the one-dimensional case, as should be expected.

Next, a non-zero temperature was investigated. 
Physically, we expect to see the system be generally more disordered, 
and the unstable equilibrium with $S_{j,\,z} < 0$ should be practically impossible. 
However, at low thermal energies, the dominant effect should still be the magnetic field, 
and the system is expected to be in equilibrium with $M(T) \approx 1$, where 
\begin{equation*}
    M(T) = \left<M(T, t)\right>_t = \left<\frac{1}{N}\sum_{j = 1}^NS_{j,\,z}\right>_t
\end{equation*}
is the magnetization of the material.

As the temperature increases, the magnetization is expected to fall as the thermal energy becomes more dominant. 
For a sufficiently large critical temperature $T_c$, it is expected that $M(T > T_c) \approx 0$.
In this simulation, thermal noise is modelled as a gaussian proportional to the square root of the temperature.
For the temporal average of the average $z$-component of the system to be 0, 
this term needs to be considerably larger than the magnetic term. 

\begin{figure}
    \centering
    \includegraphics[width=\columnwidth]{figures/g.png}
    \caption{
        A range of temperatures were used during simulations with two different particle amounts.
        For both simulations, $\alpha = 0.5$, a large damping factor helping the system reach equilibrium
        within a short time. The temporal average in $M(T)$ was calculated on the last \SI{10}{\pico\second}.
    }
    \label{fig:h}
\end{figure}

Simulating with $n_x = n_y = 10$ and $80$, and $\alpha = 0.5$, yields quantitatively the expected results.
As can be seen in figure \ref{fig:h}, the magnetization drops with increasing temperature, 
eventually becoming approximately zero. 
The amount of particles in the simulation plays a large role in the noise of the signal, 
and consequently $T_c$. For a $10x10$ grid, $T_c$ can be said to be around \SI{29}{\kelvin},
whereas for the $80x80$ grid, $M(T)=0$ is not reached within \SI{50}{\kelvin}.

As discussed, the magnetization 


% The amplitude of the thermal noise term is dependendt only on the temperature 
% of the system, as can be seen in equation \ref{eq:Fth}.
% This means that $T_c$ should roughly follow $B_0$, as the size of $B_0$ determines
% the energy threshold to overcome to reach $M(T)=0$. 
% 
% Four different values of $B_0$, seperated by powers of two, were used to 
% investigate the dependendce of $T_c$ on $B_0$. All other parameters were 
% kept constant at $\alpha = 0.5$, and $J = $\SI{1}{\milli\electronvolt}.
% The numerical results supports the qualitative prediction, 
% as can be seen in figure \note{fix}. $T_c$ increases with increasing $B_0$.
