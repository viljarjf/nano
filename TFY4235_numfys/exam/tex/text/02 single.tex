
\subsection*{Linear particle chain}

Placing $n = 10$ particles in a line on the $y$-axis, i.e. $n_x = 1$, $n_y = 10$, and with
$\vec{S}_j^0 = \vec{e}_z \,\forall j_y>1$ and $\vec{S}_1^0 = \left[-0.01,\,0,\,1\right]$, 
the results presented in figure \ref{fig:c1} were achieved. 

\begin{figure}
    \centering
    \includegraphics[width=\columnwidth]{figures/c.png}
    \caption{
        10 particles in a row, where only the first particle is disturbed initially.
        Note the wave-like behaviour of the system as the disturbance propagates.
    }
    \label{fig:c1}
\end{figure}

As each particle is exited by its previous neighbour for the first time, the phase
delay tends towards $\frac{\pi}{2}$. This is most easily seen by observing that the initial peak
of each particle's disturbance coincides with the previous particle crossing 0.
Furthermore, as the next particle is exited, the "donor" interracts with it again, 
this time recieving more than it exites. This is most easily seen in particle 9's peak, 
where the neighbouring particle 8 clearly hesitates in its path towards 0. 

Particle 9's peak is also notably larger than the other ones, 
likely a result of the coupling term at the edges being half of the corresponding values in the bulk.
This means that it does not transmit the energy of the wave to its next neighbour, 
but rather keeps it until it is sent back along the opposite way.

After some time, the phase shift reduces as the amplitude gets lower, eventually
becoming zero. This can be seen in figure \ref{fig:c2}. Also note the exponentially decaying amplitude,
similar to what was observed for a single particle. 

\begin{figure}
    \centering
    \includegraphics[width=\columnwidth]{figures/c2.png}
    \caption{
        10 particles in a row, where only the first particle is disturbed initially.
        After some time, the particles more or less align with each other.
    }
    \label{fig:c2}
\end{figure}

With $\alpha = 0$, the phase shift reduction due to amplitude decay should not be present.
Instead, one would expect standing waves to appear in the system as the disturbance propagates
back and fourth. There should be a multitude of smaller waves travelling at different speeds to the
initial disturbance, as the end points continue to affect their neighbours as they precess.

\begin{figure}
    \centering
    \includegraphics[width=\columnwidth]{figures/d1.png}
    \caption{
        30 particles in a row wihout damping, where only the left-most particle is disturbed initially.
        As time propagates downwards, a wave appears to be travelling along the chain.
    }
    \label{fig:d1}
\end{figure}

Performing the simulation with $\alpha = 0$ and $n_y = 30$ being the only differences from the previous case, 
the results presented in figure \ref{fig:d1} were obtained. 
From these, one might obsere that the wave behaves as expected, bouncing back and fhourth between the boundaries.
Note also the presence of a much slower wave originating from the boundary each time the wave leaves, 
travelling at roughly half the speed of the main wave. This is
exemplified most clearly in the dark band seen originating from the initial disturbance. 

Were the simulation to be performed with periodic boundary conditions in the $y$-direction, 
one would expect the system to have two large waves propagating in opposite directions. 
For a short time, i.e. until the initial disturbances meet in the middle, 
the system should be equivalent to placing the disturbance in the center in the previous simulation.
As previously, without the $\alpha$-term, one would expect the waves to continue propagating with the same amplitude,
within some error caused by the other disturbances.

\begin{figure}
    \centering
    \includegraphics[width=\columnwidth]{figures/e2.png}
    \caption{
        30 particles in a row wihout damping, where only the left-most particle is disturbed initially.
        With periodic boundary conditions, this immidiately interracts with the right-most particle.
    }
    \label{fig:e1}
\end{figure}

\begin{figure}
    \centering
    \includegraphics[width=\columnwidth]{figures/e.png}
    \caption{
        With periodic boundary conditions and damping off, 
        two neighbouring particles become out of phase by \SI{180}{\degree}.
    }
    \label{fig:e2}
\end{figure}

Simulations align well with the predicted behaviour. 
As seen in figure \ref{fig:e1}, there are two waves travelling in opposite directions, 
eventually meeting and passing through each other. 
Interestingly, neighbouring particles are out of phase by $\pi$, as opposed to $\frac{\pi}{2}$ with damping.
This can be seen in figure \ref{fig:e2}, plotting the first and last particle's $x$-component of $\vec{S}_j$.

The coupling term $J$ dictates how strongly the atoms interract with each other, but also how they interract.
For a $dz > 0$ and $B_0 = 0$, one should expect the atoms to eventually align in the $z$-direction as this
minimizes the energy of the system. If the initial state is random, one might observe domains of different $z$-directions,
where some atoms have aligned in the opposite direction of the rest. Between these domains are boundaries
where the atoms are misaligned with the $z$-axis, which would mean the system is in a local but not a global minima. 
If $J$ is positive, the energy is lower if the atoms are parallel, and antiparallel for negative $J$. 

\begin{figure}
    \centering
    \includegraphics[width=\columnwidth]{figures/f1.png}
    \caption{
        With periodic boundary conditions, damping on, and $dz = 0.1$,
        the two different signs of $J$ were investigated. 
        a) shows $J = \SI{1}{\milli\electronvolt}$, and
        b) shows $J = \SI{-1}{\milli\electronvolt}$
    }
    \label{fig:f}
\end{figure}

Running simulations for $|J| = \SI{1}{\milli\electronvolt}$, $n_y = 50$ and $\alpha = 0.5$, 
the predicted behaviour seems to be adhered to. Figure \ref{fig:f} shows the two situations,
and the expected domains are clearly seen in the $S_{j,\,z}$-plot in a). 
The boundaries are clearly visible as lines in the $x$- and $y$-plots in a) as well.
From b), the misalignment of neighbours is clearly visible as the stripes in the $z$-plot.
Note the anihilation of the boundaries in b), which clearly transmits the energy in the 
boundaries as waves through the material, visible as the non-zero vaues in the $x$- and $y$-plots
right after the two boundaries collide. Reducing the amount of atoms on the boundary is clearly
energetically preferrable, as the boundaries accelerate towards each other right before anihilation.
A similar scenario is seen in a), where the wave created by the boundary in the positive domain
can be seen as a dark expanding region in the $x$-plot. 

A similar simulation was performed for $n_x = n_y = 10$, and animated as GIFs. 
The ferromagnetic case can be found 
\href{https://random.timini.no/viljarjf/ferromagnet.gif}{\color{blue}{here}}, 
and the antiferromagnetic 
\href{https://random.timini.no/viljarjf/antiferromagnet.gif}{\color{blue}{here}}.
Both of these were performed with a magnetic field rather than the anisotropic term, 
and as such boundaries are much more unlikely as the magnetic field causes the 
negative-$z$ equilibrium to be unstable. Regardless, they are interesting to look at.
