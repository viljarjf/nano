\subsection*{Introduction}
Using the Heisenberg model, the behaviour of spin in a material was investigated.
A material was represented as a $n_x$ by $n_y$ two-dimensional equidistant lattice, 
where each point in the lattice had three degrees of freedom for the direction of its spin.
All spins were normalized at all times. 

The Hamiltonian of the system is given by 
\begin{equation}
    \label{eq:lewis}
    H = - \sum_{j \neq k}^NJ_{jk}\vec{S}_j\cdot\vec{S}_k
        - dz \sum_{j = 1}^N(\vec{S}_j\cdot\vec{e}_z)^2
        - \mu_s \sum_{j = 1}^N \vec{S}_j \cdot \vec{B}_j,
\end{equation}
where a subscript $j$ refers to a given position $j_x,\, j_y$, $J_{jk}$ is a matrix describing coupling, 
$N = n_xn_y$ is the amount of lattice points,
$dz$ is a anisotropy coefficient, and $\mu_s$ is the magnetic moment. 
The time-dependence of $\vec{S}_j$ is given by the Landau-Lifshitz-Gilbert equation:
\begin{equation}
    \label{eq:ode}
    \frac{\partial}{\partial t} \vec{S}_j = \frac{-\gamma}{1 + \alpha^2}
    \left(\vec{S}_j\times\vec{F}_j + \alpha\vec{S}_j \times \left(\vec{S}_j\times\vec{F}_j\right)\right),
\end{equation}
where
\begin{equation}
    \label{eq:F}
    \vec{F}_j = -\frac{1}{\mu_s}\frac{\delta H}{\delta \vec{S}_j} 
                + \sqrt{\frac{2\alpha k_BT}{\gamma\mu\Delta t}}\Gamma,
\end{equation}
with the gyromagnetic ratio $\gamma$, a damping coefficient $\alpha$, 
thermal energy $k_BT$, and a 3D standard normal gaussian $\Gamma$.

To simplify the system, $J_{jk} = J$ where $j$ and $k$ are nearest neighbours, else 0. 
Furthermore $\vec{B}_j = B_0\vec{e}_z \,\,\forall j$ and is temporally constant.

To solve the differential equation \ref{eq:ode}, Heun's method was utilized. 
In general, Heun's method states
\begin{align*}
    \vec{\tilde{S}}_j^{\tau + 1} &= \vec{S}_j^\tau + \Delta t f(\vec{S}_j^\tau)\\
    \vec{S}_j^{\tau + 1} &= \vec{S}_j^\tau + 
            \frac{\Delta t}{2}\left(f(\vec{S}_j^\tau) + f(\vec{\tilde{S}}_j^{\tau + 1})\right),
\end{align*}
where, in this case, $f$ is the right-hand side of equation \ref{eq:ode}.
Equation \ref{eq:F} can be further simplified into 
\begin{equation}
    \label{eq:simp}
    \frac{J}{\mu_s}\sum_{\text{neighbours}\,k}\vec{S}_k
        + \left(\frac{2dzS_{j, z}}{\mu_s} + B_0\right)\vec{e}_z
        + \sqrt{\frac{2\alpha k_BT}{\gamma\mu\Delta t}}\Gamma
\end{equation}
by performing the differentiation.

The program to run the simulation was written in Python, based mainly on the library Numpy,
and some code further optimized with Numba and outsourced to Scipy. 
The program architecture was based mainly on using vectorized operations wherever possible. 
This allows for full utilization of the optimized lower-level Cython-based libraries. Where such methods were slow,
Numba was used to compile iteration-based functions instead.
To increase generality, a library was written to create functions for equations 
\ref{eq:lewis} and \ref{eq:F}, based on the set of constants used in these. 
These functions were then passed on to the Heun solver, along with an initial condition.
This allows for a simple interface for further tinkering with the system.

Random numbers for the $\Gamma$-term was generated using Numpy's pseudorandom generators.
Each axis was sampled independently from a standard normal gaussian.
For generating random initial conditions, Numpy's pseudorandom sublibrary was used to generate
one pseudorandom number between -1 and 1 for each axis. Since these are subsequently normalized by projecting
onto the unit sphere, a bias towards the corners of the unit cube is observed in the sampling.
For the purposes of this simulation, this bias is not considered detrimental, and therefore not accounted for.

The program was run on a 8-core AMD Ryzen 9 5900HS CPU with 16GB of RAM, 
with some overnight calculations run on a 12-core AMD Ryzen 9 3900X CPU with 64 GB of RAM.
Multithreading the Numba operations allowed for 100\% utilization on all cores.
Since simple matrix operations typically are faster to run on a GPU than a CPU,
initial testing was performed on an Nvidia RTX 3080 GPU with 10 GB VRAM. This revealed little performance gain, 
and further development of GPU-optimized code was subsequentally dropped.

Unless otherwise specified, the following default parameters were used: 
\begin{itemize}
    \item $J = \SI{1}{\milli\electronvolt}$
    \item $dz = \SI{0}{\milli\electronvolt}$
    \item $\mu_s = \mu_b = \SI{0.0579}{\milli\electronvolt\per\tesla}$
    \item $B_0 = \SI{1.72}{\tesla}$
    \item $k_B = \SI{0.0862}{\milli\electronvolt\per\kelvin}$
    \item $\gamma = \SI{0.176}{\per\tesla\per\pico\second}$
    \item $\Delta t = \SI{1}{\femto\second}$
    \item $T = \SI{0}{\kelvin}$
\end{itemize}
