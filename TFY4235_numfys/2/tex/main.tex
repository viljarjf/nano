\documentclass{labreport}
\usepackage[utf8]{inputenc}

\usepackage[backend=biber,style=vancouver]{biblatex}
\usepackage{csquotes}

\newcommand{\candnum}{\note{FIX}}


\title{Diffusion of particles in a non-uniform transient potential}
\shorttitle{Assignment 2}
%\author{Viljar Johan Femoen \affiliation{Department of Physics, NTNU, Trondheim}} 
%\shortauthor{Femoen}
\author{Candidate number \candnum}
\shortauthor{\candnum}
\labreportreceived{\today}  
\labreportpublished{}
\course{TFY4235 Numerical Physics}
% \labreportyear{2021}
 
\addbibresource{bibliography.bib}

\begin{document}
\maketitle

\begin{abstract}
Semiconductor processing and manufacturing is becoming increasingly important for modern technology, both as electronic processing units and as other micro- and nano-scale components. Two \ce{GaAs} chips, labeled chip 218 and chip 219, were manufactured using molecular beam epitaxy, and a Hall bar was created using photolithography. The characterisation indicates that chip 218 was $n$-doped, and 219 was $p$-doped. The carrier density of chip 218 was found to be \SI{5.5(0.0)e17}{\per\centi\metre\cubed}, and \SI{1.3(0.0)e18}{\per\centi\metre\cubed} for chip 219.
For chip 218, the Hall mobility was found to be \SI{3.3(1)e3}{\centi\metre\squared\per\volt\per\second}, whereas for chip 219 it was found to be \SI{1.8(0)e2}{\centi\metre\squared\per\volt\per\second}.
\end{abstract}

\section{Introduction}

ja
det
var
olav
TRYGGVASON

\begin{center}
    \rule{2cm}{.4pt}
\end{center}
\makeatletter
\@beginparpenalty=10000
\makeatother
\nocite{*}
\printbibliography
\end{document}
