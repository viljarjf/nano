\section{Introduction}

% The first part of the introduction should be a presentation of what we are investigating (in our case semiconductor doping), and the motivation for measuring a semiconductor sample at all. Start out very general and move to the specific experiment you have performed.

% Then you present shortly what you have done to ”help exploring” the important and interesting things you mentioned above. This presentation should not be a full summary of your process, but a short section on how you did it.

% Whenever you present a number/parameter/result in your report, be aware of a few things: English decimal mark is ”.” and not ”,” as in Norwegian. The value should be followed by a white space and correct units that are not italicized. Multiplication is done with a multiplication sign such as · or ×, not ∗.

% The purpose of the lab is to manufacture a hall bar and characterize it. This includes finding the carrier concentration and mobility of the material and the contact resistance of the contacts. In addition, you should find the calibration curve for the Si source using data from other groups

Semiconductor processing and manufacturing is becoming increasingly important for modern technology, both as electronic processing units and as other micro- and nano-scale components. To achieve the desired properties, one may change the electrical properties of the semiconductor through doping. The effect of doping in a semiconductor will be further studied with a Hall bar, which enables one to measure the type and level of doping in a semiconductor. For this study, an $n$-doped and a $p$-doped Hall bar were fabricated, and later characterised.
