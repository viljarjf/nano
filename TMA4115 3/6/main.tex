\documentclass[11pt, a4paper, norsk]{NTNUoving}
\usepackage[utf8]{inputenc}
\usepackage[T1]{fontenc}
\usepackage{tikz}

\ovingnr{6}    % Nummer på innlevering
\semester{Våren 2020}
\fag{TMA 4115} %4105 er matte 2, 4115 er matte 3
\institutt{Institutt for matematiske fag}

\newenvironment{pkt}{\begin{punkt}}{\end{punkt}}

\newenvironment{matrise}[1][c]
{
\left[\begin{array}{#1}
}
{    
\end{array}\right]           
}

\newenvironment{indreprod}{
    \langle
}{
\rangle}

\newcommand{\R}{\mathbb{R}}
\renewcommand{\P}{\mathcal{P}}

\begin{document}
%#################################################
%Dette er for enkel copy-pasting
\ifx
%-------------------------
\begin{oppgave}
    \begin{punkt}
        \begin{align*}
        
        
        \end{align*}
    \end{punkt}
\end{oppgave}
%-------------------------
\begin{tikzpicture}
        \draw[step=1cm,gray,very thin] (-5.9,-5.9) grid (5.9,5.9);
        \draw (-6,0) -- (6,0);
        \draw (0,-6) -- (0,6);
        \draw[->] (-6,0) -- (6,0); 
        \draw[->] (0,-6) -- (0,6);
        \draw (0, 6.2) node {y};
        \draw (6.2, 0) node {x};
\end{tikzpicture}
\fi

%Her begynner dokumentet
%#####################################
\begin{oppgave}
    \begin{pkt}
    
        Den transponerte av venstresiden er $$\begin{matrise}[ccc]
            2 & -3 & -1 \\1 & 1 & 1 
        \end{matrise}$$
        Ganger vi den med begge sider får vi $$\begin{matrise}[cc|c]
            14 & -2 & 3\\ -2 & 3 & -2
        \end{matrise} \sim \begin{matrise}[cc|c]
            14 & -2 & 3\\ 0 & 19 & -11
        \end{matrise} \sim \begin{matrise}[cc|c]
            14 & 0 & \frac{35}{19}\\ 0 & 1 & -\frac{11}{19}
        \end{matrise} \sim \begin{matrise}[cc|c]
            1 & 0 & \frac{5}{38}\\ 0 & 1 & -\frac{11}{19}
        \end{matrise}$$
        Minste kvadraters metode gir dermed $\frac{1}{38}\begin{matrise}
        5  \\ -22 
        \end{matrise}$
    \end{pkt}
    \begin{pkt}
        Den adjungerte av venstresiden er $$\begin{matrise}[cccc]
            0 & -i & 0 & 0 \\ 1 & -i & -i & -i \\ 1 & -1 & 0 & 1 
        \end{matrise}$$
        Ganger vi den med begge sider får vi 
        \begin{align*}
        &\begin{matrise}[ccc|c]
            1 & 1 & i & 1-i \\ 1 & 4 & 1 & 3-3i \\ -i & 1 & 3 & 1-2i
        \end{matrise} \sim \begin{matrise}[ccc|c]
            1 & 1 & i & 1-i \\ 0 & 3 & 1-i & 2-2i \\ 0 & 1+i & 2 & 2-i
        \end{matrise} \sim \begin{matrise}[ccc|c]
            1 & 1 & i & 1-i \\ 0 & 3 & 1-i & 2-2i \\ 0 & 0 & \frac{4}{3} & \frac{2}{3}-i
        \end{matrise} \\ \sim& \begin{matrise}[ccc|c]
            1 & 1 & i & 1-i \\ 0 & 3 & 0 & \frac{9-3i}{4} \\ 0 & 0 & 1 & \frac{2-3i}{4}
        \end{matrise} \sim \begin{matrise}[ccc|c]
            1 & 1 & i & 1-i \\ 0 & 1 & 0 & \frac{3-i}{4} \\ 0 & 0 & 1 & \frac{2-3i}{4}
        \end{matrise} \sim \begin{matrise}[ccc|c]
            1 & 0 & 0 & \frac{-2-5i}{4} \\ 0 & 1 & 0 & \frac{3-i}{4} \\ 0 & 0 & 1 & \frac{2-3i}{4}
        \end{matrise}
        \end{align*}
        Minste kvadraters metode gir dermed $\frac{1}{4}\begin{matrise}[c]
        -2-5i \\ 3-i \\ 2-3i
        \end{matrise}$
    \end{pkt}
\end{oppgave}

\begin{oppgave}
    \begin{pkt}
        $f(x)=ax^4+bx^3+cx^2+dx+e$
        \begin{align*}
            \begin{matrise}[ccccc|c]
                0 & 0 & 0 & 0 & 1 & 1\\
                1 & 1 & 1 & 1 & 1 & 2\\
                16 & 8 & 4 & 2 & 1 & 3\\
                81 & 27 & 9 & 3 & 1 & 5\\
                256 & 64 & 16 & 4 & 1 & 7
            \end{matrise}
        \end{align*}
        Denne løser jeg ikke for hånd. Løsningen er $$\begin{matrise}
        a\\b\\c\\d\\e
        \end{matrise}=\frac{1}{12}\begin{matrise}
        -1\\ 8 \\ -17 \\ 22 \\ 12
        \end{matrise}$$
    \end{pkt}
    \begin{pkt}
    Totalmatrisen vi skal bruke minste kvadraters metode på er følgende:
        \begin{align*}
            \begin{matrise}[ccc|c]
                 0 & 0 & 1 & 1\\
                 1 & 1 & 1 & 2\\
                 4 & 2 & 1 & 3\\
                 9 & 3 & 1 & 5\\
                 16 & 4 & 1 & 7
            \end{matrise}
        \end{align*}
        Den transponerte av venstresiden er 
        \begin{align*}
            \begin{matrise}[ccccc]
                 0 & 1 & 4 & 9 & 16\\
                 0 & 1 & 2 & 3 & 4\\
                 1 & 1 & 1 & 1 & 1
            \end{matrise}
        \end{align*}
        Ganger vi denne med begge sider får vi
        \begin{align*}
            \begin{matrise}[ccc|c]
                354 & 100 & 30 & 171 \\
                100 & 30 & 10 & 51 \\
                30 & 10 & 5 & 18
            \end{matrise}
        \end{align*}
        Denne gidder jeg heller ikke gausseliminere for hånd. Løsningen er $$\begin{matrise}
        a\\b\\c
        \end{matrise} = \begin{matrise}
        \frac{3}{14} \\ \frac{9}{14} \\ \frac{36}{35}
        \end{matrise}$$
    \end{pkt}
\end{oppgave}
\begin{oppgave}
    \begin{pkt}
        \begin{align*}
            \begin{matrise}[cc]
                -0.2 & 0.5 \\ 0.2 & -0.5
            \end{matrise} \sim \begin{matrise}[cc]
                -2 & 05 \\ 0 & 0
            \end{matrise} 
        \end{align*}
        Det gir at likevektsvektoren er $\frac{1}{7}\begin{matrise}
        5\\2
        \end{matrise}$
    \end{pkt}
    \begin{pkt}
        \begin{align*}
            &\begin{matrise}[ccc]
                -0.3 & 0.2 & 0.2 \\ 0 & -0.8 & 0.4 \\ 0.3 & 0.6 & -0.6
            \end{matrise} \sim \begin{matrise}[ccc]
                -3 & 2 & 2 \\ 0 & -8 & 4 \\ 0 & 8 & -4
            \end{matrise} \sim \begin{matrise}[ccc]
                -3 & 2 & 2 \\ 0 & -2 & 1 \\ 0 & 0 & 0
            \end{matrise}\\ \sim& \begin{matrise}[ccc]
                -3 & 0 & 3 \\ 0 & -2 & 1 \\ 0 & 0 & 0
            \end{matrise} \sim \begin{matrise}[ccc]
                -1 & 0 & 1 \\ 0 & -2 & 1 \\ 0 & 0 & 0
            \end{matrise}
        \end{align*}
        Det gir at likevektsvektoren er $\frac{1}{5}\begin{matrise}
        2\\1\\2
        \end{matrise}$
    \end{pkt}
\end{oppgave}

\begin{oppgave}
    \begin{pkt}
    Ja
    \end{pkt}
    \begin{pkt}
        Her blir $a_{1\,2}=0$ for $A^n\,\forall\, n\geq 1$, så den er ikke regulær.
    \end{pkt}
\end{oppgave}

\begin{oppgave}
    Den stokastiske matrisen for systemet er slik: 
    \begin{align*}
        \begin{matrise}[ccc]
            0.7 & 0.1 & 0.1\\
            0.2 & 0.8 & 0.2 \\
            0.1 & 0.1 & 0.7
        \end{matrise}
    \end{align*}
    Den stabile tilstanden er gitt ved likevektsvektoren.
    \begin{align*}
        &\begin{matrise}[ccc]
            -0.3 & 0.1 & 0.1\\
            0.2 & -0.2 & 0.2 \\
            0.1 & 0.1 & -0.3
        \end{matrise} \sim \begin{matrise}[ccc]
            -3 & 1 & 1\\
            0 & -4 & 8 \\
            0 & 4 & -8
        \end{matrise} \sim \begin{matrise}[ccc]
            -3 & 1 & 1\\
            0 & -1 & 2 \\
            0 & 0 & 0
        \end{matrise}\\ \sim& \begin{matrise}[ccc]
            -3 & 0 & 3\\
            0 & -1 & 2 \\
            0 & 0 & 0
        \end{matrise} \sim \begin{matrise}[ccc]
            -1 & 0 & 1\\
            0 & -1 & 2 \\
            0 & 0 & 0
        \end{matrise}
    \end{align*}
    Det gir at likevektsvektoren er $\frac{1}{4}\begin{matrise}
    1\\2\\1
    \end{matrise}$. Det er dermed best å smøre for temperaturer rundt 0, siden det er 50\% sannsynlighet for det, og 25\% for de to andre.
\end{oppgave}

\begin{oppgave}
    Alle regulære stokastiske matriser har alltid en og bare en likevektsvektor. Men hvis det skal vises: 
    \begin{align*}
        \begin{matrise}[cc]
            1-a-1 & b\\a & 1-b-1
        \end{matrise} \sim \begin{matrise}[cc]
            -a & b\\0&0
        \end{matrise}
    \end{align*}
    Som gir den unike likevektsvektoren $\begin{matrise}
    b \\ a
    \end{matrise}$
\end{oppgave}
\end{document}