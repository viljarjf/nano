\subsection{Hva er en celle}
    En dynamisk enhet som utgjør en fundamental enhet av liv. 

    Har også en cytoplasmamembran som skiller inn fra ut. 

    Alle celler: 
    \begin{itemize}
        \item Metabolisme: Ta opp og bruke næringsstoffer, og kvitte seg med avfall
        \item Vekst: Lag flere celler av ting i omgivelsene
        \item Evolusjon: mutasjoner og shit skaper nye egenskaper
    \end{itemize}

    Noen celler: 
    \begin{itemize}
        \item differensiering: Nye cellestrukturer 
        \item Kommunisering: Noen celler kan kommunisere med andre celler rundt seg
        \item Utveksle gener: Noen celler kan gi/dele/bytte dna med andre uten formering
        \item Motilitet: Noen celler har flageller og kan bevege seg
    \end{itemize}


\subsection{Prokaryote}
    Alle: 
    \begin{itemize}
        \item Cytoplasmamembran
        \item Cytoplasma
        \item Nukleoid (ikke kjerne, altså nukleus)
        \item Ribosomer
    \end{itemize}

    Noen:
    \begin{itemize}
        \item cellevegg
        \item Flageller, pili
        \item plasmider
    \end{itemize}

    Gram +: cellevegg av peptidoglykan

    Gram -: tynn cellevegg av peptidoglykan under en ytre membran 

\subsection{Eukaryote}
    Alle: 
    \begin{itemize}
        \item Cytoplasmamembran
        \item Cytoplasma
        \item Ribosomer
        \item Nucleus
        \item Mitochondria
        \item endoplasmatisk retikulum: Transportfordelingssenter
        \item Golgiapparatet: Behandle og pakke inn makromolekyler
    \end{itemize}

    Noen: 
    \begin{itemize}
        \item Kloroplast
        \item Cellevegg
    \end{itemize}

\subsection{Cytoplasmamembranen}
    Dilipid-dag av fosfolipider

    Funksjon: skille mellom utenfor og innenfor, har masse pumper og diffusjonskanaler og shit (membranproteiner)
    Er også et anker for transporproteiner. Sist, men ikke minst, "Energy conservation: Site of generation and use of the proton motive force"

    Steroler (hopanoider i bakterier) styrker den strukturelle integriteten.

    Integralprotein: helt gjennom/inni

    periferalprotein: en ende inni

    H$_2$O diffunderer passivt, makromolekyler og ioner kommer ikke gjennom, men noen kan pumpes. Noen pumper trenger ikke ATP.

\subsection{Metabolisme}
    def: "the set of life sustaining chemical transformations within cells"

    som regel delt inn i to kategorier: 

    Catabolism: Breaks down organic matter and harvests energy by way of cellular respirations

    Anabolism: Uses energy to construct components of cells such as proteins and nucleic acids.

    metabolsk diversitet: ikke alle driver med glukose de finner. planter lager glukose, noen loker rundt med svovelbasert shit, 
    men basically så lager alle ATP. 

    Chemoorganotrophs: Oksiderer organiske molekyler. Noen aerobe, noen anaerobe. 

    Chemolithotrophs: oksiderer inorganiske molekyler. 

    Phototrophs: lys, kan også IKKE produsere oksygen

    Autotroph: CO$_2$ som hovedkilde til karbon

    Hetrotroph: trenger andre molekyler for karbon. 

    Extremophiles: Trenger ekstreme forhold, eks havbunnskorsteiner.


\subsection{Energi}
    kilder: nevnt over I guess

    Bruker ATP. U know the shit.

    Har og NAD, FAD for elektron carriers. 

    Lagring over tid skjer i Glykogen, stivelse, lipider o.l. (Poly-$\beta$-hydroxybutyrateand other polyhydroxyalkanoates)

\subsection{Anaerob og aerob metabolisme}
    aerob: stivelse/glykogen -> glukose -> pyrovat -> acetyl-CoA -> TCA (sitronsyresyklusen). opp til 38 ATP

    anaerob: stivelse/glykogen -> glukose -> pyrovat -> ethanol / melkesyre / acetic acid / formic acid. Rundt 1-4 ATP

    Tilstedeværelse av oksygen kan bestemme hvilke gener som kommer til utrykk (siden forskjellige enzymer trengs)

\subsection{Makromolekyler}
    Makromolekyler, byggesteiner, og funksjoner

    Karbohydrater (monosakkarider. energi, cellevegg), 
    lipider (glyserol, fettsyrer, fosfat. membraner), 
    peptider (amminosyrer. enzymer, struktur, transport, signal), 
    nukleidsyrer (nukleotider, fosfat, sukker, nitrogenbaser. informasjonslagring, transkribsjon, translasjon). 
   