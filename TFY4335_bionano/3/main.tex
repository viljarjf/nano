\documentclass[a4paper, 12pt]{article}
\usepackage[utf8]{inputenc}
\usepackage{siunitx}
\usepackage{amsmath}
\usepackage{amssymb}
\usepackage{esvect}
\usepackage{esint}
\usepackage[cmr]{emf}
\usepackage{color}   
\usepackage{hyperref}
\usepackage[table]{xcolor}
\usepackage{bm}
\usepackage{titlesec}
\usepackage{longtable}
\usepackage{multirow}
\usepackage{physics}
\newcommand{\sectionbreak}{\clearpage} % new page for each new section
\hypersetup{
    colorlinks=true, 
    linktoc=all,     
    linkcolor=blue,  
}

\title{Exercise 3, Nano Life Science}
\author{Viljar Femoen}
\makeatletter
%\renewcommand{\vec}{\vv}
\renewcommand{\vec}[1]{\bm{#1}}
\renewcommand{\phi}{\varphi}
\newcommand*{\defeq}{\stackrel{\text{def}}{=}}

% vector shortcuts
\newcommand{\E}{\ensuremath{\vec{E}}}
\renewcommand{\epsilon}{\varepsilon}
\newcommand{\e}{\ensuremath{\epsilon_0}}
\newcommand{\p}{\ensuremath{\vec{P}}}
\newcommand{\D}{\ensuremath{\vec{D}}}
\renewcommand{\j}{\ensuremath{\vec{J}}}
\newcommand{\B}{\ensuremath{\vec{B}}}
\renewcommand{\H}{\ensuremath{\vec{H}}}

% add equation numbering to align*
\newcommand\numberthis{\addtocounter{equation}{1}\tag{\theequation}}

% pasta from https://tex.stackexchange.com/questions/459167/typesetting-dalembertian-symbol?rq=1
\newcommand{\dalambert}{\mathop{\mathpalette\dalembertian@\relax}}
\newcommand{\dalembertian@}[2]{%
  \begingroup
  \sbox\z@{$\m@th#1\square$}%
  \dimen0=\fontdimen8
    \ifx#1\displaystyle\textfont\else
    \ifx#1\textstyle\textfont\else
    \ifx#1\scriptstyle\scriptfont\else
    \scriptscriptfont\fi\fi\fi3
  \makebox[\wd\z@]{%
    \hbox to \ht\z@{%
      \vrule width \dimen0
      \kern-\dimen0
      \vbox to \ht\z@{
        \hrule height \dimen0 width \ht\z@
        \vss
        \hrule height 2\dimen0
      }%
      \kern-2.5\dimen0
      \vrule width 2.5\dimen0
    }%
  }%
  \endgroup
}

\let\tmp\hat
\renewcommand{\hat}[1]{\vec{\tmp{#1}}}

%\renewcommand{\nabla}{\text{aruran}}

\makeatother

\begin{document}
\maketitle
\section*{Task 1}

The Navier-Stokes equation states 
\begin{equation*}
    \pde{\vec{v}}{t} + \left(\vec{v} \cdot \nabla\right)\vec{v} = - \frac{1}{\rho}\nabla p + \frac{\eta}{\rho}\nabla^2\vec{v}.
\end{equation*}

In our case, $\vec{v} = v\hat{z}$, and we have cylindrical symmetry, meaning $\left(\vec{v} \cdot \nabla\right)\vec{v} =  0.$
Furthermore, $\nabla p = \frac{\Delta p}{l}\hat{z}$, and we assume steady-state, i.e. $\pde{v}{t} = 0$, which means we can write the above equation as
\begin{equation*}
    \nabla^2\vec{v} = \frac{\Delta p}{l\eta}\hat{z}.
\end{equation*}

With cylindrical coordinates, the left-hand-side evaluates to

\begin{equation*}
    \nabla^2 v_z\hat{z} = \frac{1}{r}\pde{}{r}\left(r\pde{v_z}{r}\right)\hat{z}. 
\end{equation*}

Using the chain rule, we arrive at the following ODE:

\begin{equation*}
    \pde[2]{v_z}{r} + \frac{1}{r}\pde{v_z}{r} = \frac{\Delta p}{l\eta}.
\end{equation*}

Laplace transforming this, we arrive at 

\begin{equation*}
    -\frac{d}{ds}\left(s^2\tilde{v}_z - sv_z(0) + \pde{v_z}{r}_{r = 0}\right) + s\tilde{v}_z - v_z(0) = \frac{\Delta p}{l\eta s^2}.
\end{equation*}

We have the no-slip condition, i.e. $v_z(0) = 0$, and as such the above equation becomes

\begin{align*}
    \tilde{v}_z + s\ode{\tilde{v}_z}{s} &= -\frac{\Delta p}{l\eta s^3} \\
    \ode{}{s}\left(s\tilde{v}_z\right) &= -\frac{\Delta p}{l\eta s^3} \\
    s\tilde{v}_z &= c_1 - \frac{\Delta p}{2l\eta s^2} \\
    \tilde{v}_z &= \frac{c_1}{s} - \frac{\Delta p}{2l\eta s^3},
\end{align*}
for an arbitrary real constant $c_1$. Inverse laplace transforming back to real space, we arrive at

\begin{equation*}
    v_z(r) = c_1 - \frac{\Delta pr^2}{4l\eta}.
\end{equation*}

Re-applying the no-slip condition, we get the following equation for $c_1$:

\begin{equation*}
    0 = c_1 - \frac{\Delta p\left(\frac{d}{2}\right)^2}{4l\eta},
\end{equation*}
which finally lets us write
\begin{equation*}
    v_z(r) = \frac{\Delta p}{4l\eta}\left(\frac{d^2}{4} - r^2\right).
\end{equation*}

$\left<v\right>_r$ is given by

\begin{equation*}
    \int_{-\infty}^\infty rv_z(r)dr = \frac{\Delta p}{4l\eta}\int_0^\frac{d}{2} r\left(\frac{d^2}{4} - r^2\right)dr = \frac{\Delta p d^4}{256l\eta}.
\end{equation*}

The maximum velocity, $v_\text{max} = v(r = 0) = \frac{\Delta p d^2}{16l\eta}$, is a factor $\frac{d^2}{16}$ away from the average. 
Therefore, the smaller the channel, the smaller the relative difference between the max and the average will be.

The volumetric flow rate is given by

\begin{equation*}
    Q = \iint v dA = \int_0^{2\pi}\int_0^\frac{d}{2} v rdrd\theta 
    = \frac{\Delta p}{4l\eta}\int_0^{2\pi} \int_0^\frac{d}{2} r\left(\frac{d^2}{4} - r^2\right)drd\theta 
    = \frac{\pi\Delta p d^4}{128l\eta},
\end{equation*}
which was to be shown. 

\section*{Task 2}

We know

To compute $\left<x^2\right>$, one needs to evaluate the integral

\begin{equation*}
    \int_{-\infty}^\infty x^2 p(x) dx = \int_{-\infty}^\infty x^2 e^{-\beta k\frac{x^2}{2}} dx = \sqrt{\frac{2\pi}{k^3\beta^3}},
\end{equation*}

and normalize, i.e. divide by 

\begin{equation*}
    \int_{-\infty}^\infty p(x) dx = \sqrt{\frac{2\pi}{k\beta}}.
\end{equation*}

$p(x)$ is the probability of finding a particle at position $x$, and is given by the Bolzmann distribution: $e^{-\beta U(x)}$.

Rearranging, we get

\begin{eqnarray}
    k = \frac{k_bT}{\left<x^2\right>},
\end{eqnarray}
which was to be shown.
\end{document}