\documentclass[a4paper, 12pt]{article}
\usepackage[utf8]{inputenc}
\usepackage{siunitx}
\usepackage{amsmath}
\usepackage{esvect}
\usepackage{esint}
\usepackage[cmr]{emf}
\usepackage{color}   
\usepackage{hyperref}
\hypersetup{
    colorlinks=true, 
    linktoc=all,     
    linkcolor=blue,  
}

\title{Formelsamling TFE4120 Elektromagnetisme}

%\renewcommand{\vec}{\vv}
\renewcommand{\vec}[1]{\mathbf{#1}}
\newcommand{\E}{\ensuremath{\vec{E}}}
\newcommand{\e}{\ensuremath{\epsilon_0}}
\newcommand{\p}{\ensuremath{\vec{\underline{p}}}}
\newcommand{\D}{\ensuremath{\vec{D}}}
\renewcommand{\j}{\ensuremath{\vec{j}}}
\newcommand{\B}{\ensuremath{\vec{B}}}
\renewcommand{\H}{\ensuremath{\vec{H}}}
%\renewcommand{\nabla}{\text{aruran}}

\begin{document}

\author{Viljar Johan Femoen} % yes i stole the format from henrik

\flushbottom
\maketitle
\thispagestyle{empty}
\vskip 20pt
\noindent {\Large{\textbf{Disclaimer}}}\vskip 2pt
\noindent This document contains some of the formulas and results from the course, with some motivation where I felt like it. 
This is not intended as learning material, bur rather as a collection of formulas with the required assumptions specified. 
Error reporting to \textit{viljar@timini.no} is much appreciated. 
A vector quantity without an arrow is the modulus of the corresponding vector, if not otherwise specified. 
I might also have changed the vector notation from arrows to bold letters, without changing this disclaimer. 
The lectures and my notes are both in english, and as such this document will also be in english.

\tableofcontents

\section{Electrostatics}
\subsection{Electrostatic fields}
    The coloumb force experienced by a particle with charge $q'$ from a charge $q$ at a distance $r$ from each other is given by 
    \begin{equation}
        \vec{F} = \frac{qq'}{4\pi\e r^2}\vec{u_r},
    \end{equation}
    where \e is the vacuum permitivity constant, and $\vec{u_r}$ is a unit vector pointing from $q$ to $q'$. 
    In SI units, $\e \approx \SI{8.854e-12}{\farad \per \metre}$.
    We define the electrical field experienced by a charge $q'$ resulting from the charge $q$ at a distance $r$ as follows:
    \begin{equation}
        \E = \frac{\vec{F}}{q'} = \frac{q}{4\pi\e r^2}\vec{u_r}.
    \end{equation}
    From the definition of $\vec{u_r}$'s direction, note that, for positive $q$, 
    \E{} points \textit{away} from the source, and opposite for negative $q$.
    
    For multiple charges, the superposition principle applies, i.e. 
    \begin{equation}
        \E = \sum_i\vec{E_i} = \frac{1}{4\pi\e}\sum_i\frac{q_i}{{r_i}^2\vec{u_{ri}}}.
    \end{equation}
    This can be generalised for continuous charges: 
    \begin{equation}
        \E = \frac{1}{4\pi\e}\int_\mathcal{L}\,\frac{\lambda}{r^2}\,\vec{u_r}dl,
    \end{equation}
    where $\lambda$ is a linear charge distribution along the line $\mathcal{L}$. Similarly for surface- and volume-distributions: 
    \begin{equation}
        \E = \frac{1}{4\pi\e}\iint_\Sigma\,\frac{\sigma}{r^2}\,\vec{u_r}d\Sigma,
    \end{equation}
    \begin{equation}
        \E = \frac{1}{4\pi\e}\iiint_{\tau}\,\frac{\rho}{r^2}\,\vec{u_r}d\tau,
    \end{equation}
    where $\Sigma$ is a surface with area charge density $\sigma$, and $\tau$ is a volume with charge density $\rho$.
    
\subsection{Potential}
    The potential energy of a charge $q'$ at a distance $r$ from a charge $q$ is given by
    \begin{equation}
        U = \frac{qq'}{4\pi\e r}.
    \end{equation}
    This follows from the radial symmetry of \E, and the fact that $\vec{F} = - \nabla U$.
    
    
    We define the potential V of a charge $q'$ at a distance $r$ from a charge $q$ as follows: 
    \begin{equation}
        V = \frac{U}{q'} = \frac{q}{4\pi\e r}.
    \end{equation}
    Note that $\nabla V = -\E$, meaning $\E$ is conservative. Therefore, 
    \begin{equation}
        U = \int_\mathcal{L}\,\E\cdot d\vec{l} = q'[V(B) - V(A)] = U(B) - U(A)
    \end{equation}
    for any line $\mathcal{L}$ connecting point $A$ to $B$.

\subsection{Electric dipoles}
    An electric dipole consists of two charges of equal magnitude $q$ but opposite sign, separated by a distance $a$. 
    We define the polarisation vector as follows: 
    \begin{equation}
        \vec{p} = q\vec{a},
    \end{equation}
    where $\vec{a}$ is the vector from the negative  to the positive charge. 
    
    In most cases, the distances $r_1$ and $r_2$ to each of the charges are much larger than $a$. 
    Therefore, for a point $P$, the potential is given as 
    \begin{equation}
        V(P) = \frac{q}{4\pi\e}\left( \frac{1}{r_1}-\frac{1}{r_2}\right) \approx 
        \frac{q}{4\pi\e}\left(\frac{a\cos\theta}{r^2}\right),
    \end{equation}
    where $r \approx r_1 \approx r_2$ is the distance from $P$ to the dipole, and $\theta$ is the angle between $\vec{p}$ and $u_r$ 
    (the unit vector pointing from the middle of the dipole towards $P$). 
    If we define a coordinate system such that $\vec{p}$ lies on the $z$-axis, 
    with the centre of the dipole at the origin, then this can be further simplified to
    \begin{equation}
        \frac{\vec{p}\cdot \vec{u_r}}{4\pi\e r^2} = \frac{pz}{4\pi\e r^2},
    \end{equation}
    where the first expression holds in general, $p = |\vec{p}|$, and $z$ is the $z$-component of $P$.
    
    The electric field produced by the dipole is then found by $\E = -\nabla V$:
    \begin{equation}
        \E(x, y, z) = \frac{3p}{4\pi\e r^3}\left(\frac{zx}{r^2}\,\vec{u_x} + 
        \frac{zy}{r^2}\,\vec{u_y} + \left( -\frac{1}{3} +\cos^2\theta\right)\,\vec{u_z}\right).
    \end{equation}
    Spherical coordinates are prettier: 
    \begin{equation}
        \E(r, \phi, \theta) = \frac{p}{4\pi\e r^2}\left(\frac{2\cos\theta}{r}\,\vec{u_r} + \sin\theta \, \vec{u_\theta}\right),
    \end{equation}
    where $\theta$ is the angle between $\vec{p}$ and $\vec{u_r}$ (the polar angle), 
    and $\vec{u_\theta}$ is orthogonal to $\vec{u_r}$ in the direction of increasing $\theta$ (at a constant azimuthal angle $\phi$). 
    Note the independence of $\phi$ due to radial symmetry along the $z$-axis. 
    
    A dipole $\vec{p}$ placed in an external uniform electric field $\E$ experiences a torque
    \begin{equation}
        \vec{\tau} = \vec{p} \times \E.
    \end{equation}
    More general, the potential energy $U$ of a dipole $\vec{p}$ is given by
    \begin{equation}
        U = -\vec{p}\cdot \E
    \end{equation}
    for any external field $\E$. This also gives the relation 
    \begin{equation}
        \vec{F} = \nabla\left(\vec{p}\cdot\E\right).
    \end{equation}

\subsection{Gauss' law}
    \begin{equation}
        \oiint_\Sigma \E\cdot\vec{u_n}d\Sigma = \frac{q}{\e}
    \end{equation}
    holds for any closed surface $\Sigma$, for an arbitrary $\E$, 
    where $q$ is the total charge confined inside $\Sigma$ and $\vec{u_n}$ is the unit normal of $\Sigma$. Equivalenty, 
    \begin{equation}
        \nabla \cdot \E = \frac{\rho}{\e}
    \end{equation}
    where $\rho$ is a volume charge density. 

\subsection{Common applications of Gauss' law}
    \subsubsection{Infinite plane}
        Take an infinite plane with surface charge density $\sigma$, 
        and a closed cylinder surface $\Sigma$ with the circles equidistant and parallell to the plane. 
        By symmetry, the electric field is orthogonal to the plane. Therefore, $\E\cdot\vec{u_n} \neq 0$ only at the circle surfaces. 
        \begin{equation*}
            \oiint_\Sigma \E\cdot\vec{u_n}d\Sigma = 2\iint_\beta \E\cdot\vec{u_n}d\Sigma,
        \end{equation*}
        where $\beta$ is the circle surfaces of the cylinder. Thus, the integral evaluates to $2E\beta$ where $E = |\E|$. 
        Since the charge encompassed by $\Sigma$ is $\beta\sigma$, we get
        \begin{equation}
            \E = \frac{\sigma}{2\e}\,\vec{u_n}
        \end{equation}

    \subsubsection{Sphere}
        Take a point $P$ at a distance $r$ from the center of a sphere with radius $R$ and volume charge density $\rho$. 
        By symmetry, $\E$ points radially outwards from the sphere. Let $E_r$ denote the radial component of $\E$. Then, 
        \begin{equation*}
            \oiint_\Sigma \E\cdot\vec{u_n}d\Sigma = E_r4\pi\e r^2.
        \end{equation*}
        Furthermore, if $\rho$ is constant, then
        \begin{equation*}
            q = \iiint_\tau \rho d\tau = \rho4\pi R'^3,
        \end{equation*}
        Where $R' = \text{min}(r, R)$. Combining the two yields
        \begin{equation}
            \E = \frac{\rho \text{min}(r, R)^3}{3r^2\e}\,\vec{u_r}.
        \end{equation}
        For $r < R$, this simplifies to
        \begin{equation}
            \E = \frac{\rho r}{3\e}\,\vec{u_r}.
        \end{equation}

\subsection{Dielectrics}    
    An external electric field $\E_0$ induces an opposing field $\E_p$ in dielectric materials. 
    The resulting field inside the dielectric is $\E_{\epsilon_r}$. We define the \textit{relative dielectric constant} as follows: 
    \begin{equation}
        \epsilon_r = \frac{E_0}{E_{\epsilon_r}}.
    \end{equation}
    $\epsilon_r \geq 1$,  $\epsilon_r = 1$ in vacuum, and $\epsilon_r \approx 1$ in air.
    Sometimes, the \textit{electric susceptibility} is used instead:
    \begin{equation}
        \chi_e = \epsilon_r -1.
    \end{equation}
    Another notation used is 
    \begin{equation}
        \epsilon = \e\epsilon_r
    \end{equation}
    where $\epsilon$ is called the \textit{dielectric constant}.
    
    A parallel plate capacitor with surface charge density $\sigma_0$ induces a surface charge density $\sigma_p$ of opposite sign on both sides. 
    \begin{equation}
        \sigma_p = \frac{\e -1}{\epsilon_r}\,\sigma_0.
    \end{equation}
    $\sigma_p$ is called the \textit{polarisation charge density}.
    
    \subsubsection{Polarisation vector}
        The induced field opposing an imposed field in a dielectric is the result of the microscopic structure of the material, 
        where each infinitesimal volume $\Delta \tau$ has an infinitesimal dipole momentum $\vec{\Delta p}$. 
        The total dipole momentum of the dielectric is then 
        \begin{equation}
            \vec{p} = N\left< \vec{p} \right>
        \end{equation}
        Where $N\Delta \tau$ is the total volume of the dielectric. We define the \textit{polarisation vector} as follows:
        \begin{equation}
            \p = \frac{\vec{\Delta p}}{\Delta \tau}.
        \end{equation}
        It can be shown that 
        \begin{equation}
            \p = \sigma_p\,\vec{u_n} = \e \chi_e \E_{\epsilon_r}
        \end{equation}
        where $\E_{\epsilon_r}$ is the total field in the dielectric material.
        
        Applying the divergence theorem, we get
        \begin{equation}
            \nabla \cdot \p = - \rho_{pol}
        \end{equation}
        where $\rho_{pol}$ is the volume charge distribution of the polarisation charges.

\subsection{Displacement field}
    From Gauss' law, we have 
    \begin{equation*}
        \nabla \cdot E = \frac{\rho_{free} + \rho_{pol}}{\e},
    \end{equation*}
    where we have separated the charge density distributions for the free charges and the polarisation charges. 
    Combining this with the divergence of the polarisation vector, we get 
    \begin{equation*}
        \nabla \cdot \left(\e\E + \p\right) = \rho_{free}.
    \end{equation*}
    We therefore define the \textit{displacement field} as follows:
    \begin{equation}
        \D = \e\E + \p.
    \end{equation}
    The following table compares $\D$ and $\p$:
    \begin{center}
        \begin{tabular}{ c|c } 
        Polarisation & Displacement \\
        \hline
         $\nabla \cdot \p = -\rho_{pol}$ & $\nabla \cdot \D = \rho_{free}$ \\[6px] 
         $\p \cdot \vec{u_n} = \sigma_p$ & $\D \cdot \vec{u_n} = \sigma_o$
        \end{tabular}
    \end{center}

    \subsubsection{Continuity conditions at interfaces}
        It can be shown that:
        \begin{itemize}
            \item The tangential component of the $\E$- field is conserved over an interface
            \item The normal component of the $\D$-field is conserved over an interface
        \end{itemize}

\subsection{Energy of an electric field}
    To charge a capacitor with capacitance $C$ with a charge $q$, a work is required. This is given as 
    \begin{equation}
        W = \frac{q^2}{2C} = \frac{C(\Delta V)^2}{2} = \frac{q\Delta V}{2} = \frac{\e E^2\Sigma h}{2},
    \end{equation}
    where $\Sigma$ is the surface of the plates and $h$ is the separation distance. 
    So long as the electrical field is correctly considered, this equation is valid in dielectrics as well.
    
    Since $\Sigma h$ represent a volume, we define a density of energy: 
    \begin{equation}
        u_e = \frac{1}{2}\e E^2, 
    \end{equation}
    such that 
    \begin{equation}
        U_e = \iiint_\tau u_e d\tau
    \end{equation}
    
\subsection{Conductors}
    \subsubsection{Definitions and basic properties}
        Conductors are materials where charges can move, but the intrinsic net charge is zero. 
        Theoretical perfect conductors (all conductors discussed in this document are perfect unless otherwise specified) 
        have an infinite number of free charges. 
        An external electric field will therefore induce a charge distribution in a conductor, 
        such that there are a net negative charge on one side and a net positive charge on the other, 
        in the direction of $\E$. This is called \textit{electrostatic induction}.
        
        Conductors have zero electric field inside the bulk, 
        since any charges that can move will (almost) instantly move according to the external field, counteracting it. 
        
        When a charge $q$ is imposed on a conductor, the charges will lie on the surface bu the same argument. 
        Similarly, the electric field must be (locally) equal in magnitude (and normal to the surface). 
        This makes the surface of a conductor isopotential, which is also valid for conductors without an imposed charge. 

    \subsubsection{Total electrostatic induction}
        Total or complete electrostatic induction is when all field lines of an electric field pass through a conductor.
        
        An example is a point source inside a spherical conductor shell. 
        By the "rule" (my explanation is fairly hand-wavy, resulting in the quotation marks here. 
        The reader may rest assured that the result is provably true) of zero net electric field inside a conductor, 
        the induced electric field on the inner surface of the sphere shell is equal to the field from the source. 
        By conservation of charge in the conductor, 
        the corresponding induced field on the outside is equal to the field that would have been there if the conductive shell was removed. 
        This is a general property of complete electrostatic induction. 
  
\subsection{Current}
    Current is the amount of charges moving through a conductor during a time period. 
    \subsubsection{Current density vector}
        The amount charge moving through a surface $\Sigma$ during $\Delta t$ is given by
        \begin{equation}
            dq = nq\vec{v}\cdot \vec{u_r}\,\Sigma \,\Delta t
        \end{equation}
        where $n$ is the volume density of charged particles (unit \si{\per\metre\cubed}), and $q$ is the charge of each particle.
        Therefore, the current is given by
        \begin{equation}
            dI = \frac{dq}{\Delta t} = nq\vec{v}\cdot\vec{u_n}\,d\Sigma = \j\cdot\vec{u_n}\,d\Sigma
        \end{equation}
        where the surface $d\Sigma$ goes towards zero, and $\j$ is the current density vector, i.e.
        \begin{equation}
            \j = nq\vec{v}.
        \end{equation}
        
        Integrating over $\Sigma$ yields
        \begin{equation}
            I = \iint_\Sigma \j\cdot\vec{u_n}d\Sigma.
        \end{equation}
        Taking the divergence of $\j$, we get
        \begin{equation}
            \nabla \cdot \j = \frac{d\rho}{dt}
        \end{equation}
        where $\rho$ is the volume charge distribution.
        In stationary conditions ($\j_{in} = \j_{out}$, this is equal to 0. 
    
        In an external field, charges propagate in a conductor with a \textit{drift velocity} of
        \begin{equation}
            \vec{v_d} = \frac{q\tau}{m}\,\E
        \end{equation}
        where $q$ is the charge of the charge, and $\tau$ is the mean time between collisions of the charges in the conductor, 
        and $m$ is the mass of the charge.
        
        Therefore, $\j$ in a conductor in an external electric field is
        \begin{equation}
            \j = nq\vec{v} = \frac{nq^2\tau}{m}\,\E.
        \end{equation}
        We name this constant \textit{conductivity}: 
        \begin{equation}
            \sigma = \frac{nq^2\tau}{m},
        \end{equation}
        and the \textit{resistivity}:
        \begin{equation}
            \rho = \frac{1}{\sigma},
        \end{equation}
        where $n$ is the density of charges, $q$ is the charge of a single charge, 
        $\tau$ is the mean time between the charges colliding in the conductor, and $m$ is the mass of each charge. 
        
        We have the relations 
        \begin{align}
            \j &= \sigma \E \\
            \E &= \rho \j.
        \end{align}

    \subsubsection{Ohm's law}
        It can be shown that 
        \begin{equation}
            \Delta V = \frac{Il\rho}{\Sigma} = IR,
        \end{equation}
    
    \subsubsection{Resistance}
        The resistance of a conductor is given as
        \begin{equation}
            R = \int_\mathcal{L} \frac{\rho}{\Sigma}\,dl,
        \end{equation}
        where $\rho$ is the resistivity, 
        $\Sigma$ is the surface area of each infinitesimal slice of the conductor $\mathcal{L}$ along $dl$. 
        Both $\rho$ and $\Sigma$ can change as a function of $l$, for example by changing the material and the girth of the conductor.
        
        Circuits with multiple resistors behave as follows: 
        \begin{align}
            R_{tot} &= \sum_i R_i\quad \text{(series)} \\
            \frac{1}{R_{tot}} &= \sum_i \frac{1}{R_i} \quad\text{(parallel)}
        \end{align}
        
    \subsubsection{Generators}
        Let $\E^*$ be the electric field responsible for the potential difference between the poles of a voltage source, such that
        \begin{equation}
            \oint_\mathcal{L} \E\cdot \vec{dl} = \int_+^-\E_{circuit}\cdot \vec{dl} + \int_-^+(\E_{circuit} + \E^*)\cdot \vec{dl} = \int_-^+\E^*\cdot \vec{dl} = \emf
        \end{equation}
        where $\emf$ is the \textit{electromotive force} (NOT a force, $\left[\emf\right]$ = \si\volt)
        
        Real voltage sources have an internal resistance:
        \begin{equation}
            r = \frac{1}{I}\int_-^+(\E + \E^*)\cdot\vec{dl}
        \end{equation}
        
    \subsubsection{Kirchoff's laws}
        Both laws presented are direction-dependent, 
        but the choice of direction will not affect the absolute value of the answer 
        (so long as the chosen direction is correctly considered for each component). 
        \begin{equation}
            \sum_i I_i = 0
        \end{equation}
        in a node.
        \begin{equation}
            \sum_i R_iI_i = \sum_i \emf_i
        \end{equation}
        in a mesh, where $\emf$ are voltage sources. 

    \subsection{Capacitors}
    A capacitor is a configuration of two conductors resulting in total electrostatic induction. 
    The most common configuration in practice is a parallel plate configuration, 
    such that their area is significantly larger than the separation distance and we can neglect border effects.
    
    We define capacitance as follows:
    \begin{equation}
        C = \frac{q}{\Delta V}
    \end{equation}
    where $q$ is the total charge of the capacitor and $\delta V$ is the difference in potential between the conductors.
    
    Capacitance is measured in farad [F].
    
    Real capacitors are usually in a parallel plate configuration such that the plates are wound around each other in a spiral, 
    to increase the surface area while keeping the footprint of the component small.
    
    The capacitance of a capacitor with a dielectric material between the conductors is given by
    \begin{equation}
        C = \epsilon_rC_0
    \end{equation}
    where $\epsilon_r$ is the relative dielectric constant and $C_0$ is the theoretical capacitance 
    if the dielectric material had been vacuum. Since $\epsilon_r > 1$, the capacitance increases.
    
    \subsubsection{Parallel plate capacitors}
        For a given geometry of the plates, with a surface charge density $\sigma$ on one plate and $-\sigma$ on the other, 
        the electric field between the plates is given as 
        \begin{equation}
            \E = \frac{\sigma}{\e}\,\vec{u_n},
        \end{equation}
        where $\vec{u_n}$ is the normal vector of the positive plate towards the negative. This equation neglects border effects.
        
        The field outside the plates is zero.
        
        If the plates have an area $\Sigma$ and a separation distance $h << \sqrt{\Sigma}$, then the change in potential is
        \begin{equation}
            \Delta V = \frac{h\sigma}{\e}.
        \end{equation}
        Thus, the capacitance of a parallel plate capacitor is
        \begin{equation}
            C = \frac{q}{|\Delta V|} = \frac{\sigma \Sigma}{\frac{h\sigma}{\e}} = \frac{\e \Sigma}{h}.
        \end{equation}
        Note the independence of $\sigma$, i.e. the capacitance is a geometry-based property.

    \subsubsection{Cylindrical capacitors}
        If we instead consider a cylindrical capacitor, 
        where a cylindrical conductor with radius $R_1$ is placed inside a cylinder shell conductor with inner radius $R_2$, 
        both with height $h$ and constant linear charge density $\lambda$ along the height, then the capacitance is given by 
        \begin{equation}
            C = \frac{2\pi h \e}{\ln{\frac{R_2}{R_1}}}
        \end{equation}
        Naturally, if the gap between the cylinders is small ($R_2 - R_1 \approx 0$), 
        then the cylindrical capacitor is locally approximated by a parallel plate capacitor.

    \subsubsection{Charging and discarging}
        If an ideal (internal resistance $r=0$) voltage source $\emf$ is coupled in series with a resistor $R$ 
        and a capacitor $C$, where the capacitor charge is denoted $q$, then we get the following differential equation: 
        \begin{equation}
            R\frac{dq}{dt} = \emf - \frac{q}{C},
        \end{equation}
        which is solved by
        \begin{equation}
            q(t) = \emf C \left(1-\exp{\frac{-t}{RC}}\right).
        \end{equation}
        The energy stored in the capacitor is 
        \begin{equation}
            W_C = \frac{1}{2}C\emf^2,
        \end{equation}
        whereas the total energy "spent" by the circuit is \textit{twice as much}.

\section{Magnetostatics}
    \subsection{Magnetic field}
        In this course, we call the \B-field the \textit{magnetic field}. \B{} is measured in tesla (\si\tesla).
        
        The magnetic field has no divergence, i.e.
        \begin{equation}
            \nabla \cdot \B = 0.
        \end{equation}
        A charge $q$ moving in a magnetic field with (non-relativistic) velocity $\vec{v}$ experiences a force as follows: 
        \begin{equation}
            \vec{F} = q\vec{v}\times \B.
        \end{equation}
        If the field is uniform (same direction and magnitude everywhere) and the velocity is orthogonal to the field, 
        then the charge will move in a circular path with radius
        \begin{equation}
            r = \frac{mv}{qB}
        \end{equation}
        where $m$ is the mass of the charge, $v$ is the velocity, $q$ is the charge, 
        and $B$ is the magnitude of the uniform magnetic field.
        
        This holds in general for the orthogonal component of the velocity, 
        whereas the parallel component is unaffected by the presence of the magnetic field. 
        Therefore, if the velocity is neither parallel nor orthogonal to the magnetic field, 
        the resulting motion is a spiral with constant velocity, pitch and radius.
        
    \subsection{Mass spectroscopy}
        One can find the ratio of charge per mass for ionised particles by accelerating them with a parallel plate capacitor 
        before introducing a uniform magnetic field orthogonal to the direction of velocity. 
        One can then measure the distance between the entry and exit points.
        
        \begin{equation}
            \frac{q}{m} = \frac{2 \Delta V}{B^2 r^2}
        \end{equation}
        where $q$ and $m$ is the charge and mass of the particle, $B$ is the magnitude of the (uniform) magnetic field, 
        and $r$ is the measured radius of the circular motion (half of the measured distance between the entry and exit points).
        
    \subsection{Force on currents}
        By the definition of \j, we get that $\j \times \B$ represents a force per volume. Therefore, 
        \begin{equation}
            d\vec{F} = \Sigma dl \j \times \B, 
        \end{equation}
        where $\Sigma dl$ is an infinitesimal volume along a conductor, resulting in 
        \begin{equation}
            \vec{F} = \int \Sigma \j \times \B dl = \int I d\vec{l} \times \B.
        \end{equation}
        The differential form of this result is called \textit{the second Laplace elementary law}:
        \begin{equation}
            d\vec{F} = I\vec{dl} \times \B,
        \end{equation}
        where $\vec{F}$ is the force on an infinitesimal length $\vec{dl}$ along a conductor with current $I$ 
        immersed in a magnetic field $\B$.
        
    \subsection{Magnetic dipole moment}
    
        A mesh with a current $I$ and surface area $\Sigma$ immersed in a magnetic field experiences a torque:
        \begin{equation}
            \vec{\tau} = I\,\Sigma\,\vec{u_n}\times\B,
        \end{equation}
        Where $\vec{u_n}$ is found by applying the right-hand-rule to the current circulation. 
        We define the \textit{magnetic dipole moment} as
        \begin{equation}
            \vec{m} = I\Sigma\vec{u_n}.
        \end{equation}
        The magnetic dipole moment is analogous to the electric dipole moment: 
        \begin{align*}
            \vec{\tau} &= \vec{m} \times \B\\
            \vec{\tau} &= \vec{p} \times \E
        \end{align*}
    
    \subsection{Currents as magnetic field sources}
    
        The \textit{first Laplace elementary law} states the following:
        \begin{equation}
            d\B = \frac{\mu_0}{4\pi}\cdot I\frac{d\vec{l}\times\vec{u}_r}{r^2},
        \end{equation}
        relating the magnetic field in a point at a distance $r$ from a current $I$ flowing along $d\vec{l}$.
        $\mu_0 = 1.257\cdot 10^{-7}$\si{\henry\per\metre}(redefined in 2019, no longer \textit{defined} as $4\pi\cdot10^{-7}$ but still very close) is the \textit{magnetic permeability}. 
        The integral of this equation is sometimes called the \textit{Ampere-Laplace law}.

        The first Laplace elementary law can be rewritten in terms of \j{} instead:
        \begin{equation}
            d\B = \frac{\mu_0}{4\pi}\cdot \frac{\j\times\vec{u}_r}{r^2}\,d\tau,
        \end{equation}
        where one now integrates over the volume $d\tau$ rather than the length of the conductor.

        If one considers a point charge with charge $q$ and velocity $\vec{v}$, 
        the following equation describes the magnetic field:
        \begin{equation}
            \B = \frac{\mu_0}{4\pi}\cdot q\frac{\vec{v}\times\vec{u}_r}{r^2}.
        \end{equation}

        This helps us relate the electric and magnetic field resulting from a point source to each other:
        \begin{equation}
            \B = \e \mu_0 \cdot\vec{v}\times\E,
        \end{equation}
        which holds for non-relativistic velocities. Intrestingly,
        \begin{equation}
            \e\mu_0 = \frac{1}{c^2},
        \end{equation}
        where $c = 299792458$ \si{\metre\per\second} is the speed of light in vacuum. 

    \subsection{Common examples of magnetic field sources}
        \subsubsection{Straight linear conductor}
            This relation is sometimes called the \textit{Biot-Savart law}.

            Consider a point beside an infinite, straight conductive wire. Using the first Laplace elementary law, we get 
            \begin{equation}
                \B = \frac{\mu_0 I}{2\pi R}\,\vec{u}_\phi,
            \end{equation}
            where $R$ is the shortest distance to the conductor, 
            and $\vec{u}_\phi$ is the  unit vector orthogonal to both $\vec{u}_r$ and the direction of current.
            The sign is found by the right-hand rule, where the thumb should point along the current flow.
            
        \subsubsection{Closed circular loop}
            For simplicities sake, consider a point \textit{along the axis} of a circular conductive wire with radius $R$, 
            such that the distance along the axis to the point is $x$, 
            and the distance from the point to the actual conductor is $r$ 
            (this is constant regardless of where on the conductor this is measured, due to the symmetry).
            The conductor has a current flowing through it, such that the positive $x$-direction follows the right-hand rule. 

            The magnetic field is then given by
            \begin{equation}
                \B = \frac{\mu_0 I R^2}{2 \left(x^2 + R^2\right)^\frac{3}{2}}\,\vec{u}_x.
            \end{equation}
            Note that the field in the centre of the conductive loop is
            \begin{equation}
                \B = \frac{\mu_0 I}{2R}.
            \end{equation}
        
        \subsubsection{Linear solenoid}
            A solenoid is a stack of equidistant circular loops of conductive wire, each having the same dimensions and current. 
            This configuration is impossible to achieve physically, but is closely approximated by a coil. 

            Consider a point inside a solenoid with linear loop density $n$, radius $R$, and current $I$. 
            The resulting magnetic field \textit{inside} the solenoid is given by
            \begin{equation}
                d\B = \frac{\mu_0 I n}{2}\sin\theta d\theta,
            \end{equation}
            where $\theta$ is the angle between the axis of the solenoid and the vector to each point on the solenoid along its axis.
            Performing the integration yields
            \begin{equation}
                \B = \frac{\mu_0 In}{2}\left(\cos\theta_0 - \cos\theta_1\right),
            \end{equation}
            where the angles represent the angles from the axis to the first and last point of the solenoid, respectively. 

            If the solenoid is infinite, we get
            \begin{equation}
                \B = \mu_0 In,
            \end{equation}
            which is usually used as an approximation inside finite solenoids as well.
    
    \subsection{Ampere's law}
        Ampere's law describes how the sum of currents enclosed in a loop relates to the magnetic field: 
        \begin{equation}
            \oint_\mathcal{L}\B\cdot d\vec{l} = \mu_0 I_{enclosed},
        \end{equation}
        or, in differential form: 
        \begin{equation}
            \nabla\times\B = \mu_0\j
        \end{equation}
    
    \subsection{Magnetic interactions between circuits}
        \subsubsection{Force}
            It can be shown that the force acting upon a circuit $C_2$ from circuit $C_1$ is given by
            \begin{equation}
                \vec{F}_{2,1} = \frac{\mu_0}{4\pi}\cdot I_1 I_2 \cdot G_{2,1},
            \end{equation}
            where $I_i$ is the current in circuit $i$, and where 
            \begin{equation}
                G_{2,1} = \oint_{C_2} \oint_{C_1} \frac{1}{r^2}\,d\vec{l_2}\times\left(d\vec{l_1} \times \vec{u}_r\right).
            \end{equation}
            $G_{2,1}$ is a coefficient that only dependent on the geometry of the circuits. 
            This coefficient is not named in the lectures, and neither the shorthand notation of $G$.
            $r$ is the pointwise distance between the circuits. By Newton's third law of motion, 
            $F_{2,1} = -F_{1,2}$, and as such $G_{2,1} = - G_{1,2}$.

            A simple example would be two parallel infinite wires with currents $I_1$ and $I_2$, 
            separated by a distance $r$. The linear force density is then
            \begin{equation}
                f = \frac{\mu_0 I_1 I_2}{2\pi r}.
            \end{equation}
            Note that $f$ is not a force, but a linear force density, an as such is measured in \si{\newton\per\metre}. 
            The total force between infinite wires would be infinite, and therefore uninteresting.

        \subsubsection{Magnetic flux}
            The flux of \B{} through a circuit $C_2$ where circuit $C_1$ is the source of \B{} is given by
            \begin{equation}
                \Phi_{2,1} = I_1 \cdot M_{2,1},
            \end{equation}
            where $M_{2,1}$ is the \textit{coefficient of mutual induction}, and is a geometric property given by
            \begin{equation}
                M_{2,1} = \iint_{\Sigma_2} \left(\oint_{C_1}\frac{\mu_0}{4\pi r^2}\,d\vec{l_1}\times \vec{u}_r \right)\cdot \vec{u}_n d\Sigma_2.
            \end{equation}
            where $\Sigma_i$ is the surface enclosed by $C_i$, and $\vec{S}_i$ is the path along $C_i$.
            (I'm not sure if $\frac{\mu_0}{4\pi}$ should be included in $M$, as it is a constant and not relevant for the geometry.)

            It can be shown that 
            \begin{equation}
                M_{2,1} = M_{1,2}.
            \end{equation}

    \subsection{Theory behind magnetic materials}
        By waving our hands, we realise that atoms behave as tiny current loops, and they therefore have magnetic dipole momentum.
        However, 
        \begin{equation*}
            \left<\vec{m}\right> = 0
        \end{equation*}
        in most materials, since each $\vec{m}$ point in a random direction. 

        \subsubsection{Magnetisation vector}
            We define the \textit{magnetisation vector} as follows:
            \begin{equation}
                \vec{M} = n\left<\vec{m}\right>
            \end{equation}
            where $n$ is the volume density of current loops, and $\vec{m}$ is the magnetic dipole momentum of the loops. 
            Note the similarity to the polarisation vector $\p$.

            If many $\vec{m}$ are oriented in the same direction on a disc, 
            there will appear to be a current $I_mag$ along the perimiter of the disc. 
            This is not an actual current, since there are no charges moving along the perimiter.
            Taking an infinitesimally thick disc, we have 
            \begin{equation}
                \j_{s, m} = \vec{M}\times\vec{u}_n,
            \end{equation}
            where $\j_{s, m}$ is the current density on the \textbf{s}urface 
            (perimiter, the disc has no surface area on its edge due to beind infinitely thin) 
            resulting from the \textbf{m}agnets, and $\vec{u}_n$ is the unit vector orthogonal to the perimiter of the disc. 

            Generalising the disc to a cylinder with length $l$, we have the relation
            \begin{equation}
                I_{mag} = Ml = j_{s, m}l,
            \end{equation}
            a result of Ampere's law for magnetic materials given a uniform $\vec{M}$.

        \subsubsection{Ampere's law for magnetic materials}
            \begin{equation}
                \oint_\mathcal{L}\vec{M}\cdot d\vec{l} = I_{mag},
            \end{equation}
            where $\mathcal{L}$ is an arbitrary closed path encompassing $I_{mag}$ magnetisation current. 
            Rewriting to a differential form yields
            \begin{equation}
                \nabla \times \vec{M} = \j_m,
            \end{equation}
            where $\j_m$ is a surface density of magnetisation current. 
            If $\vec{M}$ is uniform (and therefore irrotational), $\j_m = 0$. 
            This does NOT imply $\j_{s, m} = 0$, since the latter is a linear density whereas $\j_m$ is a suface density.

            The following table compares magnetisation to polatisation.
            \begin{center}
                \begin{tabular}{ c|c } 
                Magnetic & Dielectric \\
                \hline
                 $\vec{M}\times\vec{u}_n = \j_{s, m}$ & $\p \cdot \vec{u_n} = \sigma_p$ \\[6px] 
                 $\nabla \times \vec{M} = \j_m,$ & $\nabla \cdot \p = -\rho_p$
                \end{tabular}
            \end{center}

    \subsection{Magnetising field}
        By a similar argument as with \D, combining the general Ampere's law with the one for magnetic materials, we get
        \begin{equation*}
            \oint_\mathcal{L}\left(\frac{\B}{\mu_0}-\vec{M}\right)d\vec{l} = I_{encl},
        \end{equation*}
        and as such we define the \textit{magnetising field} as 
        \begin{equation}
            \H = \frac{\B}{\mu_0}-\vec{M}.
        \end{equation}
        Note that the name of \H{} can vary in litterature; some call it the magnetic field 
        (the name we gave \B), and some call it the magnetic field intensity or amplitude 
        (names which might be interpreted as scalar fields).

        \begin{equation}
            \oint_\mathcal{L}\H = I_{encl},
        \end{equation}
        and 
        \begin{equation}
            \nabla\times\H = \j.
        \end{equation}
        
        \subsubsection{Continuity conditions}
            It can be shown that:
            \begin{itemize}
                \item The normal component of the \B-field is conserved over an interface
                \item The tangential component of the \H-field is conserved over an interface
            \end{itemize}

    \subsection{Real magnetic materials}
        \subsubsection{Paramagnetic materials}
            Examples: Al, Pt, W, Ti.

            $\left<m\right>=0$ if $\B_{\text{ext}} = 0$. Else, 
            \begin{equation}
                \vec{M} = \chi_m\H,
            \end{equation}
            where $\chi_m$ is the magnetic suceptability of the material. 
            It usually ranges around $10^{-6}$ to $10^{-3}$, and is always positive (since $\vec{M} || \H$).

            Curie's law states that 
            \begin{equation}
                \chi_m = \frac{C}{T}, 
            \end{equation}
            where $C$ is a material-dependent constant. Note that this makes $\vec{M}$ larger for lower temperatures.

        \subsubsection{Diamagnetic materials}
            Examples: $\text{H}_2$, $\text{N}_2$, Na, Cu, Hg. The common trait is an unpaired electron in the molecular orbitals.
            
            These materials do NOT have a microscopic dipole momentum $\vec{m}$. Still, since electrons are moving, 
            an external magnetic field will affect the material similarly to paramagnetic materials:
            \begin{equation}
                \vec{M} = \chi_m\H,
            \end{equation}
            where the magnetic suceptability $\chi_m$ ranges from around $10^{-9}$ to $10^{-5}$. 
            For diamagnetic materials, $\chi_m < 0$ and $\vec{M}$ is antiparallel to \H. 
            This effect is present in ALL materials (as all materials have moving electrons), but is usually negligable.
        
        For both para- and diamagnetic materials, we can write
        \begin{equation}
            \B = \mu_0 \mu_r \H,
        \end{equation}
        where $\mu_r = 1+\chi_m$ is the relative magnetic permeability of the material.
        
        \subsubsection{Ferromagnetic materials}
            In ferromagnetic materials, each grain in the crystal has aligned $\vec{m}$ separately. 
            In an external \B-field, more and more grains align. 
            Unlike para- and diamagnetic materials, the relationship between $\vec{M}$ and \H{} is NOT linear.

            If the applied \H{} is increased, $\vec{M}$ will eventually reach \textit{saturation magnetisation}. 
            If \H{} is then decreased to zero, there will still be a \textit{residual magnetisation} $\vec{M} > 0$. 
            Similarly and equivalently for continuing the reduction of \H{} into high negative values, 
            a saturation magnetisation of equal magnitude as before will be reached.

        
\section{Time-variant fields}
    \subsection{Magnetic field}
        A concuctive mesh with resistance $R$ will, in the presence of an external magnetic field \B, 
        have an induced current $I_i$ such that
        \begin{equation}
            \emf_\text{i} = RI_\text{i} = -\frac{\partial \Phi(\B)}{\partial t},
        \end{equation}
        where $\emf_\text{i}$ is the induced electromotive force, and $\Phi(\B)$ is the flux of \B{} through the mesh. 
        Any change in the flux will result in an induced current, 
        be it by rotating or stretching the mesh (changing the projected area),
        moving the mesh or source, or change the field intensity (if the source has such properties).

        Since an $\emf$ must be accompanied by an electric field, 
        there will be an induced electric field $\E_\text{i}$ such that
        \begin{equation}
            \oint_\mathcal{L}\E_\text{i}\cdot d\vec{l} = -\frac{\partial \Phi(\B)}{\partial t},
        \end{equation}
        valid for \textit{any} line $\mathcal{L}$, regardless of path or medium. 
        Therefore, this field is present in vacuum and requires no conductors.

        Note that $\E_\text{i}$ is NOT conservative in general. The differential form of the equation is
        \begin{equation}
            \nabla \times \E_\text{i} = -\frac{\partial \Phi(\B)}{\partial t}.
        \end{equation}
        Note that this also holds in general, since $\nabla\times\E = 0$ for an electrostatic field \E.
    \subsection{Examples of magnetically induced current}
        \subsubsection{Rod on a rail}
            Two conductive parallel rails, seperated by a distance $b$ and connected with a resistor $R$, stretch infinitley long, 
            and is immersed in a static and uniform \B-field (i.e. constant in time and space). 
            A conductive rod with resistance $r$ is placed between the rods at a distance $x_0$ from the resistor, 
            connecting the rods and forming a mesh. The rod is then given a velocity $v$ along the rails, away from the resistor.
            \B{} points parallel to the normal vector of the mesh. We assume zero friction. 
            The flux of the magnetic field through the mesh is given by
            \begin{equation}
                \Phi(\B)(t) = Bb(vt+x_0) \rightarrow -\frac{\partial \Phi(\B)}{\partial t} = -Bbv.
            \end{equation}
            Therefore, the induced current in the mesh is
            \begin{equation}
                I_\text{i} = -\frac{Bbv}{R+r}.
            \end{equation}
            Since all the values are positive, the minus sign dictates the direction of the current. Applying the right hand rule,
            the thumb should therefore point antiparallel to \B{}. 
            (More rigorously, the flux was calculated with the assumption of a surface normal parallel to \B. 
            The minus sign therefore flips the direction of current.)

            This induced current creates an opposing force (opposite direction of $v$): 
            \begin{equation}
                \vec{F}_\text{i} = -\frac{B^2b^2}{R+r}\vec{v}.
            \end{equation}
            It is clear that, if no external force is applied, the rod will stop moving.

        \subsubsection{Rotating mesh}
            A rectangular mesh (resistance $R$) with area $\Sigma$ rotates along an axis along its surface with angular velocity $\omega$. 
            It is immersed in a static, uniform \B-field (i.e. constant in time and space).
            We get 
            \begin{align}
                \Phi(\B)(t) &= B\Sigma\cos (\omega t), \\
                \emf_\text{i} &= \omega\Sigma B \sin (\omega t).
            \end{align}
            The instantaneous and average dissapated power is 
            \begin{align}
                P_\text{i} &= \frac{\omega^2 B^2 \Sigma^2}{R}\sin^2(\omega t),\\
                \left<P_\text{i}\right> &= \frac{1}{T}\int_0^T P_\text{i}dt = \frac{\omega^2 B^2 \Sigma^2}{2R},
            \end{align}
            where $T = \frac{2\pi}{\omega}$ is the period of a turn.
            Since power is dissapated, a torque is required to keep the mesh spinning: 
            \begin{equation}
                \vec{\tau}_\text{ext} = \vec{m}\times\B = \frac{\omega B^2 \Sigma^2}{R}\sin^2(\omega t).
            \end{equation}
        
        
\end{document}