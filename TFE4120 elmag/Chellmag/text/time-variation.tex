\subsection{Time-varying magnetic field}
    A concuctive mesh with resistance $R$ will, in the presence of an external magnetic field \B, 
    have an induced current $I_i$ such that
    \begin{equation}
        \emf_\text{i} = RI_\text{i} = -\frac{\partial \Phi(\B)}{\partial t},
    \end{equation}
    where $\emf_\text{i}$ is the induced electromotive force, and $\Phi(\B)$ is the flux of \B{} through the mesh. 
    Any change in the flux will result in an induced current, 
    be it by rotating or stretching the mesh (changing the projected area),
    moving the mesh or source, or change the field intensity (if the source has such properties).

    Since an $\emf$ must be accompanied by an electric field, 
    there will be an induced electric field $\E_\text{i}$ such that
    \begin{equation}
        \oint_\mathcal{L}\E_\text{i}\cdot d\vec{l} = -\frac{\partial \Phi(\B)}{\partial t},
    \end{equation}
    valid for \textit{any} line $\mathcal{L}$, regardless of path or medium. 
    Therefore, this field is present in vacuum and requires no conductors.

    Note that $\E_\text{i}$ is NOT conservative in general. The differential form of the equation is
    \begin{equation}
        \nabla \times \E_\text{i} = -\frac{\partial \B}{\partial t}.
    \end{equation}
    Note that this also holds in general, since $\nabla\times\E = 0$ for an electrostatic field \E.

\subsection{Time-varying electric field} \label{Time-varying electric field}
    A changing electrical field induces a \textit{displacement current} such that
    \begin{equation}
        \nabla \cdot \j_s = -\nabla\left(\e\frac{\partial\E}{\partial t}\right),
    \end{equation}
    where $\j_s$ is the \textit{displacement current density}. This is not a real current in the sense that charges are moving, 
    but it behaves as a current density.

    $\j_{\text{tot}} = \j_{\text{conduction}} + \j_s$ is always solenoidal. 

    Note that 
    \begin{equation*}
        I_s = \Phi(\j_s) = \iint_S \e\frac{\partial\E}{\partial t}\cdot d\vec{S} = \e\frac{\partial}{\partial t}\Phi_S(\E)
    \end{equation*}

    Inserting this into Ampere's law, we get 
    \begin{align}
        \oint_{\partial S}\B\cdot d\vec{l} &= \mu_0 I_{\text{enclosed}} + \mu_0\e\frac{\partial}{\partial t}\Phi_S\left(\E\right)\\
        \nabla\times\B &= \mu_0\j_{\text{concuctive}} + \mu_0\e\frac{\partial\E}{\partial t},
    \end{align}
    where we often omit the subscript on $\j$, and $\partial S$ is the perimiter of the surface $S$.

\subsection{Examples of magnetically induced current}
    \subsubsection{Rod on a rail}
        Two conductive parallel rails, seperated by a distance $b$ and connected with a resistor $R$, stretch infinitley long, 
        and is immersed in a static and uniform \B-field (i.e. constant in time and space). 
        A conductive rod with resistance $r$ is placed between the rods at a distance $x_0$ from the resistor, 
        connecting the rods and forming a mesh. The rod is then given a velocity $v$ along the rails, away from the resistor.
        \B{} points parallel to the normal vector of the mesh. We assume zero friction. 
        The flux of the magnetic field through the mesh is given by
        \begin{equation}
            \Phi(\B)(t) = Bb(vt+x_0) \rightarrow -\frac{\partial \Phi(\B)}{\partial t} = -Bbv.
        \end{equation}
        Therefore, the induced current in the mesh is
        \begin{equation}
            I_\text{i} = -\frac{Bbv}{R+r}.
        \end{equation}
        Since all the values are positive, the minus sign dictates the direction of the current. Applying the right hand rule,
        the thumb should therefore point antiparallel to \B. 
        (More rigorously, the flux was calculated with the assumption of a surface normal parallel to \B. 
        The minus sign therefore flips the direction of current.)

        This induced current creates an opposing force (opposite direction of $v$): 
        \begin{equation}
            \vec{F}_\text{i} = -\frac{B^2b^2}{R+r}\vec{v}.
        \end{equation}
        It is clear that, if no external force is applied, the rod will stop moving.

    \subsubsection{Rotating mesh}
        A rectangular mesh (resistance $R$) with area $S$ rotates along an axis along its surface with angular velocity $\omega$. 
        It is immersed in a static, uniform \B-field (i.e. constant in time and space).
        We get 
        \begin{align}
            \Phi(\B)(t) &= BS\cos (\omega t), \\
            \emf_\text{i} &= \omega S B \sin (\omega t).
        \end{align}
        The instantaneous and average dissapated power is 
        \begin{align}
            P_\text{i} &= \frac{\omega^2 B^2 S^2}{R}\sin^2(\omega t),\\
            \left<P_\text{i}\right> &= \frac{1}{T}\int_0^T P_\text{i}dt = \frac{\omega^2 B^2 S^2}{2R},
        \end{align}
        where $T = \frac{2\pi}{\omega}$ is the period of a turn.
        Since power is dissapated, a torque is required to keep the mesh spinning: 
        \begin{equation}
            \vec{\tau}_\text{ext} = \vec{m}\times\B = \frac{\omega B^2 S^2}{R}\sin^2(\omega t).
        \end{equation}
    
    \subsubsection{Self-induction}
        A circuit with current induces a magnetic field. 
        If the current varies, then the flux of the induced magnetic field through the circuit also varies. 
        We get
        \begin{equation}
            \frac{\partial \Phi(\B)}{\partial t} = L_{1,1}\frac{\partial I}{\partial t} = -\emf_{\text{i}},
        \end{equation}
        where $L_{1,1}$ is the \textit{coefficient of self-inductance}. (see \ref{Magnetic flux} for mutual inductance)
\subsection{Transient circuits}
    \subsubsection{RL-circuits}
        The simplest RL-circuits consist of a source $\emf$, a resistor $R$, and a coil (solenoid) $L$.
        Here, we add a switch as well.

        Applying Kirchoff's first law, we arrive at the differential equation
        \begin{equation}
            \emf - RI = L_{1,1}\frac{\partial I}{\partial t},
        \end{equation}
        where $A$ is an integration constant and is determined by the initial conditions.

        With initial conditions $t=0,\, I=0$, we get 
        \begin{equation}
            I(t) = \frac{\emf}{R}\left(1-e^{\frac{-L}{R}t}\right),
        \end{equation}
        i.e. closing the switch at $t=0$ results in an inverse exponential increase in current 
        with an upper bound of $\frac{\emf}{R}$, i.e. conductor-like behaviour for $L$.

        Consider initial conditions $t=0\, I = \frac{\emf}{R}$, i.e. we open the switch at $t=0$.
        We now consider the switch as a resistor with $R' >> R$, since a spark will still allow current for a while.
        \begin{equation}
            I(t) = \frac{\emf}{R'}\left(1-e^{\frac{-L}{R+R'}t}\right) + \frac{\emf}{R} e^{\frac{-L}{R+R'}t},
        \end{equation}
        which is closely approximated (since $R+R' \approx R'$ and $R'^{-1} \approx 0$) by
        \begin{equation}
            I(t) \approx \frac{\emf}{R}e^{\frac{-L}{R'}t}.
        \end{equation}
        The resulting behaviour of the coil is then to keep the current flowing for a while, but the decay is exponential.

    \subsubsection{RLC-circuits}
        A system with a resistor with resistance $R$, a coil with coefficient of self induction $L$, and a capacitor with capacitance $C$, 
        all in series, is described by the following differential equation: 
        \begin{equation} \label{Current in RLC}
            \frac{\partial^2}{\partial t^2}I + \frac{R}{L}\frac{\partial}{\partial t}I + \frac{1}{LC}I = 0.
        \end{equation}
        We define the \textit{damping coefficient} $\gamma$ and the \textit{eigen frequency} $\omega_0$ as follows:
        \begin{align*}
            \gamma &= \frac{R}{2L}\\
            \omega_0 &= \sqrt{\frac{1}{LC}},
        \end{align*}
        yielding in the boring exponential solutions for $\gamma^2 \geq {\omega_0}^2$:
        \begin{equation}
            I(t) = e^{-\gamma t}\left(Ae^{\sqrt{\gamma^2 - {\omega_0}^2}t}+Be^{-\sqrt{\gamma^2 - {\omega_0}^2}t}\right),
        \end{equation}
        where $A$ and $B$ are integration coefficients. The system is more interesting for $\gamma^2 \leq {\omega_0}^2$,
        where the complex exponential yields oscilating behaviour:
        \begin{equation}
            I(t) = e^{-\gamma t}\cdot A\sin\left(\omega t + \phi\right),
        \end{equation}
        where $A$ and $\phi$ are integration constants, and $\omega = \sqrt{\gamma^2 - {\omega_0}^2} > 0$.

        With $R=0$, there will be no energy dissapated over the resistor, and we get 
        \begin{equation*}
            E = E_c + E_L = \frac{1}{2}C{V_c(t)}^2 + \frac{1}{2}L{I(t)}^2,
        \end{equation*}
        where the current $I$ and the potential difference between the capacitor plates $V_c$ are given by 
        \begin{align*}
            I(t) &= A\sin\left(\frac{1}{\sqrt{LC}}t + \phi\right) \\
            V_c (t) &= \sqrt{\frac{L}{C}} \, A\cos\left(\frac{1}{\sqrt{LC}}t + \phi\right)
        \end{align*}

    \subsubsection{Permanent oscilations in RLC-circuits}

        Adding an AC source $\emf(t) = \emf_0\cos\left(\omega t + \phi\right)$ to our RLC-circuit, where all components are in parallel, 
        we get equation \ref{Current in RLC} equal to $\frac{-\omega\emf_0}{L}\sin\left(\omega t + \phi\right)$.
        
        We wave our hands, and get the following as a solution: 
        \begin{equation}
            I(t) = I_0 \cos\left(\omega t\right),
        \end{equation}
        where 
        \begin{align*}
            \phi &= \text{atan} \left(\frac{\omega L - \left(\omega C\right)^{-1}}{R}\right)\\
            I_0 &= \frac{\emf_0}{\sqrt{R^2 + \left(\omega L - \left(\omega C\right)^{-1}\right)^2}},
        \end{align*}
        where $\phi$ is the phase of the AC source.

        Plotting $I_0(\omega)$ (usually) yields a bell-curve. We define $\Delta\omega$ as the width of the curve at $I_0 = \frac{\emf_0}{\sqrt{2}R}$: 
        \begin{align*}
            \omega_1 &= \frac{-R}{2L} + \sqrt{\frac{R^2}{4L^2} + {\omega_0}^2}\\
            \omega_2 &= \frac{R}{2L} + \sqrt{\frac{R^2}{4L^2} + {\omega_0}^2}\\
            \Delta\omega &= \frac{R}{L},
        \end{align*}
        where $\omega_0 = \frac{1}{\sqrt{LC}}$.

        Note that $\omega_1\omega_2={\omega_0}^2$, and $|\omega_1 - \omega_0| \neq |\omega_2 - \omega_0|$, i.e. the curve is asymmetric.

        $\Delta\omega$ is used for the \textit{quality factor}:
        \begin{equation}
            Q = \frac{\omega_0}{\Delta\omega}.
        \end{equation}

        This is useful for sensors where the measurement is produced by a change in capacitance; 
        if $Q$ is large then we get a narrow range of frequencies with substantial current, and a changed capacitance 
        (resulting in a changed $\omega_0$) is easy to notice. 

        A small $Q$, on the other hand, will result in a wide band of "accepted" frequencies, which can be useful in other circumstances.
    
        Which size ranges are considered "small" and "large" was not discussed during lectures. 

\subsection{Electromagnetic waves}
    \subsubsection{Magnetic vector potential}
        Since \B{} is solenoidal, it can be written as the curl of another vector field. 
        
        Let $\vec{A}$ be a vector field, such that
        \begin{equation}
            \nabla \times \vec{A} = \B.
        \end{equation}
        $\vec{A}$ is not a physical field, but it is a helpful mathematical tool. 

        Manipulating Maxwell's equations, we arrive at 
        \begin{equation}
            \E = -\nabla V - \frac{\partial \vec{A}}{\partial t}.
        \end{equation}
        
        $\vec{A}$ can be changed by adding the gradient of any arbitrary scalar function, 
        however, to keep the measurable fields \B{} and \E{} unchanged, one also needs to change $V$ correspondingly:
        \begin{align*}
            \vec{A}' &= \vec{A} + \nabla S \\
            V' &= V - \frac{\partial S}{\partial t}
        \end{align*}
        where $S$ is a scalar function with both a spatial- and a time-parameter.
        Applying this mathematical feature is called a \textit{Gauge transform}.
    \subsubsection{Arriving at the wave equation}
        Inserting $V$ and $\vec{A}$ into Maxwell's equations, we get the following: 
        \begin{align*}
            \nabla^2V - \e\mu_0\frac{\partial^2V}{\partial t^2} &= \frac{-\rho}{\e}\\
            \nabla^2\vec{A} - \e\mu_0\frac{\partial^2\vec{A}}{\partial t^2} &= \mu_0\j,
        \end{align*}
        which is clearly the wave equation if there are no sources present ($\rho = \j = 0)$, 
        where the velocity of propagation is $c$ (since $\e\mu_0 = c^{-2}$).

        We use the shorthand notation of the \textit{d'Alembertian}: 
        \begin{equation*}
            \dalambert = \nabla^2 - \frac{1}{c^2}\frac{\partial^2}{\partial t^2}
        \end{equation*}

        and arrive at
        \begin{center}
            \begin{tabular}{ c|c }
                $\dalambert V = 0$&$ \dalambert\,\E = 0$\\
                $\dalambert\,\vec{A} = 0 $&$\dalambert\,\B = 0$
            \end{tabular}
        \end{center}
    \subsubsection{Solution to the wave equation}
        The solution  can be written as
        \begin{equation*}
            f(\vec{r}, t) = A\sin\left(\vec{k}\cdot\vec{r} - \omega t\right),
        \end{equation*}
        where $A$ is the amplitude, $|\vec{k}| = \frac{2\pi}{\lambda}$ is the wave "number", pointing in the direction of travel, 
        and $\omega = 2\pi\nu$ is the angular frequency. $\lambda$ is the wavelength and $\nu$ is the frequency, such that
        $\lambda\nu = c$.

        Both \E, \B{} and $\vec{k}$ are orthogonal, i.e. electromagnetic waves are transverse. Also note that 
        \begin{equation}
            E = cB
        \end{equation}

    \subsubsection{Energy of an electromagnetic wave}
        Summing the electric and magnetic components yields the following energy density: 
        \begin{equation}
            u_{\text{em}} = \e E^2 = \frac{1}{\mu_0}\, B^2.
        \end{equation}
    
    \subsubsection{Poynting vector}
        The Poynting vector describes the power of an electromagnetic wave per area:
        \begin{equation}
            \vec{S} = \frac{1}{\mu_0}\,\E\times\B.
        \end{equation}
        $\vec{S}$ is measured in \si{\joule\per\second\metre\squared}.
        The average length of $\vec{S}$ can be found by multiple methods:
        \begin{equation}
            \left<|\vec{S}|\right> = \frac{1}{T}\int_T Sdt = \frac{1}{2}\e c\E^2 = \frac{1}{2\mu_0}c\B^2
        \end{equation}
    
\subsection{Polarisation of electromagnetic waves}
    There are two types of polarisation discussed in the lectures: linear and eliptical.
    By convention, the polarisation refers to the electric field, 
    but the magnetic field, being orthogonal, will behave similarly.
    
    Most non-electrical light-sources produce \textit{unpolarised light}, which, contrary to what the name suggests, 
    contains both linearly and eliptically polarised light in somewhat equal proportions. 

    \subsubsection{Linearly polarised electromagnetic waves}
        Both fields of the electromagnetic waves only oscilate in one direction,
        i.e. if the wave propagates in the $z$-direction, 
        then the phase difference between the $x$- and $y$-component of each field is zero.
        Mathematically, the wave can be written as
        \begin{equation}
            \E(z, t) = E_{0x}\sin\left(kz - \omega t + \phi\right)\hat{x} +
            E_{0y}\sin\left(kz - \omega t + \phi\right)\hat{y},
        \end{equation}
        where $E_{0x,y}$ are the decomposed amplitude of the wave, 
        $k$ is the wave vector (one-dimensional in this case), 
        $\omega$ is the angular frequency, 
        and $\phi$ is an arbitrary initial phaseshift which is usually set to 0.

    \subsubsection{Eliptically polarised electromagnetic waves}
        Eliptically polarised waves propagating in the $z$-direction can be mathematically described by
        \begin{equation}
            \E(z, t) = E_{0x}\sin\left(kz - \omega t + \phi_x \right)\hat{x} +
            E_{0y}\sin\left(kz - \omega t + \phi_y \right)\hat{y},
        \end{equation}
        where $E_{0x,y}$ are the decomposed amplitude of the wave, 
        $k$ is the wave vector (one-dimensional in this case), 
        $\omega$ is the angular frequency, 
        and $\phi$ is an arbitrary initial phaseshift. 
        In this course, only $\phi_x = \frac{\pi}{2}$ and $\phi_y = 0$ is discussed.
        
        The phase difference of the two components yield a helical behavior, 
        as compared to the linear behaviour of the linearly polarised counterpart.

        Note that, if $E_{0x} = E_{0y}$, (and $\phi_x = \phi_y + \frac{\pi}{2}$, which is true for this course)
            then the wave is circularly polarised.

        Applying the right hand rule, 
        one can determine if the polarised wave is a right-hand or left-hand circularly polarised.

\subsection{The electromagnetic spectrum}
    The following table shows some basic info for the different classifications of electromagnetic waves.
    \begin{longtable}[c]{@{}c|c@{}} 
        \caption{The electromagnetic spectrum}    
        \endfirsthead
        \hline
        \multirow{2}{*}{$\displaystyle\lambda < \SI{0.1}{\angstrom}$}
         & Gamma radiation. \\
         &"You become Hulk" - Emiliano Descorvi.\\
        \hline    
        \multirow{3}{*}{$\displaystyle\lambda < \SI{10}{\angstrom}$}
         & X-rays. \\
         & Two types: soft and hard, hard being the more energetic. \\
         & Usually the result of the deexitisation of a radioactive nucleide. \\
        \hline
        \multirow{3}{*}{$\displaystyle\lambda < \SI{400}{\nano\metre}$}
         & Ultraviolet radiation. \\
         & Three types: A, B, and C, C being the more energetic. \\
         & Usually the result of the deexitisation of an electron.\\
        \hline
        \multirow{2}{*}{$\displaystyle\lambda < \SI{700}{\nano\metre}$}
         & Visible light. The wavelength limit is approximate, \\
         & as the visible spectrum varies for different people.\\
        \hline
        \multirow{3}{*}{$\displaystyle\lambda < \SI{1}{\milli\metre}$}
         & Infrared radiation. \\
         & Three types: near, mid, and far, near being the more energetic. \\
         & Usually caused by molecular vibrations.\\
        \hline
        \multirow{2}{*}{$\displaystyle\lambda < \SI{100}{\milli\metre}$}
         & Microwaves.\\
         & Not actually micro-scale.\\
        \hline
        \multirow{2}{*}{else}
         & Radiowaves. \\
         & These can have an arbitrarily long wavelength.\\
        \hline
    \end{longtable}

\subsection{Retarded potentials}
    Retarded potentials are the $\vec{A}$- and $V$-fields generated by a time-varying current (or charge distrubution) in the past, 
    since the fields propagate with the speed of light and not instantly. 
    
    By applying the \textit{Lorenz Gauge} $\nabla\cdot\vec{A} + \e\mu_0\frac{\partial V}{\partial t} = 0$, 
    Maxwell's equations for potentials become
    \begin{align}
        \dalambert V &= \frac{\rho\left(\vec{r}, t\right)}{\e} \\
        \dalambert \vec{A} &= \mu_0 \j\left(\vec{r}, t\right),
    \end{align}
    which are solved by 
    \begin{align} 
        V(\vec{r}, t) &= \frac{1}{4\pi\e} \iiint_\tau \frac{\rho\left(\vec{r'}, t'\right)}{|\vec{r} - \vec{r'}|}d\tau \\
        \vec{A}(\vec{r}, t) &= \frac{\mu_0}{4\pi} \iiint_\tau \frac{\j\left(\vec{r'}, t'\right)}{|\vec{r} - \vec{r'}|}d\tau,
        \label{retarded potentials}
    \end{align}
    where $\tau$ is the volume where $\rho$ and \j{} are defined, 
    and we integrate over all points $\vec{r'}$ in $\tau$.
    $t' = t - \frac{|\vec{r} - \vec{r'}|}{c}$ is the \textit{retarded time}. 
    The notion of retarded time is to take the propagation speed into account.

\subsection{Dipoles as electromagnetic wave sources}
    In this chapter, an example of retarded potentials is discussed. 

    If a charge $-q_0$ is placed at the origin of a refrence frame, 
    and a charge $q$ moves such that $\vec{r}(t) = z_0 \sin\omega t \hat{z}$, 
    then this is an oscilating dipole with dipole momentum 
    $\vec{p}\left(t\right) = q_0 z_0 \sin\left(\omega t\right)\hat{z} = p_0 \sin\left(\omega t\right)\hat{z}$.

    This configutarion is implactical, but a conductor with an oscilating current behaves similarly, since
    \begin{equation*}
        I(t) = \frac{\partial q}{\partial t} = \frac{1}{z_0} \frac{\partial p}{\partial t} = \frac{p_0}{z_0}\omega\cos\left(\omega t\right).
    \end{equation*}
    
    Dimensional analysis of equation \ref{retarded potentials}, along with the simplification $\vec{r} - \vec{r}' \approx \vec{r}$, 
    somehow leads to 
    \begin{equation}
        \vec{A}\left(\vec{r}, t\right) = \frac{\mu_0}{4\pi r} \frac{\partial }{\partial t}\vec{p}\left(t'\right),
    \end{equation} 
    where $t' = t - \frac{{r}}{c}$ is a simplified retarded time ($r = |\vec{r}|$).

    The potential is found by applying the Lorenz Gauge again:
    \begin{equation}
        V(\vec{r}, t) =
        \left(\frac{1}{cr}\frac{\partial}{\partial t} + \frac{1}{r^2} \right)
        \frac{p\left(t'\right)}{4\pi\e},
    \end{equation}
    where $p = |\vec{p}|$ and $r = |\vec{r}|$.

    We can now find \B{} and \E, since $\B = \nabla\times\vec{A}$ and $\E = -\nabla V - \frac{\partial\vec{A}}{\partial t}$:
    \begin{align}
        E_r &= \frac{2\cos\theta}{4\pi\e r}\left[\frac{1}{cr}\frac{\partial}{\partial t} + \frac{1}{r^2}\right]
                \eval{p\left(t\right)}_{t=t'}\\
        E_\phi &= 0 \\
        E_\theta &= \frac{\sin\theta}{4\pi\e r}\left[\e\mu_0\frac{\partial^2}{\partial t^2} 
                + \frac{1}{cr}\frac{\partial}{\partial t} + \frac{1}{r^2}\right]
                \eval{p\left(t\right)}_{t=t'}\\
        B_r &= 0 \\
        B_\phi &= \frac{\mu_0 \sin\theta}{4\pi r}\left[\e\mu_0\frac{\partial^2}{\partial t^2} + \frac{1}{r^2}\right]
            \eval{p\left(t\right)}_{t=t'}\\
        B_\theta &= 0,
    \end{align}
    where we still have $p = |\vec{p}|$, $r = |\vec{r}|$, and $t' = t - \frac{{r}}{c}$. The expressions in brackets are operators, working in $p(t)$.
