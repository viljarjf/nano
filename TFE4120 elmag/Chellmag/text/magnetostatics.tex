\subsection{Magnetic field}
    In this course, we call the \B-field the \textit{magnetic field}. \B{} is measured in tesla (\si\tesla).
    
    The magnetic field has no divergence, i.e.
    \begin{equation}
        \nabla \cdot \B = 0.
    \end{equation}
    A charge $q$ moving in a magnetic field with (non-relativistic) velocity $\vec{v}$ experiences a force as follows: 
    \begin{equation}
        \vec{F} = q\vec{v}\times \B.
    \end{equation}
    If the field is uniform (same direction and magnitude everywhere) and the velocity is orthogonal to the field, 
    then the charge will move in a circular path with radius
    \begin{equation}
        r = \frac{mv}{qB}
    \end{equation}
    where $m$ is the mass of the charge, $v$ is the velocity, $q$ is the charge, 
    and $B$ is the magnitude of the uniform magnetic field.
    
    This holds in general for the orthogonal component of the velocity, 
    whereas the parallel component is unaffected by the presence of the magnetic field. 
    Therefore, if the velocity is neither parallel nor orthogonal to the magnetic field, 
    the resulting motion is a spiral with constant velocity, pitch and radius.
    
\subsection{Mass spectroscopy}
    One can find the ratio of charge per mass for ionised particles by accelerating them with a parallel plate capacitor 
    before introducing a uniform magnetic field orthogonal to the direction of velocity. 
    One can then measure the distance between the entry and exit points.
    
    \begin{equation}
        \frac{q}{m} = \frac{2 \Delta V}{B^2 r^2}
    \end{equation}
    where $q$ and $m$ is the charge and mass of the particle, $B$ is the magnitude of the (uniform) magnetic field, 
    and $r$ is the measured radius of the circular motion (half of the measured distance between the entry and exit points).
    
\subsection{Force on currents}
    By the definition of \j, we get that $\j \times \B$ represents a force per volume. Therefore, 
    \begin{equation}
        d\vec{F} = S dl \j \times \B, 
    \end{equation}
    where $S dl$ is an infinitesimal volume along a conductor $\mathcal{L}$, resulting in 
    \begin{equation}
        \vec{F} = \int_\mathcal{L} S \j \times \B dl = \int_\mathcal{L} I d\vec{l} \times \B.
    \end{equation}
    The differential form of this result is called \textit{the second Laplace elementary law}:
    \begin{equation}
        d\vec{F} = I\vec{dl} \times \B,
    \end{equation}
    where $\vec{F}$ is the force on an infinitesimal length $\vec{dl}$ along a conductor with current $I$ 
    immersed in a magnetic field $\B$.
    
\subsection{Magnetic dipole moment}

    A mesh with a current $I$ and surface area $S$ immersed in a magnetic field experiences a torque:
    \begin{equation}
        \vec{\tau} = I\,S\,\hat{n}\times\B,
    \end{equation}
    Where $\hat{n}$ is the normal vector of $S$. Its direction is found by applying the right-hand-rule to the current circulation. 
    We define the \textit{magnetic dipole moment} as
    \begin{equation}
        \vec{m} = IS\hat{n}.
    \end{equation}
    The magnetic dipole moment is analogous to the electric dipole moment: 
    \begin{align*}
        \vec{\tau} &= \vec{m} \times \B\\
        \vec{\tau} &= \vec{p} \times \E
    \end{align*}

\subsection{Currents as magnetic field sources}

    The \textit{first Laplace elementary law} states the following:
    \begin{equation}
        d\B = \frac{\mu_0}{4\pi}\cdot I\frac{d\vec{l}\times\hat{r}}{r^2},
    \end{equation}
    relating the magnetic field in a point at a distance $r$ from a current $I$ flowing along $d\vec{l}$.
    $\mu_0 = 1.257\cdot 10^{-6}$\si{\henry\per\metre}(redefined in 2019, 
    no longer \textit{defined} as $4\pi\cdot10^{-7}$ but still very close) is the \textit{magnetic permeability}. 
    The integral of this equation is sometimes called the \textit{Ampere-Laplace law}.

    The first Laplace elementary law can be rewritten in terms of \j{} instead:
    \begin{equation}
        d\B = \frac{\mu_0}{4\pi}\cdot \frac{\j\times\hat{r}}{r^2}\,d\tau,
    \end{equation}
    where one now integrates over the volume $d\tau$ rather than the length of the conductor.

    If one considers a point charge with charge $q$ and velocity $\vec{v}$, 
    the following equation describes the magnetic field:
    \begin{equation}
        \B = \frac{\mu_0}{4\pi}\cdot q\frac{\vec{v}\times\hat{r}}{r^2}.
    \end{equation}

    This helps us relate the electric and magnetic field resulting from a point source to each other:
    \begin{equation}
        \B = \e \mu_0 \cdot\vec{v}\times\E,
    \end{equation}
    which holds for non-relativistic velocities. Intrestingly,
    \begin{equation}
        \e\mu_0 = \frac{1}{c^2},
    \end{equation}
    where $c = 299792458$ \si{\metre\per\second} is the speed of light in vacuum. 

\subsection{Common examples of magnetic field sources}
    \subsubsection{Straight linear conductor}
        This relation is sometimes called the \textit{Biot-Savart law}.

        Consider a point beside an infinite, straight conductive wire. Using the first Laplace elementary law, we get 
        \begin{equation}
            \B = \frac{\mu_0 I}{2\pi R}\,\hat{\phi},
        \end{equation}
        where $R$ is the shortest distance to the conductor, 
        and $\hat{\phi}$ is the  unit vector orthogonal to both $\hat{r}$ and the direction of current.
        The sign is found by the right-hand rule, where the thumb should point along the current flow.
        
    \subsubsection{Closed circular loop}
        For simplicities sake, consider a point \textit{along the axis} of a circular conductive wire with radius $R$, 
        such that the distance along the axis to the point is $x$, 
        and the distance from the point to the actual conductor is $r$ 
        (this is constant regardless of where on the conductor this is measured, due to the symmetry).
        The conductor has a current flowing through it, such that the positive $x$-direction follows the right-hand rule. 

        The magnetic field is then given by
        \begin{equation}
            \B = \frac{\mu_0 I R^2}{2 \left(x^2 + R^2\right)^\frac{3}{2}}\,\hat{x}.
        \end{equation}
        Note that the field in the centre of the conductive loop is
        \begin{equation}
            \B = \frac{\mu_0 I}{2R}.
        \end{equation}
    
    \subsubsection{Linear solenoid}
        A solenoid is a stack of equidistant circular loops of conductive wire, each having the same dimensions and current. 
        This configuration is impossible to achieve physically, but is closely approximated by a coil. 

        Consider a point inside a solenoid with linear loop density $n$, radius $R$, and current $I$. 
        The resulting magnetic field \textit{inside} the solenoid is given by
        \begin{equation}
            d\B = \frac{\mu_0 I n}{2}\sin\theta d\theta,
        \end{equation}
        where $\theta$ is the angle between the axis of the solenoid and the vector to each point on the solenoid along its axis.
        Performing the integration yields
        \begin{equation}
            \B = \frac{\mu_0 In}{2}\left(\cos\theta_0 - \cos\theta_1\right),
        \end{equation}
        where the angles represent the angles from the axis to the first and last point of the solenoid, respectively. 

        If the solenoid is infinite, we get
        \begin{equation}
            \B = \mu_0 In,
        \end{equation}
        which is usually used as an approximation inside finite solenoids as well.

\subsection{Ampere's law}
    Ampere's law describes how the sum of currents enclosed in a loop relates to the magnetic field: 
    \begin{equation}
        \oint_\mathcal{L}\B\cdot d\vec{l} = \mu_0 I_{\text{enclosed}},
    \end{equation}
    or, in differential form: 
    \begin{equation}
        \nabla\times\B = \mu_0\j.
    \end{equation}
    This holds for stationary conditions, non-stationary conditions is discussed in chapter \ref{Time-varying electric field}. 

\subsection{Magnetic flux} \label{Magnetic flux}
    The flux of \B{} through a circuit $C_2$ where circuit $C_1$ is the source of \B{} is given by
    \begin{equation}
        \Phi_{2,1} = I_1 \cdot L_{2,1},
    \end{equation}
    where $L_{2,1}$ is the \textit{coefficient of mutual induction}, and is a geometric property given by
    \begin{equation}
        L_{2,1} = \iint_{S_2} \left(\oint_{C_1}\frac{\mu_0}{4\pi r^2}\,d\vec{l_1}\times \hat{r} \right)\cdot \hat{n} dS_2.
    \end{equation}
    where $S_i$ is the surface enclosed by $C_i$, and $\vec{l}_i$ is the path along $C_i$.

    It can be shown that 
    \begin{equation}
        L_{2,1} = L_{1,2}.
    \end{equation}.

\subsection{Magnetic energy}
    The \textit{magnetic energy volume density} is given by
    \begin{equation}
        u_m = \frac{1}{2\mu}\B^2,
    \end{equation}
    valid point by point, where $\mu = \mu_0 \mu_r $. 
    For linear materials (meaning \H{} is linear in \B, this does NOT apply to ferromagnetic materials), we get another way to express this: 
    \begin{equation}
        u_m = \frac{1}{2}\B\cdot\H = \frac{1}{2}\mu\H^2,
    \end{equation}
    since $\B = \mu\H{}$. Integrating these over a volume yields the total magnetic energy inside the volume.
    Another way to express the energy is by the induction coefficient (see \ref{Magnetic flux}):
    \begin{equation}
        U_m = \frac{1}{2}L_{1,1}I^2,
    \end{equation}
    where $L$ is the coefficient of self induction, and $I$ is the current flowing in the circuit.

    For ferromagnetic materials, the energy can be found by
    \begin{equation}
        U_m = \oint \H \cdot d\B.
    \end{equation}
    The process of changing \B{} dissapates this energy as heat.

    For a two-circuit system, the energy has contributions from mutual inductance as well: 
    \begin{equation}
        U_m = \sum_{i,j}\frac{1}{2}L_{i,j}I_iI_j = \frac{1}{2}L_1{I_1}^2 + \frac{1}{2}L_2{I_2}^2 + MI_1I_2,
    \end{equation}
    where, for simplicity, $L_1 = L_{1,1},\, L_2 = L_{2,2}$, and $M = L_{1,2}$.

\subsection{Magnetic force between circuits}
    It can be shown that the force acting upon a circuit $C_2$ from circuit $C_1$ is given by
    \begin{equation}
        \vec{F}_{2,1} = \frac{\mu_0}{4\pi}\cdot I_1 I_2 \cdot G_{2,1},
    \end{equation}
    where $I_i$ is the current in circuit $i$, and where 
    \begin{equation}
        G_{2,1} = \oint_{C_2} \oint_{C_1} \frac{1}{r^2}\,d\vec{l_2}\times\left(d\vec{l_1} \times \hat{r}\right).
    \end{equation}
    $G_{2,1}$ is a coefficient that only dependent on the geometry of the circuits. 
    (This coefficient is not named in the lectures, and neither the shorthand notation of $G$, so it will not be found in the notes).
    $r$ is the pointwise distance between the circuits. By Newton's third law of motion, 
    $F_{2,1} = -F_{1,2}$, and as such $G_{2,1} = - G_{1,2}$.

    A geometrically simple example would be two parallel infinite wires with currents $I_1$ and $I_2$, 
    separated by a distance $r$. The linear force density is then
    \begin{equation}
        f = \frac{\mu_0 I_1 I_2}{2\pi r}.
    \end{equation}
    Note that $f$ is not a force, but a linear force density, an as such is measured in \si{\newton\per\metre}. 
    The total force between infinite wires would be infinite, and therefore uninteresting.

\subsection{Theory behind magnetic materials}
    By waving our hands, we realise that atoms behave as tiny current loops, and they therefore have magnetic dipole momentum.
    However, 
    \begin{equation*}
        \left<\vec{m}\right> = 0
    \end{equation*}
    in most materials, since the direction is uniformly distributed. 

    \subsubsection{Magnetisation vector}
        We define the \textit{magnetisation vector} as follows:
        \begin{equation}
            \vec{M} = n\left<\vec{m}\right>
        \end{equation}
        where $n$ is the volume density of current loops, and $\vec{m}$ is the magnetic dipole momentum of the loops. 
        Note the similarity to the polarisation vector $\p$.

        If many $\vec{m}$ are oriented in the same direction on a disc, 
        there will appear to be a current $I_{\text{mag}}$ along the perimiter of the disc. 
        This is not an actual current, since there are no charges moving along the perimiter.
        Taking an infinitesimally thin disc, we have 
        \begin{equation}
            \j_{s, m} = \vec{M}\times\hat{n},
        \end{equation}
        where $\j_{s, m}$ is the current density on the \textbf{s}urface 
        (which is the perimiter, the disc has no surface area on its edge due to beind infinitely thin) 
        resulting from the \textbf{m}agnets, and $\hat{n}$ is the unit normal vector along the perimiter of the disc. 

        Generalising the disc to a cylinder with length $l$, we have the relation
        \begin{equation}
            I_{\text{mag}} = \int \j_{s, m}\cdot d\vec{l}.
        \end{equation}

    \subsubsection{Ampere's law for magnetic materials}
        \begin{equation}
            \oint_\mathcal{L}\vec{M}\cdot d\vec{l} = I_{\text{mag}},
        \end{equation}
        where $\mathcal{L}$ is an arbitrary closed path encompassing $I_{\text{mag}}$ magnetisation current. 
        Rewriting to a differential form yields
        \begin{equation}
            \nabla \times \vec{M} = \j_m,
        \end{equation}
        where $\j_m$ is a surface density of magnetisation current. 
        If $\vec{M}$ is uniform (and therefore irrotational), $\j_m = 0$. 
        This does NOT imply $\j_{s, m} = 0$, since the latter is a linear density whereas $\j_m$ is a suface density.

        The following table compares magnetisation to polatisation.
        \begin{center}
            \begin{tabular}{ c|c } 
            Magnetic & Dielectric \\
            \hline
                $\vec{M}\times\hat{n} = \j_{s, m}$ & $\p \cdot \hat{n} = \sigma_p$ \\
                $\nabla \times \vec{M} = \j_m,$ & $\nabla \cdot \p = -\rho_p$
            \end{tabular}
        \end{center}

\subsection{Magnetising field}
    By a similar argument as with \D, combining the general Ampere's law with the one for magnetic materials, we get
    \begin{equation*}
        \oint_\mathcal{L}\left(\frac{\B}{\mu_0}-\vec{M}\right)d\vec{l} = I_{\text{encl}},
    \end{equation*}
    and as such we define the \textit{magnetising field} as 
    \begin{equation}
        \H = \frac{\B}{\mu_0}-\vec{M}.
    \end{equation}
    Note that the name of \H{} can vary in litterature; some call it the magnetic field 
    (the name we gave \B), and some call it the magnetic field intensity or amplitude 
    (names which might be interpreted as scalar fields).

    \begin{equation}
        \oint_\mathcal{L}\H = I_{\text{encl}},
    \end{equation}
    and 
    \begin{equation}
        \nabla\times\H = \j.
    \end{equation}
    
    \subsubsection{Continuity conditions}
        It can be shown that:
        \begin{itemize}
            \item The normal component of the \B-field is conserved over an interface
            \item The tangential component of the \H-field is conserved over an interface
        \end{itemize}

\subsection{Real magnetic materials}
    \subsubsection{Paramagnetic materials}
        Examples: Al, Pt, W, Ti.

        $\left<m\right>=0$ if $\B_{\text{ext}} = 0$. Else, 
        \begin{equation}
            \vec{M} = \chi_m\H,
        \end{equation}
        where $\chi_m$ is the magnetic suceptability of the material. 
        It usually ranges around $10^{-6}$ to $10^{-3}$, and is always positive (since $\vec{M} || \H$).

        Curie's law states that 
        \begin{equation}
            \chi_m = \frac{C}{T}, 
        \end{equation}
        where $C$ is a material-dependent constant. Note that this makes $\vec{M}$ larger for lower temperatures.

    \subsubsection{Diamagnetic materials}
        Examples: $\text{H}_2$, $\text{N}_2$, Na, Cu, Hg. The common trait is an unpaired electron in the molecular orbitals.
        
        These materials do NOT have a microscopic dipole momentum $\vec{m}$. Still, since electrons are moving, 
        an external magnetic field will affect the material similarly to paramagnetic materials:
        \begin{equation}
            \vec{M} = \chi_m\H,
        \end{equation}
        where the magnetic suceptability $\chi_m$ ranges from around $10^{-9}$ to $10^{-5}$. 
        For diamagnetic materials, $\chi_m < 0$ and $\vec{M}$ is antiparallel to \H. 
        This effect is present in ALL materials (as all materials have moving electrons), but is usually negligable.
    
    For both para- and diamagnetic materials, we can write
    \begin{equation}
        \B = \mu_0 \mu_r \H,
    \end{equation}
    where $\mu_r = 1+\chi_m$ is the relative magnetic permeability of the material.
    
    \subsubsection{Ferromagnetic materials}
        In ferromagnetic materials, each grain in the crystal has aligned $\vec{m}$ separately. 
        In the presence an external \B-field, more and more grains align. 
        Unlike para- and diamagnetic materials, the relationship between $\vec{M}$ and \H{} is NOT linear.

        If the applied \H{} is increased, $\vec{M}$ will eventually reach \textit{saturation magnetisation}. 
        If \H{} is then decreased to zero, there will still be a \textit{residual magnetisation} $\vec{M} > 0$. 
        Similarly and equivalently for continuing the reduction of \H{} into high negative values, 
        a saturation magnetisation of equal magnitude as before (but opposite direction, since $\vec{M}$ aligns with \H) will be reached.
