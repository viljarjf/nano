\subsection{Electrostatic fields}
    The coloumb force experienced by a particle with charge $q'$ from a charge $q$ at a distance $r$ from each other is given by 
    \begin{equation}
        \vec{F} = \frac{qq'}{4\pi\e r^2}\hat{r},
    \end{equation}
    where \e is the vacuum permitivity constant, and $\hat{r}$ is a unit vector pointing from $q$ to $q'$. 
    In SI units, $\e \approx \SI{8.854e-12}{\farad \per \metre}$.
    We define the electrical field experienced by a charge $q'$ resulting from the charge $q$ at a distance $r$ as follows:
    \begin{equation}
        \E = \frac{\vec{F}}{q'} = \frac{q}{4\pi\e r^2}\hat{r}.
    \end{equation}
    From the definition of $\hat{r}$'s direction, note that, for positive $q$, 
    \E{} points \textit{away} from the source, and opposite for negative $q$.
    
    For multiple charges, the superposition principle applies, i.e. 
    \begin{equation}
        \E = \sum_i\vec{E_i} = \frac{1}{4\pi\e}\sum_i\frac{q_i}{{r_i}^2}\hat{r}_i.
    \end{equation}
    This can be generalised for continuous charges: 
    \begin{equation}
        \E = \frac{1}{4\pi\e}\int_\mathcal{L}\,\frac{\lambda}{r^2}\,\hat{r}dl,
    \end{equation}
    where $\lambda$ is a linear charge distribution along the line $\mathcal{L}$. Similarly for surface- and volume-distributions: 
    \begin{equation}
        \label{surface density integral}
        \E = \frac{1}{4\pi\e}\iint_S\,\frac{\sigma}{r^2}\,\hat{r}dS,
    \end{equation}
    \begin{equation}
        \E = \frac{1}{4\pi\e}\iiint_{\tau}\,\frac{\rho}{r^2}\,\hat{r}d\tau,
    \end{equation}
    where $S$ is a surface with area charge density $\sigma$, and $\tau$ is a volume with charge density $\rho$.
    
\subsection{Potential}
    The potential energy of a charge $q'$ at a distance $r$ from a charge $q$ is given by
    \begin{equation}
        U = \frac{qq'}{4\pi\e r}.
    \end{equation}
    This follows from the radial symmetry of \E, and the fact that $\vec{F} = - \nabla U$.
    
    
    We define the potential V of a charge $q'$ at a distance $r$ from a charge $q$ as follows: 
    \begin{equation}
        V = \frac{U}{q'} = \frac{q}{4\pi\e r}.
    \end{equation}
    Note that $\nabla V = -\E$, meaning $\E$ is conservative. Therefore, 
    \begin{equation}
        U = \int_\mathcal{L}\,\E\cdot d\vec{l} = q'[V(B) - V(A)] = U(B) - U(A)
    \end{equation}
    for any line $\mathcal{L}$ connecting point $A$ to $B$.

\subsection{Electric dipoles}
    An electric dipole consists of two charges of equal magnitude $q$ but opposite sign, separated by a distance $a$. 
    We define the polarisation vector as follows: 
    \begin{equation}
        \vec{p} = q\vec{a},
    \end{equation}
    where $\vec{a}$ is the vector from the negative  to the positive charge. 
    
    In most cases, the distances $r_1$ and $r_2$ from an arbitrary point $P$ to each of the charges are much larger than $a$. 
    Therefore, the potential is given as 
    \begin{equation}
        V(P) = \frac{q}{4\pi\e}\left( \frac{1}{r_1}-\frac{1}{r_2}\right) \approx 
        \frac{q}{4\pi\e}\left(\frac{a\cos\theta}{r^2}\right),
    \end{equation}
    where $r \approx r_1 \approx r_2$ is the distance from $P$ to the dipole, and $\theta$ is the angle between $\vec{p}$ and $u_r$ 
    (the unit vector pointing from the middle of the dipole towards $P$). 
    If we define a coordinate system such that $\vec{p}$ lies on the $z$-axis, 
    with the centre of the dipole at the origin, then this can be further simplified to
    \begin{equation}
        \frac{\vec{p}\cdot \hat{r}}{4\pi\e r^2} = \frac{pz}{4\pi\e r^2},
    \end{equation}
    where the first expression holds in general, $p = |\vec{p}|$, and $z$ is the $z$-component of $P$.
    
    The electric field produced by the dipole is then found by $\E = -\nabla V$:
    \begin{equation}
        \E(x, y, z) = \frac{3p}{4\pi\e r^3}\left(\frac{zx}{r^2}\,\hat{x} + 
        \frac{zy}{r^2}\,\hat{y} + \left( -\frac{1}{3} +\cos^2\theta\right)\,\hat{z}\right).
    \end{equation}
    Spherical coordinates are prettier: 
    \begin{equation}
        \E(r, \phi, \theta) = \frac{p}{4\pi\e r^2}\left(\frac{2\cos\theta}{r}\,\hat{r} + \sin\theta \, \hat{\theta}\right),
    \end{equation}
    where $\theta$ is the angle between $\vec{p}$ and $\hat{r}$ (the polar angle), 
    and $\hat{\theta}$ is orthogonal to $\hat{r}$ in the direction of increasing $\theta$ (at a constant azimuthal angle $\phi$). 
    Note the independence of $\phi$ due to radial symmetry along the $z$-axis. 
    Also note how $\theta$ and $\phi$ are switched compared to the notation used in TMA4105. 
    This is done here to stay consistent with the notation in this course.
    
    A dipole $\vec{p}$ placed in an external uniform electric field $\E$ experiences a torque
    \begin{equation}
        \vec{\tau} = \vec{p} \times \E.
    \end{equation}
    More general, the potential energy $U$ of a dipole $\vec{p}$ is given by
    \begin{equation}
        U = -\vec{p}\cdot \E
    \end{equation}
    for any external field $\E$. This also gives the relation 
    \begin{equation}
        \vec{F} = \nabla\left(\vec{p}\cdot\E\right).
    \end{equation}

\subsection{Gauss' law}
    \begin{equation}
        \oiint_S \E\cdot d\vec{S} = \frac{q}{\e}
    \end{equation}
    holds for any closed surface $S$, for an arbitrary $\E$, 
    where $q$ is the total charge confined inside $S$ and $d\vec{S} = \hat{n}dS$ where $\hat{n}$ is the unit normal of $S$. Equivalenty, 
    \begin{equation}
        \nabla \cdot \E = \frac{\rho}{\e}
    \end{equation}
    where $\rho$ is a volume charge density. 

\subsection{Common applications of Gauss' law}
    \subsubsection{Infinite plane}
        Take an infinite plane with surface charge density $\sigma$, 
        and a closed cylinder surface $S$ with the circles equidistant and parallell to the plane. 
        By symmetry, the electric field is orthogonal to the plane. Therefore, $\E\cdot\hat{n} \neq 0$ only at the circle surfaces. 
        \begin{equation*}
            \oiint_S \E\cdot d\vec{S} = 2\iint_\beta \E\cdot d\vec{S},
        \end{equation*}
        where $\beta$ is the circle surfaces of the cylinder. Thus, the integral evaluates to $2E\beta$ where $E = |\E|$. 
        Since the charge encompassed by $S$ is $\beta\sigma$, we get
        \begin{equation}
            \E = \frac{\sigma}{2\e}\,\hat{n}
        \end{equation}

    \subsubsection{Sphere}
        Take a point $P$ at a distance $r$ from the center of a sphere with radius $R$ and volume charge density $\rho$. 
        By symmetry, $\E$ points radially outwards from the sphere. Let $E_r$ denote the radial component of $\E$. Then, 
        \begin{equation*}
            \oiint_S \E\cdot d\vec{S} = E_r4\pi\e r^2.
        \end{equation*}
        Furthermore, if $\rho$ is constant, then
        \begin{equation*}
            q = \iiint_\tau \rho d\tau = \rho4\pi R'^3,
        \end{equation*}
        Where $R' = \text{min}(r, R)$. Combining the two yields
        \begin{equation}
            \E = \frac{\rho \text{min}(r, R)^3}{3r^2\e}\,\hat{r}.
        \end{equation}
        For $r < R$, this simplifies to
        \begin{equation}
            \E = \frac{\rho r}{3\e}\,\hat{r}.
        \end{equation}

\subsection{Dielectrics}    
    An external electric field $\E_0$ induces an opposing field $\E_p$ in dielectric materials. 
    The resulting field inside the dielectric is $\E_{\epsilon_r}$. We define the \textit{relative dielectric constant} as follows: 
    \begin{equation}
        \epsilon_r = \frac{E_0}{E_{\epsilon_r}}.
    \end{equation}
    $\epsilon_r \geq 1$,  $\epsilon_r = 1$ in vacuum, and $\epsilon_r \approx 1$ in air.
    Sometimes, the \textit{electric susceptibility} is used instead:
    \begin{equation}
        \chi_e = \epsilon_r -1.
    \end{equation}
    Another notation used is 
    \begin{equation}
        \epsilon = \e\epsilon_r
    \end{equation}
    where $\epsilon$ is called the \textit{dielectric constant}.
    
    A parallel plate capacitor with surface charge density $\sigma_0$ induces a surface charge density $\sigma_p$ of opposite sign on both sides. 
    \begin{equation}
        \sigma_p = \frac{\e -1}{\epsilon_r}\,\sigma_0.
    \end{equation}
    $\sigma_p$ is called the \textit{polarisation charge density}.
    
\subsection{Polarisation vector}
    The induced field opposing an imposed field in a dielectric is the result of the microscopic structure of the material, 
    where each infinitesimal volume $\Delta \tau$ has an infinitesimal dipole momentum $\vec{\Delta p}$. 
    The total dipole momentum of the dielectric is then 
    \begin{equation}
        \vec{p} = N\left< \vec{p} \right>
    \end{equation}
    Where $N\cdot\Delta\tau$ is the total volume of the dielectric. We define the \textit{polarisation vector} as follows:
    \begin{equation}
        \p = \frac{\vec{\Delta p}}{\Delta \tau}.
    \end{equation}
    It can be shown that 
    \begin{equation}
        \label{P as surface density}
        \p = \sigma_p\,\hat{n} = \e \chi_e \E_{\epsilon_r}
    \end{equation}
    where $\E_{\epsilon_r}$ is the total field in the dielectric material.
    
    Applying the divergence theorem, we get
    \begin{equation}
        \nabla \cdot \p = - \rho_{\text{pol}}
    \end{equation}
    where $\rho_{\text{pol}}$ is the volume charge distribution of the polarisation charges.

    \subsubsection*{Microscopic motivation}
        Let $\E^*$ be the field "felt" by a single molecule, meaning the neighbouring molecules contribute 
        (as opposed to the macro-case, where the neighbours cancel out in the bulk). We then have
        \begin{equation}
            \label{polarisability}
            \p = n\left<\vec{p}\right> = n\alpha\E^*,
        \end{equation}
        where $\alpha$ is a proportionality constant named \textit{polarisability}, with units \si{\per\metre\cubed}. 
        A large polarisability then means that a molecule exerts a larger dipole momentum for a given electric field strength. 

        To get any useful mathematics out of this, imagine a infinitesimal spherical cavity in a dielectric. 
        $\E^*$ is clearly then the sum of $\E_{\epsilon_r}$ 
        and all infinitesimal contibutions due to the surface charge density $\sigma_p^*$ on the sphere ($\E_\text{int}$). 
        Due to $\E_{\epsilon_r}$, $\sigma_p^*$ is not uniform, since the molecules tend to align with the applied electric field. 
        The configuration is symmetric in the polar angle, depending only on the azimuthal angle $\theta$. 
        By combining equation \eqref{surface density integral} and the first half of equation \eqref{P as surface density}, we arrive at
        \begin{equation}
            \E_\text{int} = \iint \frac{dq}{4\pi\e} 
            = \int_0^{2\pi}\int_0^\pi\frac{|\p |}{4\pi\e}\cos^2\theta\sin\theta d\theta d\phi = \frac{|\p |}{3\e}.
        \end{equation}
        Inserting this into equation \eqref{polarisability}, and combining it with the second half of equation \eqref{P as surface density}, 
        we arrive at the \textit{Claussius- Mossotti formula}:
        \begin{equation}
            \frac{n\alpha}{3} = \frac{\chi_e}{\epsilon_r +2}, 
        \end{equation}
        where $n$ is the volume density of infinitesimal dipoles (molecules), $\alpha$ is the polarisability, 
        and $\epsilon_r$ is the relative permitivity of the dielectric material. 

        For low density materials like gases, $n << 1$, and we get the approximations
        \begin{align}
            \E^* &= \E \\
            \chi_e &= n\alpha.
        \end{align}

\subsection{Displacement field}
    From Gauss' law, we have 
    \begin{equation*}
        \nabla \cdot E = \frac{\rho_{\text{free}} + \rho_{\text{pol}}}{\e},
    \end{equation*}
    where we have separated the charge density distributions for the free charges and the polarisation charges. 
    Combining this with the divergence of the polarisation vector, we get 
    \begin{equation*}
        \nabla \cdot \left(\e\E + \p\right) = \rho_{\text{free}}.
    \end{equation*}
    We therefore define the \textit{displacement field} as follows:
    \begin{equation}
        \D = \e\E + \p.
    \end{equation}
    The following table compares $\D$ and $\p$:
    \begin{center}
        \begin{tabular}{ c|c } 
        Polarisation & Displacement \\
        \hline
         $\nabla \cdot \p = -\rho_{\text{pol}}$ & $\nabla \cdot \D = \rho_{\text{free}}$ \\
         $\p \cdot \hat{n} = \sigma_p$ & $\D \cdot \hat{n} = \sigma_p$
        \end{tabular}
    \end{center}

    \subsubsection{Continuity conditions at interfaces}
        It can be shown that:
        \begin{itemize}
            \item The tangential component of the $\E$- field is conserved over an interface
            \item The normal component of the $\D$-field is conserved over an interface
        \end{itemize}

\subsection{Energy of an electric field}
    To charge a capacitor with capacitance $C$ with a charge $q$, a work is required. This is given as 
    \begin{equation}
        W = \frac{q^2}{2C} = \frac{C(\Delta V)^2}{2} = \frac{q\Delta V}{2} = \frac{\e E^2S h}{2},
    \end{equation}
    where $S$ is the surface of the plates and $h$ is the separation distance. 
    So long as the electrical field is correctly considered, this equation is valid in dielectrics as well.
    
    Since $S h$ represent a volume, we define a density of energy: 
    \begin{equation}
        u_e = \frac{1}{2}\e E^2, 
    \end{equation}
    such that 
    \begin{equation}
        U_e = \iiint_\tau u_e d\tau
    \end{equation}
    
\subsection{Conductors}
    \subsubsection{Definitions and basic properties}
        Conductors are materials where charges can move, but the intrinsic net charge is zero. 
        Theoretical perfect conductors (all conductors discussed in this document are perfect unless otherwise specified) 
        have an infinite number of free charges. 
        An external electric field will therefore induce a charge distribution in a conductor, 
        such that there are a net negative charge on one side and a net positive charge on the other, 
        in the direction of $\E$. This is called \textit{electrostatic induction}.
        
        Conductors have zero electric field inside the bulk, 
        since any charges that can move will (almost) instantly move according to the external field, counteracting it. 
        
        When a charge $q$ is imposed on a conductor, the charges will lie on the surface bu the same argument. 
        Similarly, the electric field must be (locally) equal in magnitude (and normal to the surface). 
        This makes the surface of a conductor isopotential, which is also valid for conductors without an imposed charge. 

    \subsubsection{Total electrostatic induction}
        Total or complete electrostatic induction is when all field lines of an electric field pass through a conductor.
        
        An example is a point source inside a spherical conductor shell. 
        By the "rule" (my explanation is fairly hand-wavy, resulting in the quotation marks here. 
        The reader may rest assured that the result is provably true) of zero net electric field inside a conductor, 
        the induced electric field on the inner surface of the sphere shell is equal to the field from the source. 
        By conservation of charge in the conductor, 
        the corresponding induced field on the outside is equal to the field that would have been there if the conductive shell was removed. 
        This is a general property of complete electrostatic induction. 
  
\subsection{Current}
    Current is the amount of charges moving through a conductor during a time period. 
    \subsubsection{Current density vector}
        The amount charge moving through a surface $S$ during $\Delta t$ is given by
        \begin{equation}
            dq = nq\vec{v}\cdot \hat{r}\,S \,\Delta t
        \end{equation}
        where $n$ is the volume density of charged particles (unit \si{\per\metre\cubed}), and $q$ is the charge of each particle.
        Therefore, the current is given by
        \begin{equation}
            dI = \frac{dq}{\Delta t} = nq\vec{v}\cdot\hat{n}\, dS = \j\cdot\hat{n}\, dS
        \end{equation}
        where the surface $dS$ goes towards zero, and $\j$ is the current density vector, i.e.
        \begin{equation}
            \j = nq\vec{v}.
        \end{equation}
        
        Integrating over $S$ yields
        \begin{equation}
            I = \iint_S \j\cdot d\vec{S}.
        \end{equation}
        Taking the divergence of $\j$, we get
        \begin{equation}
            \nabla \cdot \j = \frac{d\rho}{dt}
        \end{equation}
        where $\rho$ is the volume charge distribution.
        In stationary conditions ($\j_{\text{in}} = \j_{\text{out}}$), this is equal to 0. 
    
        In an external field, charges propagate in a conductor with a \textit{drift velocity} of
        \begin{equation}
            \vec{v_d} = \frac{q\tau}{m}\,\E
        \end{equation}
        where $q$ is the charge of the charge, and $\tau$ is the mean time between collisions of the charges in the conductor, 
        and $m$ is the mass of the charge.
        
        Therefore, $\j$ in a conductor in an external electric field is
        \begin{equation}
            \j = nq\vec{v} = \frac{nq^2\tau}{m}\,\E.
        \end{equation}
        We name this constant \textit{conductivity}: 
        \begin{equation}
            \sigma = \frac{nq^2\tau}{m},
        \end{equation}
        and the \textit{resistivity}:
        \begin{equation}
            \rho = \frac{1}{\sigma},
        \end{equation}
        where $n$ is the density of charges, $q$ is the charge of a single charge, 
        $\tau$ is the mean time between the charges colliding in the conductor, and $m$ is the mass of each charge. 
        Although they use the same symbol, conductivity and resistivity are NOT to be confused with surface and volume charge distributions.
        
        We have the relations 
        \begin{align}
            \j &= \sigma \E \\
            \E &= \rho \j.
        \end{align}
    
    \subsubsection{Resistance}
        The resistance of a conductor is given as
        \begin{equation}
            R = \int_\mathcal{L} \frac{\rho}{S}\, dl,
        \end{equation}
        where $\rho$ is the resistivity, 
        $S$ is the cross-sectional area of the conductor $\mathcal{L}$ along $dl$. 
        Both $\rho$ and $S$ can change as a function of $l$, for example by changing the material and the girth of the conductor.
        
        Circuits with multiple resistors behave as follows: 
        \begin{align}
            R_{\text{tot}} &= \sum_i R_i\quad \text{(series)} \\
            \frac{1}{R_{\text{tot}}} &= \sum_i \frac{1}{R_i} \quad\text{(parallel)}
        \end{align}
        
    \subsubsection{Ohm's law}
        The potential difference $\Delta V$ over a conductor with resistance $R$ and current $I$ is
        \begin{equation}
            \Delta V = IR
        \end{equation}
        
    \subsubsection{Generators}
        Let $\E^*$ be the electric field responsible for the potential difference between the poles of a voltage source, such that
        \begin{equation}
            \oint_\mathcal{L} \E\cdot \vec{dl} = \int_+^-\E_{\text{circuit}}\cdot \vec{dl} + \int_-^+(\E_{\text{circuit}} + \E^*)\cdot \vec{dl} = \int_-^+\E^*\cdot \vec{dl} = \emf
        \end{equation}
        where $\emf$ is the \textit{electromotive force} (NOT a force, $\left[\emf\right]$ = \si\volt)
        
        Real voltage sources have an internal resistance:
        \begin{equation}
            r = \frac{1}{I}\int_-^+(\E + \E^*)\cdot\vec{dl}
        \end{equation}
        
    \subsubsection{Kirchoff's laws}
        Both laws presented are direction-dependent, meaning that the chosen direction will dictate the sign of the answer. 
        The result is the same regardless of the chosen direction, so long as it is used consistently (including for interpreting the result).
        \begin{equation}
            \sum_i I_i = 0
        \end{equation}
        in a node.
        \begin{equation}
            \sum_i R_iI_i = \sum_i \emf_i
        \end{equation}
        in a mesh, where $\emf$ are voltage sources. 

\subsection{Capacitors}
    \subsubsection{Definitions and basic properties}
        A capacitor is a configuration of two conductors resulting in total electrostatic induction. 
        The most common configuration is  two parallel plates.
        For most all calculations, we assume the plates to be infinitely large, such that there are no border effects. 
        This approximation is sufficiently good when the separation distance between the plates is significantly smaller than their area.
        
        We define capacitance as follows:
        \begin{equation}
            C = \frac{q}{\Delta V}
        \end{equation}
        where $q$ is the total charge of the capacitor and $\Delta V$ is the difference in potential between the conductors.
        
        Capacitance is measured in farad [F].
        
        Real capacitors are usually in a parallel plate configuration such that the plates are wound around each other in a spiral, 
        to increase the surface area while keeping the footprint of the component small.
        
        The capacitance of a capacitor with a dielectric material between the conductors is given by
        \begin{equation}
            C = \epsilon_rC_0
        \end{equation}
        where $\epsilon_r$ is the relative dielectric constant and $C_0$ is the theoretical capacitance 
        if the dielectric material had been vacuum. Since $\epsilon_r > 1$, the capacitance increases.
    
    \subsubsection{Parallel plate capacitors}
        For a given geometry of the plates, with a surface charge density $\sigma$ on one plate and $-\sigma$ on the other, 
        the electric field between the plates is given as 
        \begin{equation}
            \E = \frac{\sigma}{\e}\,\hat{n},
        \end{equation}
        where $\hat{n}$ is the normal vector of the positive plate towards the negative. This equation neglects border effects.
        
        The field outside the plates is zero.
        
        If the plates have an area $S$ and a separation distance $h << \sqrt{S}$, then the change in potential is
        \begin{equation}
            \Delta V = \frac{h\sigma}{\e}.
        \end{equation}
        Thus, the capacitance of a parallel plate capacitor is
        \begin{equation}
            C = \frac{q}{|\Delta V|} = \frac{\sigma S}{\frac{h\sigma}{\e}} = \frac{\e S}{h}.
        \end{equation}
        Note the independence of $\sigma$, i.e. the capacitance is a geometry-based property.

    \subsubsection{Cylindrical capacitors}
        If we instead consider a cylindrical capacitor, 
        where a cylindrical conductor with radius $R_1$ is placed inside a cylinder shell conductor with inner radius $R_2$, 
        both with height $h$ and constant linear charge density $\lambda$ along the height, then the capacitance is given by 
        \begin{equation}
            C = \frac{2\pi h \e}{\ln{\frac{R_2}{R_1}}}
        \end{equation}
        Naturally, if the gap between the cylinders is small ($R_2 - R_1 \approx 0$), 
        then the cylindrical capacitor is locally approximated by a parallel plate capacitor.

    \subsubsection{Charging and discarging}
        If an ideal (internal resistance $r=0$) voltage source $\emf$ is coupled in series with a resistor $R$ 
        and a capacitor $C$, where the capacitor charge is denoted $q$, then we get the following differential equation: 
        \begin{equation}
            R\frac{dq}{dt} = \emf - \frac{q}{C},
        \end{equation}
        which is solved by
        \begin{equation}
            q(t) = \emf C \left(1-\exp{\frac{-t}{RC}}\right).
        \end{equation}
        The energy stored in the capacitor is 
        \begin{equation}
            W_C = \frac{1}{2}C\emf^2,
        \end{equation}
        whereas the total energy "spent" by the circuit is \textit{twice as much}.
