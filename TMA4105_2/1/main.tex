\documentclass[11pt, a4paper, norsk]{NTNUoving}
\usepackage[utf8]{inputenc}
\usepackage[T1]{fontenc}

\ovingnr{1}    % Nummer på innlevering
\semester{Høsten 2019}
\fag{TMA 4105}
\institutt{Institutt for matematiske fag}


\begin{document}

% Kommentar

% Et felt starter ofte med \begin{<sett in kommando>}, da er det viktig å avslutte med \end{<sett in kommando>}. Det er mange eksempler på dette nedenfor!

% Du må alltid bruke $<sett inn matematikk>$, $$<sett inn matematikk>$$ eller \[<sett inn matematikk>\] for å bruke mattekommandoer.


%Dette er for enkel copy-pasting
\ifx
\begin{oppgave}
    \begin{punkt}
        \begin{align*}
        
        
        \end{align*}
    \end{punkt}
\end{oppgave}

\begin{oppgave}
    \begin{align*}
    
    
    \end{align*}
\end{oppgave}
\fi

%Her begynner dokumentet
%#####################################
\begin{oppgave}
    \begin{equation}
        z = 2 - x^2 -y^2
        \label{nr11}
    \end{equation}
    \begin{equation}
        z = x^2 -2x +y^2 -4y
        \label{nr12}
    \end{equation}
    Setter de to høyresidene lik hverandre:
    \begin{align*}
        2 - x^2 -y^2 &= x^2 -2x +y^2 -4y\\
        2 &= 2x^2 + 2x + 2y^2 + 4y\\
        1 &= x^2 + x + y^2 + 2y\\
        1 &= \left(x- \frac{1}{2}\right)^2 - \frac{1}{4} + (y-1)^2 -1\\
        \frac{9}{4} &= \left(x- \frac{1}{2}\right)^2 + (y-1)^2
    \end{align*}
    Dette er likningen for en sirkel med senter i $(\frac{1}{2}, 1)$ or radius lik $\frac{3}{2}$. Siden $\sin^2(t)+\cos^2(t) = 1$ for alle $t \in \mathbb{R}$ kan den parameteriseres slik:
    \begin{equation}
        \begin{cases}
            x=\frac{3}{2}\cos{t}+\frac{1}{2}\\
            y=\frac{3}{2}\sin{t}+1
        \end{cases}
        \label{nr13}
    \end{equation}
    For å finne en parameterisering for $z$ er det bare å sette inn $x$ og $y$ i en av de oppgitte likningene. Her velger jeg \eqref{nr11}:
    \begin{align*}
        z &= 2 - x^2 -y^2
        \\&= 2 - \left(\frac{3}{2}\cos{t}+\frac{1}{2}\right)^2 - \left(\frac{3}{2}\sin{t} + 1\right)^2
        \\&= 2 - \frac{9}{4}\cos{^2(t)} - \frac{9}{4}\sin{^2(t)}- \frac{3}{2}\cos{(t)} - 3\sin{(t)} - \frac{5}{4}
        \\&= -\frac{3}{2}\cos{(t)} - 3\sin{(t)} - \frac{3}{2}
    \end{align*}
    En parameterisering for skjæringskurven mellom kurvene \eqref{nr11} og \eqref{nr12} blir da:
    \begin{align*}
        \begin{cases}
            x=\frac{3}{2}\cos{t}+\frac{1}{2}\\
            y=\frac{3}{2}\sin{t}+1\\
            z= -\frac{3}{2}\cos{(t)} - 3\sin{(t)} - \frac{3}{2}
        \end{cases}
    \end{align*}
    
    For å se hva slags kjeglesnitt det blir projisert ned på $xy$-planet, kan en sette $z=0$. Da får en parameteriseringen \eqref{nr13}, som er en sirkel.
\end{oppgave}

\begin{oppgave}
    Hvert blad i kurven korresponderer til et toppunkt i $|sin(n\theta)|$, ettersom dette blir maksimal avstand fra origo. 
    \begin{align*}
        |sin(n\theta)| &=1\\
        n\theta &= \frac{\pi}{2} +k\pi \text{, } k\in\mathbb{Z}\\
        \theta &= \frac{\pi}{2n} +\frac{k\pi}{n}.
    \end{align*}
    For $\theta \in [0,2\pi]$ får vi da $2n$ løsninger, som da korresponderer til $2n$ blader.
    
    Arealet til kurven kan løses med formelen 
    \begin{equation}
        \frac{1}{2}\int_a^b r(\theta)^2d\theta.
        \label{areal_polar}
    \end{equation}
    Med  en absoluttverdi i funksjonen er det likevel enklere å finne arealet til ett blad, og å deretter multiplisere med antallet blader. Det gir
    \begin{align*}
        2n\cdot \frac{1}{2}\int_0^{\frac{2\pi}{2n}} \sin{(n\theta)}d\theta &=
        n\left[-\frac{1}{n} \cos{(n\theta)}\right]_0^{\frac{\pi}{n}}
        \\&= -\cos{n\frac{\pi}{n}} -(-\cos{0})
        \\&= -(-1) + 1 = 2
    \end{align*}
    Arealet kurven utspenner er altså lik 2. 
\end{oppgave}

\begin{oppgave}
    For at $f$ skal være kontinuerlig må $$\lim_{(x,y)\to(0,0)} f(x,y) = f(0,0) = c$$. 
    
    La $x = 0 +r\cos(\theta)$ og $y = 0+ r\sin(\theta))$. Da er
    \begin{align*}
        \lim_{(x,y)\to(0,0)} f(x,y) &= \lim_{r\to 0^{+}}\frac{2r^2\cos(\theta)\sin(\theta) + r^4\cos^4(\theta)}{r^2\cos^2(\theta)+r^2\sin^2(\theta)}
        \\&= \lim_{r\to 0^{+}}\frac{2r^2\cos(\theta)\sin(\theta) + r^4\cos^4(\theta)}{r^2}
        \\&=\lim_{r\to 0^{+}}2\cos(\theta)\sin(\theta) + r^2\cos^4(\theta)
        \\&=2\cos(\theta)\sin(\theta)
    \end{align*}
    Dette er ikke uavhengig av $\theta$, og dermed eksisterer ikke grenseverdien. Det finnes altså ikke et tall $c$ slik at $f$ er kontinuerlig. 
\end{oppgave}

\begin{oppgave}
    En likning til tangentplanet til grafen en funksjon $f$ i punktet $(a,b, f(a,b))$ er gitt ved   
    \begin{align*}
      z = f(a,b) + \frac{\partial f}{\partial x}(a,b)\cdot(x-a) + \frac{\partial f}{\partial y}(a,b)\cdot(y-b).
    \end{align*}
    For $f(x,y) = \cos^2(x)+\sin(x)\cos(y)$ og punktet $(\frac{\pi}{4}, \frac{\pi}{4}, 1)$, må vi først undersøke om punktet ligger på grafen. 
    \begin{align*}
        f\left(\frac{\pi}{4}, \frac{\pi}{4}\right) &= \cos^2\left(\frac{\pi}{4}\right) + \sin\left(\frac{\pi}{4}\right)\cos\left(\frac{\pi}{4}\right)
        \\&= \left(\frac{\sqrt{2}}{2}\right)^2 + \frac{\sqrt{2}}{2} \cdot \frac{\sqrt{2}}{2}
        \\&=\frac{2}{4}+\frac{2}{4} = 1
    \end{align*}
    Punktet ligger altså på grafen. 
    \begin{align*}
        \frac{\partial f}{\partial x}(a,b) &= -2\sin(a)\cos(a) + \cos(a)\cos(b)
        \\&= -2\sin\left(\frac{\pi}{4}\right)\cos\left(\frac{\pi}{4}\right) + \cos\left(\frac{\pi}{4}\right)\cos\left(\frac{\pi}{4}\right)
        \\&= -2\frac {\sqrt{2}}{2}\cdot \frac{\sqrt{2}}{2} + \frac{\sqrt{2}}{2} \cdot \frac{\sqrt{2}}{2}
        \\&= -\frac{1}{2}
    \end{align*}
    \begin{align*}
        \frac{\partial f}{\partial y}(a,b) &= 0 -\sin(a)\sin(b)
        \\&= -\frac{\sqrt{2}}{2} \cdot \frac{\sqrt{2}}{2}
        \\&= -\frac{1}{2}
    \end{align*}
    Innsatt i formelen gir dette $z=1-\frac{1}{2}(x-\frac{\pi}{4}) -\frac{1}{2}(y-\frac{\pi}{4})$.
\end{oppgave}
\end{document}