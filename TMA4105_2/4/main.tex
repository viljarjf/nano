\documentclass[11pt, a4paper, norsk]{NTNUoving}
\usepackage[utf8]{inputenc}
\usepackage[T1]{fontenc}
\usepackage{tikz}

\ovingnr{4}    % Nummer på innlevering
\semester{Våren 2020}
\fag{TMA 4105} %4105 er matte 2, 4115 er matte 3
\institutt{Institutt for matematiske fag}

\newenvironment{pkt}{\begin{punkt}}{\end{punkt}}

\newenvironment{matrise}[1][]{
        \left[
            \begin{array}{#1}
    }
    {    
    \end{array}
    \right]           
}

\newenvironment{indreprod}{
    \langle
}{
\rangle}

\newcommand{\R}{\mathbb{R}}
\renewcommand{\P}{\mathcal{P}}

\begin{document}
%#################################################
%Dette er for enkel copy-pasting
\ifx
%-------------------------
\begin{oppgave}
    \begin{punkt}
        \begin{align*}
        
        
        \end{align*}
    \end{punkt}
\end{oppgave}
%-------------------------
\begin{tikzpicture}
        \draw[step=1cm,gray,very thin] (-5.9,-5.9) grid (5.9,5.9);
        \draw (-6,0) -- (6,0);
        \draw (0,-6) -- (0,6);
        \draw[->] (-6,0) -- (6,0); 
        \draw[->] (0,-6) -- (0,6);
        \draw (0, 6.2) node {y};
        \draw (6.2, 0) node {x};
\end{tikzpicture}
\fi

%Her begynner dokumentet
%#####################################
\begin{oppgave}
    $$\nabla \cdot F = \frac{\partial}{\partial x}xe^{2z}+\frac{\partial}{\partial y}ye^{2z}-\frac{\partial}{\partial z}e^{2z} = e^{2x}+e^{2x}-2e^{2x}=0$$.
    
    $$F = \nabla \times G = \left(\frac{\partial G_3}{\partial y} - \frac{\partial G_2}{\partial z}\right)\textbf{i}+ \left(\frac{\partial G_1}{\partial z} - \frac{\partial G_3}{\partial x}\right)\textbf{j} + \left(\frac{\partial G_2}{\partial x} - \frac{\partial G_1}{\partial y}\right)\textbf{k}$$.

    \begin{align*}
        xe^{2z} &= \frac{\partial G_3}{\partial y} - \frac{\partial G_2}{\partial z}\\
        ye^{2z} &= \frac{\partial G_1}{\partial z} - \frac{\partial G_3}{\partial x}\\
        -e^{2z} &= \frac{\partial G_2}{\partial x} - \frac{\partial G_1}{\partial y}
    \end{align*}
    La $G_2=0$. Det gir 
    \begin{align*}
        xe^{2z} &= \frac{\partial G_3}{\partial y} &G_3=xye^{2z} + f(x, z)\\
        ye^{2z} &= \frac{\partial G_1}{\partial z} - \frac{\partial G_3}{\partial x}\\
        -e^{2z} &=   -\frac{\partial G_1}{\partial y} &G_1=ye^{2z}+g(x, z)
    \end{align*}
    La $f(x,z)=0$. Det gir
    \begin{align*}
            ye^{2z} &= \frac{\partial}{\partial z}\left(ye^{2z}+g(x, z)\right) - \frac{\partial}{\partial x}xye^{2z}\\
            &= 2ye^{2z}+\frac{\partial g(x, z)}{\partial z} -ye^{2z}\\
            \frac{\partial g(x, z)}{\partial z} &= 0
    \end{align*}
    Det gir $g(x, z)=g(x)$. La $g(x)=0$. Da har vi $\textbf{G}(x, y, z)=\left[ye^{2z},\, 0,\,  xye^{2z}\right]$. Siden $\nabla \times G = \nabla \times (G+\nabla\phi)\,\forall\,\phi: \R^n \rightarrow \R$ dersom $\phi$ er glatt, vil $\textbf{H}(x, y, z)=\left[ye^{2z}+1,\, 1,\,  xye^{2z}+1\right]$ være en løsning.
    
\end{oppgave}
\begin{oppgave}
    Greens teorem sier at $\oint_\mathcal{C}\textbf{F}\cdot d\textbf{r}=\iint_R\left(\frac{\partial g}{\partial x}-\frac{\partial f}{\partial y}\right)dA$, der $\textbf{F}(x,y)=[f(x,y),\,g(x,y)]$, dersom visse forutsetninger ved området er oppfylt. 
    
    \begin{align*}
        \iint_R\left(\frac{\partial }{\partial x}(\cos x - xy)-\frac{\partial }{\partial y}(\sin x + y^2)\right)dA &= \int_{-2}^2\int_0^{4-x^2}-\sin x - y - (0+2y)dydx
        \\&=-\int_{-2}^2\left[y\sin x +\frac{3}{2}y^2\right]_0^{4-x^2}dx
        \\&=-\int_{-2}^2(4-x^2)\sin x +\frac{3}{2}(4-x^2)^2dx.
    \end{align*}
    
    Av symmetrigrunner er dette lik
    \begin{align*}
        -\int_{-2}^2 \frac{3}{2}(4-x^2)^2dx &= -\frac{3}{2}\int_{-2}^2 16 - 8x^2 + x^4dx
        \\&= -\frac{3}{2}\left[16x-\frac{8}{3}x^3+\frac{1}{5}x^5\right]_{-2}^2
        \\&= -\frac{3}{2}\left(-\frac{8}{3}8+\frac{1}{5}32-\left(\frac{8}{3}8-\frac{1}{5}32\right)\right)
        \\&=-\frac{256}{5}
    \end{align*}
    
\end{oppgave}
\begin{oppgave}
    \begin{pkt}
        \begin{align*}
            \iiint_TdV &= \int_3^4\int_0^{2\pi}\int_{0}^{\sqrt{4-z}}rdrd\theta dz
            \\&=\pi\int_3^4 4-zdz
            \\&=\pi\left[4z-\frac{1}{2}z^2\right]_3^4
            \\&=\pi\left(16-\frac{16}{2}-\left(12-\frac{9}{2}\right)\right)
            \\&=\frac{\pi}{2}
        \end{align*}
    \end{pkt}
    \begin{pkt}
    La $R$ være sirkelskiven som lukker $T$. Fluksintegralet av $\mathcal{S}$ er da
    $$\iiint_T\nabla\cdot \textbf{F}dV-\iint_R \textbf{F}\cdot \hat{\textbf{N}}dR$$.
        \begin{align*}
             \iint_R \textbf{F}\cdot \hat{\textbf{N}}dR &= \iint_R -3dR
             \\&=-3\pi,
        \end{align*}
        ettersom $R$ er en sirkel med radius 1.
        \begin{align*}
            \iiint_T\nabla\cdot \textbf{F}dV &= \iiint_T -2xy+2yx+1dV
            \\&= \iiint_TdV
            \\&=\frac{\pi}{2}
        \end{align*}
        Det gir at fluksintegralet av \textbf{F} over $\mathcal{S}$ er $\frac{7\pi}{2}$.
    \end{pkt}
\end{oppgave}
\begin{oppgave}
    $\mathcal{C}$ er en sirkel med sentrum i (0, 0, $\sqrt{10}$), og radius 3. En parameterisering av kurven er da $r(t)=(x(t)=3\cos(t),\, y(t)=3\sin(t),\,z(t)=\sqrt{10}$).
    \begin{align*}
        \oint_\mathcal{C}\textbf{F}\cdot d\textbf{r} &= \int_0^{2\pi}\textbf{F}(r(t))\cdotr'(t)dt
        \\&=\int_0^{2\pi}\left[3\cos(t)+3\sin(t),\, 12\cos(t)-3\sin(t),\, 10+9\cos(t)\sin(t)\right]
        \\&\cdot \left[-3\sin(t),\, 3\cos(t),\,0\right] dt
        \\&=\int_0^{2\pi} -9\sin(t)\cos(t)-9\sin^2(t)+36\cos^2(t)-9\sin(t)\cos(t)dt
    \end{align*}
    Av symmetrigrunner er dette lik 
    \begin{align*}
        \int_0^{2\pi} -9\sin^2(t)+36\cos^2(t)dt&=\int_0^{2\pi} -\frac{9}{2}\sin(2t)-\frac{9}{2}+18\cos(2t)+18dt
        \\&=36\pi-9\pi = 27\pi
    \end{align*}
\end{oppgave}
\end{document}